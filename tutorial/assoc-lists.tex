% vim: ft=tex ff=unix ts=4 sw=4 et wm=8 tw=0

\chapter{Association Lists}

Association lists are dictionary structures defined simply as lists of
@Key - Value@ pairs.  The standard library module @assoc_list@ provides the
corresponding type definition and a suite of utility predicates.  For
most applications a @map@ is usually a more appropriate choice, although
an @assoc_list@ may be more convenient in some cases.

Association lists have $O(1)$ for insertion and updates and $O(n)$ for
searching and deletion.

\section{Definition}

\begin{verbatim}
:- type assoc_list(K, V) == list(pair(K, V)).
\end{verbatim}
where the type @pair/2@ is taken from @std_util@.

\section{Searching and Changing}

Updates to an @assoc_list@ are made by simply consing @Key - Value@
pairs to the @list@.  If more than one @Key - Value@ pair with the same
@key@ is present in an @assoc_list@, only the first such is significant.

The @search@ predicate unifies its @V@ argument with the right hand side
of the first @pair@ in the @list@ whose left hand side unifies with its
@K@ argument and fails if there is no such pair.
\begin{verbatim}
:- pred search(assoc_list(K, V), K,  V  ).
:- mode search(in,               in, out) is semidet.
\end{verbatim}

The @remove@ predicate acts similarly, except that it also returns the
@assoc_list@ with the @Key - Value@ pair removed.
\begin{verbatim}
:- pred remove(assoc_list(K, V), K,  V,   assoc_list(K, V)).
:- mode remove(in,               in, out, out             ) is semidet.
\end{verbatim}
Note that if there is more than one @Key - Value@ pair in the list for
the same @Key@ then only the first such will be removed by 
@remove(AList, Key)@, which means there will still be a mapping for
@Key@ in the resulting @assoc_list@.  \XXX{Using a closure here
shouldn't upset anyone...}

\section{Utility Functions}

\begin{verbatim}
:- func reverse_members(assoc_list(K, V)) = assoc_list(V, K).

:- func from_corresponding_lists(list(K), list(V)) = assoc_list(K, V).

:- func keys(assoc_list(K, V)) = list(K).

:- func values(assoc_list(K, V)) = list(V).
\end{verbatim}
The same caveats apply to the utility functions if an @assoc_list@
contains more than one @Key - Value@ pair for a given @Key@.  In this
case, @keys@ will contain duplicate entries and @values@ may include items
that are not accessible via the @search@ predicate.

@reverse_members@ preserves the order of the members and, if the same
@Value@ appears in more than one @Key - Value@ pair, then @search@ will
only see the first such in the output from @reverse_members@.  \XXX{This
is not actually part of the spec.  Tighten up the documentation in
@assoc\_list.m@.}

@from_corresponding_lists@ will throw an exception if its list arguments
differ in length.  The order of keys and values is preserved in the
resulting @assoc_list@.

\XXX{Spread some examples around.}
