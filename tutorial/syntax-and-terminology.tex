% vim: ft=tex ff=unix ts=4 sw=4 et wm=8 tw=0

\chapter{Basic Syntax and Terminology}



\section{Introduction}

Describing Mercury takes a number of terms that are well understood
in the declarative programming community, but are rarely, if ever, used
in the more common imperative schools.  This chapter introduces the bulk
of the terms the reader may not be familiar with, but which occurr
frequently throughout the rest of this book.  The intention is to
provide a ``gloss'' to aid comprehension of what follows; more full
explanations will be given as the terms are used formally for the first
time.



\section{Declarative and Imperative}

Languages like C, C++, Java and so forth are all \emph{imperative}
programming languages.

Imperative programming considers a program to be a \emph{sequence of
instructions}.  Run-time control-flow decisions are made on the basis of
the contents of the \emph{mutable variables} that represent the
\emph{state} of the program.  Instructions fall into three categories:
those that update mutable variables with the result of some computation;
those that change the control flow of the program depending upon the
state of those variables; and those that perform input/output (IO)
operations.

Languages like Mercury and Haskell are \emph{declarative} programming
languages.

Declarative programming considers a program to be a \emph{computable
relationship} between its inputs and outputs.  There are no mutable
variables in declarative programming; the term \emph{variable} instead
refers to the name given to the result of some computation.  Some
relationships are simple enough to be computed directly, such as
@X = Y + Z@ where any two of the values are known.  Other relationships
have to be expressed by the programmer in terms of combinations of these
very basic relationships.

The ``moral'' motivation behind declarative programming is that the
resulting program is merely a refinement of the statement of the
problem, whereas it is actually very difficult to say exactly what
relationship a program is computing, making it harder to decide whether
the program contains a bug or not.

The practical motivation is that you generally get more bang for your
buck in terms of progress made per line of code written.  Moreover,
because declarative programs are essentially just statements of
mathematics, compilers can and do perform extensive checking and
verification tests on the code.  A large class of bugs that are part of
the day-to-day life of the imperative programmer simply cannot be
expressed in a declarative language; those that can be are usually
caught by the compiler, well before one needs to fire up the debugger.

\XXX{I think I'm selling this too hard.  The speech should probably go
in the introduction.}



\section{Predicates, Modes and Determinism}

In logic, a \emph{predicate} is simply the name given to a relationship
between some number of variables.  For example, @father(C, F)@ might
represent the relationship of @F@ being the father of @C@.  @C@ and @F@
are said to be the \emph{parameters} of @father@.

Given any bindings for @F@ and @C@ we can ask whether @father(C, F)@ is
true or not.  Moreover, we may be able to give the bindings for only a
subset of the variables.  Say we fix @F@, we can ask what the possible
bindings for @C@ are that would make @father(C, F)@ true.  Conversely,
we might ask what the possible values of @F@ are for a given @C@.
Indeed, we may go one step further and specify \emph{neither} @F@ nor
@C@ and simply ask for all possible pairwise bindings with the
appropriate property.

We use the term \emph{determinism} to refer to the number of possible
solutions a predicate can have in a particular \emph{mode} (i.e. given
fixed values for certain the variables.)  A predicate is
\emph{deterministic} if there is exactly one solution (set of bindings)
for the remaining variables, otherwise it is \emph{nondeterministic}.
(We will see that in Mercury it is useful to further subdivide the
determinism categories.)

Informally, we refer to the arguments with given values as inputs and
those whose value is not fixed as outputs.

For example, @father@ is deterministic if we fix only @C@ since every child
has exactly one father.  However, if we fix only @F@ we may get any
number of possible bindings for @C@ depending upon how many children @F@
has had, hence this mode of @father@ is nondeterministic.

We use the term ``procedure'' (or sometimes overload ``mode'') to refer
to a predicate restricted to a given set of argument modes.

\subsection{Syntax}

Mercury understands several determinisms: @erroneous@, @semidet@, @det@,
@nondet@, @multi@, @cc_nondet@ and @cc_multi@.  These will be explained
in detail in \XXX{}; for now it suffices to know that @semidet@ and
@nondet@ can fail (i.e. have no solutions), @semidet@ may have a single
solution, @det@ has exactly one solution (i.e. it describes a
\emph{functional} relationship -- see below), and @nondet@ and @multi@
may have more than one solution.

The two most important built-in modes are @in@ and @out@, corresponding
to input and output arguments respectively.  Modes are discussed in
\XXX{}.

The mode declarations (one per procedure) for the @father@ predicate
described above would look like this:
\begin{verbatim}
    % father(Child, Father).
    %
:- mode father(in,  out) is det.
:- mode father(out, in ) is nondet.
:- mode father(in,  in ) is semidet.
\end{verbatim}

\XXX{I'm not sure how/when/where to introduce syntax in this chapter.
It may be best to merge this chapter with the main Mercury-as-it-is-
written chapter.}



\section{Functions and Expressions}

A deterministic procedure is said to describe a \emph{functional}
relationship between its inputs and its outputs (that is, the output is
uniquely determined for each possible input.)  For instance, the
sum @X = Y + Z@ is functional given any two of @X@, @Y@ or @Z@.

Special syntax exists for functions with a single (last) output
argument.  The main reason for this is that it supports \emph{functional
composition}, which allows us to avoid naming intermediate results.  For
example, rather than writing
\begin{verbatim}
    Tmp1 = A    + B,
    Tmp2 = Tmp1 - C,
    Tmp3 = Tmp2 * D,
    X    = Tmp3 / E
\end{verbatim}
we can simply write
\begin{verbatim}
    X = (((A + B) - C) * D) / E
\end{verbatim}
(in fact the precedence and associativity of the operators @+@, @-@,
@*@, @/@ etc. has been chosen so that parentheses can often be
avoided.)

An \emph{expression} describes either a simple value or the
\emph{application} of a function to other expressions.



\section{Goals}

A goal describes either a simple \emph{predicate call} or XXX HERE!



Procedures.

Goals.

Terms.

Variables.

Quantifiers.

Conjunction, disjunction, negation, conditional.

...



