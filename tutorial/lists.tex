\section{Lists}

Lists are perhaps the single most useful data structure in the
programmer's armoury.

The @list@ module in the Mercury standard library defines lists as
follows:
\begin{verbatim}
:- type list(T) ---> []
                ;    [T | list(T)].
\end{verbatim}
The notation @[A | B]@ is special syntactic sugar recognised by the
Mercury parser for @[|](A, B)@ -- that is, @[|]/2@ is the basic list
data constructor, with @[]@ standing for the empty list.

Moreover, an expression of the form @[A, B, C]@ is syntactic sugar for
@[A | [B | [C | []]]]@ which, of course, is identical to
@[|](A, [|](B, [|](C, [])))@.

Also, an expression of the form @[A, B, C | Xs]@ is syntactic sugar for
@[A | [B | [C | Xs]]]@ which, of course, is identical to
@[|](A, [|](B, [|](C, Xs)))@.

These are obviously singly linked lists.  \XXX{Discussion of the pros
and cons of the various list ADTs out there.}

\subsection{The Main List Operations}

The higher order functions @map@, @foldl@ and @foldr@ provide easy ways
of iterating over lists, either transforming them member by member or
somehow accumulating the result of processing each member in turn.

The section on higher order programming \XXX{} has already dealt with
@map@, @foldl@ and @foldr@ in some detail, so we merely summarise that
discussion here and concentrate on operations we have not previously
examined.

\subsubsection{Miscellany}

\begin{verbatim}
    % length([X1, X2, X3, ..., XN]) = N
    %
:- func length(list(T)) = int.

length(Xs) = foldl((func(_, N) = N + 1), Xs, 0).

    % reverse([X1, X2, X3, ..., XN]) = [XN, ..., X3, X2, X1]
    %
:- func reverse(list(T)) = list(T).

reverse(Xs) = foldl((func(X, Ys) = [X | Ys]), Xs, []).
\end{verbatim}

\subsubsection{Membership}

\XXX{Should have a section on equality and @compare@ and so forth.}

@member@ is used to decide membership of a @list@ under equality and to
non-deterministically project members from a @list@ (the argument ordering
is arguably unfortunate for higher order programming, but this is
historically how things have been done):
\begin{verbatim}
    % member(X, Xs) iff X is a member of Xs.
    %
:- pred member(T,   list(T)).
:- mode member(in,  in     ) is semidet.
:- mode member(out, in     ) is nondet.

member(X, [X | _ ]).
member(X, [_ | Xs]) :- member(X, Xs).
\end{verbatim}
From time to time one wants to access members of a @list@ by their index
(\ie distance from the start of the @list@).  There are two sets of
operations for doing so, depending upon whether it is most convenient to
give the head of the list an index of 1 or 0:
\begin{verbatim}
    % index0(Xs, I, X) iff 0 =< I < length(Xs) and
    % X is the I+1th member of Xs.  That is,
    % index0([X1, X2, X3], 1, X2) while
    % index0([X1, X2, X3], 3, _ ) fails.
    %
:- pred index0(list(T), int, T  ).
:- mode index0(in,      in,  out) is semidet.

    % index1(Xs, I, X) iff 1 =< I =< length(Xs) and
    % X is the Ith member of Xs.  That is
    % index1([X1, X2, X3], 1, X1) while
    % index1([X1, X2, X3], 3, _ ) fails.
    %
:- pred index1(list(T), int, T  ).
:- mode index1(in,      in,  out) is semidet.

    % These functions correspond to the predicates above, but differ
    % in that an exception is thrown if the index is out of range.
    %
:- func index0(list(T), int) = T.
:- func index1(list(T), int) = T.
\end{verbatim}

\subsubsection{Mapping}

@map@ applies its function argument to each member of its @list@ argument
and returns the corresponding @list@ of results.
\begin{verbatim}
    % map(F, [X1, X2, X3, ...]) = [F(X1), F(X2), F(X3), ...]
    %
:- func map(func(T1) = T2, list(T1)) = list(T2).

map(_, []      ) = [].
map(F, [X | Xs]) = [F(X) | map(F, Xs)].
\end{verbatim}

A related function is @map_corresponding@ which is used to map a
function combining the corresponding members of \emph{two} lists
(@map_corresponding@ will throw an exception if the lists are of
different lengths.)
\begin{verbatim}
    % map_corresponding(F, [X1, X2, X3, ...], [Y1, Y2, Y3, ...]) =
    %       [F(X1, Y1), F(X2, Y2), F(X3, Y3), ...]
    %
:- func map_corresponding(func(T1, T2) = T3, list(T1), list(T2)) =
            list(T3).

map_corresponding(_, [],       []      ) = [].
map_corresponding(_, [],       [_ | _ ]) = <<throw exception>>.
map_corresponding(_, [_ | _ ], []      ) = <<throw exception>>.
map_corresponding(F, [X | Xs], [Y | Ys]) =
    [F(X, Y) | map_corresponding(F, Xs, Ys)].
\end{verbatim}

There is also a @map_corresponding3@ which works for functions of three
arguments:
\begin{verbatim}
:- func map_corresponding3(func(T1, T2, T3) = T4,
            list(T1), list(T2), list(T3)) = list(T4).
\end{verbatim}

\XXX{Should probably include a subsubsection on zipping and
interleaving.}

\subsubsection{Folding}

The two main folding operations are @foldl@ and @foldr@.  We have
already seen a definition of @foldr@ (the version supplied in the
Mercury standard library unfortunately uses a slightly different
argument ordering):
\begin{verbatim}
    % foldr(F, [X1, X2, X3], A) = F(X1, F(X2, F(X3, A)))
    %
:- func foldr(func(T1, T2) = T2, list(T1), T2) = T2.

foldr(_, [], A) = A.
foldr(F, [X | Xs], A) = F(X, foldr(Xs, A)).
\end{verbatim}
In many situations the function @F@ will be commutative or we will want
to process the @list@ starting with the leftmost member, in which case
it is more efficient to use the tail recursive @foldl@:
\begin{verbatim}
    % foldl(F, [X1, X2, X3], A) = F(X3, F(X2, F(X1, A)))
    %
:- func foldl(func(T1, T2) = T2, list(T1), T2) = T2.

foldl(_, [],       A) = A.
foldl(F, [X | Xs], A) = foldl(F, Xs, F(X, A))).
\end{verbatim}
As an example, here's how we could define the @reverse@ function, as
well as more efficient versions of the @sum@ and @prod@ functions
introduced in the section on higher order programming \XXX{}:
\begin{verbatim}
reverse(Xs) = foldl((func(X, Ys) = [X | Ys]), Xs, []).
sum(Xs)     = foldl((func(X, A ) = X + A   ), Xs, 0 ).
prod(Xs)    = foldl((func(X, A ) = X * A   ), Xs, 1 ).
\end{verbatim}

\subsubsection{XXX More To Come}

\subsection{General Advice}

\XXX{This probably deserves its own top-level section.}

While lists are easy to understand and work with, most programmers show
an unfortunate tendency to use lists when another data structure may be
more appropriate.  It is worth spending some time looking at the various
types provided by the Mercury standard library to see what is available.
The decision as to which data structure is best for a given situation is
one that can only be made in the light of experience, although as a rule
of thumb you cannot go far wrong by picking the data structure with the
most useful set of support functions for the problem in hand
(occasionally an @assoc_list@ may be preferable to a @map@, but in most
situations it won't be.)

