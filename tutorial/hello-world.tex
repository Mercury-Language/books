\chapter{Hello, World!}

Some example programs.

\section{Hello, World!}

Because it's traditional...  Type the following into a file called
@hello.m@:
\begin{verbatim}
:- module hello.

:- interface.
:- import_module io.

:- pred main(io, io).
:- mode main(di, uo) is det.

:- implementation.
:- import_module string.

    % The show starts here.
    %
main(IO0, IO) :-
    io__print("Hello, world!\n", IO0, IO).
\end{verbatim}
Compile and run the program with
\begin{verbatim}
$ mmc --make hello
$ ./hello
Hello, world!
\end{verbatim}
We'll start by just listing some of the salient points illustrated
by the ``Hello, world!'' program.
\begin{itemize}
\item Modules live in files with the same name (with a @.m@ suffix).
\item Every module starts with a declaration giving its name.
\item Non-code directives and declarations are introduced with @:-@.
\item Every declaration ends with a full stop.
\item Modules are divided into interface and implementation sections.
\item The interface section lists the things that are exported by the
  module.  The top-level module in a Mercury program must export a
  predicate @main/2@ (functors are conventionally referred to with
  @name/arity@).
\item The implementation section provides the code for the functions
  and predicates exported in interface section.
\item A module has to be imported before the things it defines can be
  used.  Hence @io__print@ refers to the print predicate defined in
  the io module.
\item The basic computational device in Mercury is the predicate.
  Predicates have type signatures and mode signatures -- the latter
  specifying which arguments are inputs and which are outputs.
\item Comments start with a @%@ sign and extend to the end of the line.
\end{itemize}

(IO is handled in Mercury by explicitly passing around the ``state
of the world''" -- each IO operation takes the current state and
produces a new one; the old state becoming unavailable for use
thereafter.  This might sound odd, but is a necessary part of
keeping Mercury a pure language.  \XXX{This will be explained in
more detail in the section on declaration IO.})

Special syntax exists to simplify passing state variable pairs
around; for ``Hello, world!'' this becomes
\begin{verbatim}
main(!IO) :-
    io__print("Hello, world!\n", !IO).
\end{verbatim}

