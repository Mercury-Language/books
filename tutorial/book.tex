%------------------------------------------------------------------------------%
% book.tex
% Ralph Becket <rafe@cs.mu.oz.au>
% Mon Jul 15 12:11:53 EST 2002
% vim: ft=tex ff=unix ts=4 sw=4 et wm=8 tw=0
%
%------------------------------------------------------------------------------%



%- Preamble -------------------------------------------------------------------%

\documentclass[a4paper,11pt,notitlepage,onecolumn]{book}
    %
    % [options]
    %   (10|11|12)pt            -- (default 10pt)
    %   (a4|letter)paper        -- (default letterpaper)
    %   fleqn                   -- (default centred) left-align formulae as
    %   leqno                   -- (default right) number formulae on the left
    %   [no]titlepage           -- [do not] start new page after the title
    %   (one|two)column         -- (default onecolumn)
    %   (one|two)side
    %   open(right|any)         -- open chapters on (right|any) pages only
    % {class}
    %   (article|report|book|slides)
    %                           -- consider FoilTeX instead of slides

\usepackage{doc}              % -- I want \MakeShortVerb
\usepackage{amsmath}          % -- I want align*
\usepackage{verbatim}         % -- It's better than the default, apparently
\usepackage{epsfig}           % -- I want \epsfbox
\usepackage{multirow}         % -- I want \multirow for tables

\pagestyle{headings}
    %
    % {style}
    %   (plain|headings|empty)  -- (default plain)
    %                           -- use \thispagestyle{} for new styles

%- Definitions and Customization ----------------------------------------------%

    % \newenvironment{name}[numargs]{begincmds}{endcmds}

    % \newcommand{name}[numargs]{definition}
    %                           -- (default numargs 0)
    %                           -- refer to args as #1, #2 etc. in definition
    %                           -- use % at EOL if definition not finished
    %                           -- space after a cmd is ignored unless
    %                           --  preceded by {}

\newcommand{\eg} {e.g.\@ }
\newcommand{\ie} {i.e.\@ }
\newcommand{\XXX}[1] {{\small\textbf{XXX} \emph{#1}}}

\newcommand{\Aside}[1]%
{{\small{\begin{description}\item{\textbf{Aside:}} #1\end{description}}}}

\newcommand{\BoxedPar}[1]%
{{\center{\fbox{\parbox{\linewidth}{#1}}}}}

\newcommand{\True} {\top}
\newcommand{\False} {\perp}
\newcommand{\Not}[1] {\neg{}#1}
\newcommand{\Conj} {\wedge}
\newcommand{\Disj} {\vee}
\newcommand{\Imp} {\Rightarrow}
\newcommand{\Bimp} {\Leftarrow}
\newcommand{\Eqv} {\Leftrightarrow}
\newcommand{\All}[2] {\forall\ #1.\ #2}
\newcommand{\Some}[2] {\exists\ #1.\ #2}

\newcommand{\Union} {\cup}
\newcommand{\Intersection} {\cap}
\newcommand{\Excluding} {\backslash}

\newcommand{\Odd} {\text{odd}}
\newcommand{\FV} {\text{FV}}
\newcommand{\Wolf} {\text{wolf}}
\newcommand{\Fox} {\text{fox}}
\newcommand{\Bird} {\text{bird}}
\newcommand{\Caterpillar} {\text{caterpillar}}
\newcommand{\Snail} {\text{snail}}
\newcommand{\Animal} {\text{animal}}
\newcommand{\Herbivorous} {\text{herbivorous}}
\newcommand{\Carnivorous} {\text{carnivorous}}
\newcommand{\Eats} {\text{eats}}
\newcommand{\Plant} {\text{plant}}
\newcommand{\Grain} {\text{grain}}
\newcommand{\BiggerThan} {\text{bigger-than}}


%- Start of Document ----------------------------------------------------------%

\begin{document}

\setlength{\parindent}{0pt}
\setlength{\parskip}{\baselineskip}
% \setlength{\hoffset}{}        % -- left margin is this + 1in
% \setlength{\voffset}{}        % -- top margin is this + 1in
% \setlength{\textheight}{}
% \setlength{\textwidth}{}
% \setlength{\marginparwidth}{}
    %
    % -- or can use \addtolength{parameter}{length}
    % -- or can use \settoheight{parameter}{text}
    % -- or can use \settodepth{parameter}{text}
    % -- or can use \settowidth{parameter}{text}

\title{The Art of Mercury}
\author{Ralph Becket \\ \texttt{\small rafe@cs.mu.oz.au}}
\date{15 October 2001}

% \maketitle

% \begin{abstract}
%
% ...
%
% \end{abstract}

%\tableofcontents

%- Body -----------------------------------------------------------------------%

\MakeShortVerb{\@}            % -- @...@ is now shorthand for \verb@...@
\MakeShortVerb{\#}            % -- #...# is now shorthand for \verb#...#

    % \text(rm|tt|sf|bf|it|sc|sl|up|md|normal){} \emph{}

    % \math(rm|tt|sf|bf|it|sc|sl|up|md|normal){}

    % \tiny \scriptsize \footnotesize \small \normalsize
    % \large \Large \LARGE \huge \Huge
    %                           -- also work as environments

    % ~                         -- small, fixed, nonbreaking space
    % \hspace{size}             -- soft space (may be lost at SOL or EOL)
    % \hspace*{size}            -- hard space
    %                           -- size may be \stretch{factor}
    % \hfill                    -- same as \hspace{\fill}

    % \vspace{size}             -- soft space (may be lost at TOP or BOP)
    % \vspace*{size}            -- hard space
    %                           -- size may be \stretch{factor}
    %                           --  of \smallskip or \bigskip

    % \\                        -- linebreak
    % \\*                       -- linebreak but prohibit page break
    % \newpage

    % \rule[lift]{width}{height}

    % \parbox[(c|t|b)]{width}{text}
    % \begin{minipage}[(c|t|b)]{width} text \end{minipage}
    % \mbox{text}               -- prevents word breaking

    % \begin{(flushleft|flushright|center|quote|verse)}
    % text
    % \end{(flushleft|flushright|center|quote|verse)}

    % \verb@verbatim text@
    % \begin{verbatim}
    % verbatim text
    % \end{verbatim}

    % \begin{tabular}{(l|r|c|p{width}|<bar>|@{colsep})...}
    % datum & datum & ... \
    % datum & datum & ... \
    % \hline
    % datum & datum & ... \
    % \cline{1-2}   & ... \
    % \multicolumn{2}{(l|c|r)}{wide datum} & ... \
    % \end{tabular}

    % \begin{(figure|table)}[[!](h|t|b|p)...]
    % ...
    % \caption{caption text}
    % \end{(figure|table)}

    % \label{marker}            -- set a marker
    % \ref{marker}              -- section containing marker
    % \pageref{marker}          -- page number of marker

    % $ ... $                   -- inline mathematics
    % \[ ... \]                 -- display mathematics
    % \begin{equation}          -- align* suppresses numbering
    % ...
    % \end{equation}
    % \begin{array}{...}        -- as tabular
    % ...
    % \end{array}
    % \begin{eqnarray}          -- as {array}{rcl} but with numbering
    % ...                       -- use \nonumber to suppress numbering of a row
    % \end{eqnarray}
    % \, \; \<spc> \quad \qquad -- math-mode spacing
    % \(over|under)line{...}
    % \(over|under)brace{...}_{...}
    % \wide(tilde|hat)
    % \overrightarrow
    % \frac{top}{bottom}
    % {... \choose ...}         -- adds parentheses
    % {... \atop ...}           -- no parentheses
    % \stackrel{topsym}{linesym}
    % \left<brasym> ... \right<ketsym>
    %                           -- use \right. for no <keysym>
    %                           -- empty lines etc. forbidden in math mode

% \include{}
    %
    % -- \includeonly takes a list of file names and filters any \includes
    %    (can only appear in the preamble.)
    % -- \include takes a single file name, starts a new page
    % -- \input does not start a new page
    % -- omit the .tex suffix from the file names

% \section{Introduction}

\Aside{This is intended as a brief sketch of the book, to be used as a guide
for writing the real text.}

\XXX{Check consistency of hyphenation for non- sub- etc.}




% % vim: ft=tex ff=unix ts=4 sw=4 et wm=8 tw=0

\chapter{Hello, World!}

Some example programs.



\section{Hello, World!}

Because it's traditional...  Type the following into a file called
@hello.m@:
\begin{verbatim}
:- module hello.

:- interface.
:- import_module io.

:- pred main(io, io).
:- mode main(di, uo) is det.

:- implementation.
:- import_module string.

    % The show starts here.
    %
main(IO0, IO) :-
    io__print("Hello, world!\n", IO0, IO).
\end{verbatim}
Compile and run the program with
\begin{verbatim}
$ mmc --make hello
$ ./hello
Hello, world!
\end{verbatim}
We'll start by just listing some of the salient points illustrated
by the ``Hello, world!'' program.
\begin{itemize}
\item Modules live in files with the same name (with a @.m@ suffix).
\item Every module starts with a declaration giving its name.
\item Non-code directives and declarations are introduced with @:-@.
\item Every declaration ends with a full stop.
\item Modules are divided into interface and implementation sections.
\item The interface section lists the things that are exported by the
  module.  The top-level module in a Mercury program must export a
  predicate @main/2@ (functors are conventionally referred to with
  @name/arity@).
\item The implementation section provides the code for the functions
  and predicates exported in interface section.
\item A module has to be imported before the things it defines can be
  used.  Hence @io__print@ refers to the print predicate defined in
  the io module.
\item The basic computational device in Mercury is the predicate.
  Predicates have type signatures and mode signatures -- the latter
  specifying which arguments are inputs and which are outputs.
\item Comments start with a @%@ sign and extend to the end of the line.
\end{itemize}

(IO is handled in Mercury by explicitly passing around the ``state
of the world''" -- each IO operation takes the current state and
produces a new one; the old state becoming unavailable for use
thereafter.  This might sound odd, but is a necessary part of
keeping Mercury a pure language.  \XXX{This will be explained in
more detail in the section on declaration IO.})

Special syntax exists to simplify passing state variable pairs
around; for ``Hello, world!'' this becomes
\begin{verbatim}
main(!IO) :-
    io__print("Hello, world!\n", !IO).
\end{verbatim}



\section{Rot13}

The simplest form of Caesar cypher is @rot13@, whereby each letter in a
message is replaced with one thirteen letters ahead in the alphabet
(wrapping around from @z@ to @a@.) That is, @a@ becomes @n@, @b@ becomes
@o@, \ldots, @z@ becomes @m@.  There being twenty six letters in the
Roman alphabet, a second application of @rot13@ to the cyphertext will
reveal the original plaintext message.

It is the custom in some public fora, such as USENET, to use @rot13@ to
encode such things as plot spoilers for films and books or jokes that
may offend, so that those not wishing to see the material in question
will not inadvertently catch a glimpse.

The Mercury code for the @rot13@ program looks like this:
\begin{verbatim}
:- module rot13.

:- interface.
:- import_module io.

:- pred main(io, io).
:- mode main(di, uo) is det.

:- implementation.
:- import_module int, exception.

main(!IO) :-
    io__read_byte(Result, !IO),
    (
        Result = ok(X),
        io__write_byte(
            ( if      0'a =< X, X =< 0'z
              then    (X + 13 - 0'a) `mod` 26 + 0'a
              else if 0'A =< X, X =< 0'Z
              then    (X + 13 - 0'A) `mod` 26 + 0'A
              else    X
            ),
            !IO
        ),
        main(!IO)
    ;
        Result = eof
    ;
        Result = error(_),
        throw(Result)
    ).
\end{verbatim}
\begin{itemize}
\item The IO read operation returns a success code in @Result@.
\item We \emph{must} handle the return code!
\item We \emph{switch} on the the result to decide what to do.
\item Mercury has conditional expressions.
\item The ASCII code of the character @a@ is written @0'a@.
\item Notice the use of @`@backquotes@`@ to treat a name as an infix
binary operator (@mod@ in this case.)
\end{itemize}



\section{A Spelling Checker}

\begin{verbatim}
:- module spell.

:- interface.
:- import_module io.

:- pred main(io, io).
:- mode main(di, uo) is det.

:- implementation.
:- import_module list, char, map, string, std_util, exception.

:- type dictionary == map(string, unit).

:- func dictionary = string.

    % This may be something else, like "/usr/dict/words".
    %
dictionary = "/usr/share/dict/words".

main(!IO) :-
    build_dictionary(Dict, !IO),
    check_spellings(Dict, !IO).

:- pred build_dictionary(dictionary, io, io).
:- mode build_dictionary(out,        di, uo) is det.

build_dictionary(Dict, !IO) :-
    io__open_input(dictionary, OpenResult, !IO),
    (   OpenResult = ok(InputStream)
    ;   OpenResult = error(_),          throw(OpenResult)
    ),
    io__read_file_as_string(InputStream, ReadResult, !IO),
    (   ReadResult = ok(String)
    ;   ReadResult = error(_, _),       throw(ReadResult)
    ),
    io__close_input(InputStream, !IO),
    Dict =
        foldl(
            func(Word, Dict0) = Dict0 ^ elem(to_lower(Word)) := unit,
            split_into_words(String),
            map__init
        ).

:- func split_into_words(string) = list(string).

split_into_words(String) = words(isnt(char__is_alnum), String).

:- pred check_spellings(dictionary, io, io).
:- mode check_spellings(in,         di, uo) is det.

check_spellings(Dict, !IO) :-
    io__read_line_as_string(ReadResult, !IO),
    (
        ReadResult = ok(String),
        list__foldl(check_spelling(Dict), split_into_words(String), !IO),
        check_spellings(Dict, !IO)
    ;
        ReadResult = eof
    ;
        ReadResult = error(_),
        throw(ReadResult)
    ).

:- pred check_spelling(dictionary, string, io, io).
:- mode check_spelling(in,         in,     di, uo) is det.

check_spelling(Dict, Word, !IO) :-
    ( if   Dict `contains` to_lower(Word)
      then true
      else io__format("%s\n", [s(Word)], !IO)
    ).
\end{verbatim}
\begin{itemize}
\item @build_dictionary/3@ returns a @dictionary@ in @Dict@.
\item @check_spellings/3@ \emph{uses} @Dict@.
\item We use higher order functions and predicates (@foldl@) where an
imperative programmer would use a loop.
\item We package up complex operations as \emph{closures} via partial
application (@check_spelling(Dict)@).
\item We can use lambdas or closures as the body of the loop.  It's all
the same thing.
\item @foldl@ is overloaded in that it comes in both function and
predicate versions and the predicate version has multiple modes (it can
handle unique IO states, for instance.)
\item A @map@ is an ADT that supports dictionary-like operations (you
use one to set up a mapping between keys and values -- an associative
array, if you like.)  @map@s support O(log n) lookup and insert times.
\item In this case, we only need to check that the @dictionary@ contains
a particular word.  In this case we defined a @dictionary@ to be a map
from @string@s to @unit@s, where @unit@ is the type that carries no
information -- it has but one data constructor, also called @unit@.
\end{itemize}



\section{Finding a Path Through a Maze}

\begin{verbatim}
:- module route.
:- interface.
:- import_module int, list.

:- type vertex(T) == T.
:- type exits(T)  == list(vertex(T)).
:- type maze(T)   == list({vertex(T), exits(T)}).
:- type path(T)   == list(vertex(T)).

:- func sample_maze = maze(char).

:- pred find_path(maze(T), vertex(T), vertex(T), path(T)).
:- mode find_path(in,      in,        in,        out    ) is nondet.

:- implementation.

    %   a --> b     c
    %         |     ^
    %         v     |
    %   d <-- e <-- f
    %   |     ^     ^
    %   v     |     |
    %   g --> h --> i
    %
sample_maze =
    [   {a, [b]},       {b, [e]},       {c, []},
        {d, [g]},       {e, [d]},       {f, [c, e]},
        {g, [h]},       {h, [e, i]},    {i, [f]}
    ].

find_path(Maze, From, To, Path) :-
    find_reverse_path(Maze, To, [From], RPath),
    Path = reverse(RPath).

:- pred find_reverse_path(maze(T), vertex(T), path(T), path(T)).
:- mode find_reverse_path(in,      in,        in,      out    ) is nondet.

find_reverse_path(Maze, To, RPath0 @ [Vertex | _], RPath) :-
    ( if Vertex = To then
        RPath = RPath0
      else
        list__member(Maze, {To, Exits}),
        list__member(Exit, Exits),
        not list__member(Exits, Path0),
        find_reverse_path(Maze, To, [Exit | Path0], Path)
    ).
\end{verbatim}
\begin{itemize}
\item Mercury has parametrically polymorphic types.
\item Mercury allows \emph{unification expressions} like
\verb!RPath0 \@ [Vertex | _]!.
\item Mercury supports non-determinism (where programs -- or more
correctly, \emph{goals} -- may \emph{fail}
or have \emph{more than one solution}).  Failure leads to
\emph{backtracking} to look for alternative ways to a solution.
\emph Mercury supports multiple modes for a predicate (e.g.
@list__member@ is being used in both the @(out, in) is nondet@ and
@(in, in) is semidet@ modes.
\emph Mercury goals may be negated.
\end{itemize}

% % vim: ft=tex ff=unix ts=4 sw=4 et wm=8 tw=0

\chapter{Declarative vs Imperative Programming}

\section{Definitions}

\begin{description}
\item{Imperative:} a sequence of instructions for transforming state.
\item{Declarative:} a specification of \emph{what} is to be computed.
\end{description}

Declarative languages typically have a simple translation into
conventional mathematical logic, which is how the meaning of a program
is defined.

Purely declarative programming languages exhibit referential
transparency.  In a nutshell, this means that anywhere you see a
reference to a name, you can replace it with the body of the
definition of the name and it will make no difference.

In more formal language,
\[
(\text{let}\ x = e\ \text{in}\ M)  \equiv  M[e/x]
\]
This is clearly not true of imperative languages.  Consider
the following C program:

\begin{verbatim}
int g = 0;

int f(int x)
{
    g = g + 1;
    return x + g;
}

void main(int argc, char **argv)
{
    int tmp = f(1);
    int a   = tmp  + tmp;
    int b   = f(1) + f(1);

    if(a == b)
        printf("equivalent\n");
    else
        printf("not equivalent\n");
}
\end{verbatim}

According to C semantics, this program should print out "not
equivalent", proving that the expression @f(1)@ is not equal to
itself!  This sort of thing is great for writing buggy, hard to
maintain code.\footnote{One can give a reasonably simple
operational semantics to C whereby we repeatedly substitute the
body of @f(1)@ into a sequence of \emph{instructions}, but as we
see here, this does not necessarily support the more intuitive,
declarative reading one might hope for.}

Since referential transparency means no side-effects, you can't have
variables that change their state as the program evolves.  Instead, the
term \emph{variable} in a declarative programming language refers to a
label given to a value or the result of a computation.

The nearest Mercury equivalent to the function @f()@ in the C
program above is

\begin{verbatim}
:- pred f(int, int, int, int).
:- mode f(in,  out, in,  out) is det.

f(X, Result, Old_Value_of_G, New_Value_of_G) :-
    New_Value_of_G = Old_Value_of_G + 1,
    Result         = New_Value_of_G + X.
\end{verbatim}

Since there are no variables in Mercury (and hence no global
variables), any ``global'' state has to be explicitly passed
around wherever it is needed.  In this case @f/4@ takes the old
``state'' as its third argument and returns the new ``state'' in
its fourth argument.

\section{Benefits Of Declarative Style}

While the lack of mutable state might seem like a serious
drawback to programmers raised on imperative languages, in
practice it turns out to be something of a boon.  Experienced
imperative programmers acknowledge that fewer globals and less
state means clearer, more maintainable, more reusable code
with fewer bugs.

The philosophy underlying declarative languages is to simplify the
\emph{writing} of bug free programs and to leave the tedious
business of identifying programming errors, run-time book
keeping such as memory management, and non-algorithmic
optimization to the compiler.  Referential transparency means
that more optimizations can be applied in more places in a
program than is the case with imperative programs, simply
because proving that it is safe to apply a particular
optimization is so much easier in the absence of side effects.

Mercury was designed as a purely declarative, industrial
strength programming language aimed at the rapid development
of medium and large scale systems with the emphasis on
producing fast, correct programs.

To this end, Mercury doesn't support a corner-cutting
programming style: you \emph{have} to get the types right, check
return codes and so forth -- the quick-and-dirty fix is rarely
an option.\footnote{Mercury does have support for impure code
via calls to another language, but one has to label each such
call explicitly; in the end it is often easier just to do the
right thing.}  Very often, Mercury programmers find that once
a program is accepted by the compiler, it also does exactly
what was intended; in the author's long experience this almost
never happens with imperative languages, even for small
programs.

\XXX{Place to mention types, garbage collection, polymorphism,
pattern matching, etc etc?}

\section{Pragmatism}

\begin{quote}
``A foolish consistency is the hobgoblin of small minds.'' \\
\hfill --- Ralph Waldo Emmerson
\end{quote}

Purity is all very well, but the fact remains that
occasionally one has to interoperate with code written in
impure languages and that (very rarely and usually when that
last drop of speed is required) some low-level algorithms may
be best expressed as impure constructs.

For the former, Mercury has a simple and well developed
foreign language interface allowing the programmer to write
foreign code in-line with the Mercury program (provided the
Mercury compiler has an appropriate back-end for the foreign
language in question -- the alternative is to use C as the
lingua franca in the traditional style.)  It is the
programmer's responsiblility to supply the appropriate purity
declarations for predicates defined in terms of foreign code.

The latter is handled using purity annotations.  Impure code
(written in a foreign language) must be labelled as such.
These labels also have to be applied to all predicates that
use the impure code.  At some point we hope that a pure
interface can be presented to the programmer, in which case
the program must include a \emph{promise} to the compiler that
impurity annotations are not required for users of the
top-level predicate with an impure definition.

\section{Mercury Philosophy}

\XXX{I think I've already covered this one.  Tyson suggests an
implementation philosophy section (\eg no distributed fat) in
a much later section.}




% % vim: ft=tex ff=unix ts=4 sw=4 et wm=8 tw=0

\chapter{Logic and Logic Programming}

\XXX{Convert all logic to tt and Mercury syntax.  Add comments in the
intro to make it clear what it would look like in a logic text book.}

While programming in a purely functional style is a fairly intuitive
notion, the notion of programming in logic might at first seem a little
odd.  Nevertheless, in this section we shall demonstrate that there is
an easily understood logic-based programming paradigm.  A firm grasp of
logic is a great help in understanding Mercury, so we start with a
refresher course on basic logic.

\section{The Abstract Stuff}

To quote Douglas Adams, \emph{Don't Panic!}

\subsection{What's the Point?}

Logics help us to reason about the world.  Ordinary, natural language is
generally too verbose and too imprecise to use effectively in complex
situations.  When thinking about computer programs the problem is
exacerbated because computers take everything we say to them literally
and they have absolutely no imagination.

A logic allows us to say simple things clearly and concisely and gives
us an unambiguous mechanism for deducing more complex things thereafter.

The simplicity of logic can be deceptive in two ways.  The first is it
often seems \emph{too} simple to be of any use.  The second is that one
is often tempted to say of a particular statement, ``Why do we need
logic here?  This is clearly correct.''  The answer in both cases is
that you'd be amazed at how often these points of view are wrong.

Finally, when we use logic to help us think about and solve problems on
paper, we can very often turn that effort directly into a reliable,
working computer program with the minimum of effort.

\subsection{What Is a Logic?}

In the most abstract sense, a \emph{logic} consists of
\being{itemize}
\item a \emph{language} which specifies the set of well formed
\emph{sentences}, and
\item a set of \emph{inference rules} which allow us to deduce new
sentences from a given set of sentences.
\end{itemize}
A sentence is often also referred to as a \emph{formula}.  We will use
both terms interchangably throughout the book.

\subsection{What is a Theory?}

A theory is the set of sentences that can be obtained with respect to a
particular logic where each sentence is either
\begin{itemize}
\item a member of a particular starting set of \emph{axioms}, which are
taken to be ``true without proof'', or
\item can be \emph{proved} from other sentences in the theory.
\end{itemize}
A sentence in a theory is referred to as a \emph{theorem}.

\subsection{What is a Proof?}

A \emph{proof} is a sequence of sentences where each step is either
\begin{itemize}
\item an axiom or
\item a sentence obtained by application of an inference rule to one or
more of the preceding sentences in the proof.
\end{itemize}

\subsection{What is an Interpretation?}

An \emph{interpretation} can be viewed as the set of sentences that are
true in a particular problem domain, all other sentences being deemed to
be false.  One way of looking at an interpretaion is as a way of
relating sentences in the logic to things in the problem domain.

\subsection{Consistency, Completeness and Correctness}

A theory is \emph{consistent} if it does not contain any mutually
contradictory theorems.

A theory is \emph{complete} with respect to an interpretation if the
interpretation is a subset of the theory (otherwise the interpretation
contains some sentences that are true but are not part of the theory.)

A theory is \emph{correct} with respect to an interpretation if it is a
subset of the interpretation (otherwise the theory claims some theorems
are true that are in fact false under the given interpretation.)

\subsection{Putting it All Together}

So, for any particular problem, we need the following if we are going to
reason about it logically:
\begin{itemize}
\item a language defining the sentences we can consider;
\item a set of rules allowing us to deduce new sentences from old;
\item a set of axioms defining our starting point;
\item an interpretation connecting the results of our abstract, logical
efforts to the problem domain we are interested in.
\end{itemize}

\section{Propositional Logic}

The simplest useful logic is the \emph{propositional calculus}.
The propositional calculus is used to reason about logical formulae
composed of atomic propositions -- that is, simple statements that are
either true or false.

In what follows we use @P@ and @Q@ to stand for arbitrary sentences.

\subsection{Language}

\begin{itemize}
\item We use @word@s or letters, @a@, @b@, @c@, \ldots to stand for
\emph{atomic} propositions;
\item we usually include @true@ as a proposition that is true in all
interpretations and @false@ as a proposition that is false in all
interpretations;
\item @not P@ stands for the \emph{negation} ``@P@ is false'';
\item @(P, Q)@ stands for the \emph{conjunction} ``@P@ and @Q@'' meaning
\emph{both} @P@ and @Q@ are true;
\item @(P ; Q)@ stands for the \emph{disjunction} ``@P@ or @Q@'' meaning
\emph{at least one of} @P@ and @Q@ is true;
\item @(P => Q)@ stands for the \emph{implication} ``if @P@ then @Q@''
meaning if @P@ is true, then @Q@ must be true (it says nothing about @Q@
in the case where @P@ is false).  @P@ is referred to as the
\emph{antecedent} and @Q@ as the \emph{consequent} of the implication.
\end{itemize}

We may occasionally use @(Q <= P)@ as a convenient notation for
@(P => Q)@ and @(P <=> Q)@ for @((P => Q), (P <= Q))@ (the latter type
of formula is called an \emph{equivalence}).

\textbf{Note on syntax:} we use Mercury style syntax throughout this
chapter.  Books on logic will tend to use slightly different symbols for
the logical connectives:
\begin{itemize}
\item @not P@ is conventionally written as $\Not$@P@;
\item @(P, Q)@ is conventionally written as @P@$\Conj$@Q@;
\item @(P ; Q)@ is conventionally written as @P@$\Disj$@Q@;
\item @(P => Q)@ is conventionally written as @P@$\Imp$@Q@.

\subsection{Rules of Inference}

We present rules of inference using \emph{sequent notation}.  Sequents
can be read thus: given sentences matching what appears above the line,
we may deduce the sentence below the line.

We can always infer @true@, which is the negation of @false@.
The double negation of a proposition is the same as asserting that
proposition.
\begin{verbatim}
                                not not P       P
-----           ----------      ----------      ----------
true            not false       P               not not P
\end{verbatim}
(If we omit the double negation rule then we have an intuitionistic
logic.)

We can form the conjunction of any two true propositions.
We can reorder conjunctions.
We can project a conjunct from a conjunction.
Conjunction with any false proposition is also false.
\begin{verbatim}
P
Q               P, Q            P, Q            not P
-----           -----           -----           ----------
P, Q            Q, P            P               not (P, Q)
\end{verbatim}
Conjunction is therefore commutative and associative.

We can form a disjunction of any other proposition with a true
proposition.
We can reorder disjunctions.
If a disjunction is false then so are its disjuncts.
\begin{verbatim}
                                                not P
P               P ; Q           not (P ; Q)     not Q
------          ------          ------------    ------------
P ; Q           Q ; P           not P           not (P ; Q)
\end{verbatim}
Disjunction is therefore commutative and associative.

If the antecedent of an implication is true then the consequent must
also be true (\emph{modus ponens}).
If the consequent of an implication is false then the antecedent must
also be false (\emph{modus tollens}).
We can introduce an implication using any proposition at all for a true
consequent or a false antecedent:
\begin{verbatim}
P               not Q
P => Q          P => Q          Q               not P
-------         -------         -------         -------
Q               not P           P => Q          P => Q
\end{verbatim}

\subsection{On Implication}

The rules for implication might seem a little odd.  One should be
careful to avoid confusing logical implication with causation.  With
causation, the consequent must follow ``because'' of the antecedent.
Logical implication, on the other hand, need not make any kind sense.
It is perfectly all right to put any proposition at all on either side
of a logical implication; whether the implication is justified or not is
another matter entirely and is dictated by its context within a theory
or proof.

Another common source of confusion is that implication says nothing
about what happens if the antecedent is false.  The idea is that if we
have @(P => Q)@ and also @not P@ then the implication isn't
``triggered'' and hence cannot make any false claim.  The only situation
where an implication @(P => Q)@ is false is if @P@ is true, but @Q@ is
\emph{false}.  In this case the implication is ``triggered'', but
contradicts @not Q@ which has already been established.  Consequently,
if we have @(P => Q)@ and we know that @Q@ is false then the only
consistent deduction is that @P@ is also false.

\subsection{Some Basic Proofs}

It is instructive to see some simple proofs.  When constructing a proof
we work bottom-up: we start with the theorem we wish to prove and then
use the inference rules in reverse to get back to whatever we've allowed
ourselves to take as given.

First off, let's show that from @(false ; P)@ we can infer @P@:
\begin{verbatim}
 1      (false ; P)             given

 2      not false               from the negation rules
 3      P                       resolution of 2 and 1
\end{verbatim}

Next, let's show that @not (false, P)@ is always true:
\begin{verbatim}
 1      not false               from the negation rules
 2      not (false, P)          from the conjunction rules and 1
\end{verbatim}

The following relationship is extremely useful: @(P => Q)@ is equivalent
to @(not P ; Q)@.  We prove this in two stages, first starting with the
assumption @(not P ; Q)@.
Since we start with a disjunction, it is sufficient to show that if each
disjunct independently supports the conclusion then the disjunction as a
whole must support the conclusion.
\begin{verbatim}
 1      (not P ; Q)             given

 2.L    not P                   assuming the left hand side of 1
 3.L    (P => Q)                from the implication rules and 2.L

 2.R    Q                       assuming the right hand side of 1
 3.R    (P => Q)                from the implication rules and 2.R

 4      (P => Q)                from 1, 3.L and 3.R
\end{verbatim}
To finish, we start with the assumption @(P => Q)@.  Since we start with
an implication, it is sufficient to show that we can prove @(not P ; Q)@
if either the antecedent is false or the consequent is true.
\begin{verbatim}
 1      (P => Q)                given

 2.L    not P                   assuming the antecedent of 1 is false
 3.L    (not P ; Q)             from the disjunction rules and 2.L

 2.R    Q                       assuming the consequent of 1
 3.R    (not P ; Q)             from the disjunction rules and 2.R

 4      (not P ; Q)             from 1, 3.L and 3.R
\end{verbatim}

Augustus De Morgan, the 19th century logician, showed that @not (P, Q)@
is equivalent to @(not P ; not Q)@, which is the same as saying
@not (P ; Q)@ is equivalent to @(not P, not Q)@.  We prove this in two
stages, first starting with the assumption @not (P, Q)@.
\begin{verbatim}
 1      not (P, Q)              given

 2.L    not P                   assuming the left hand side of 1 is false
 3.L    (not P ; not Q)         from the disjunction rules and 2.L

 2.R    not P                   assuming the right hand side of 1 is false
 3.R    (not P ; not Q)         from the disjunction rules and 2.R

 4      (not P ; not Q)         from 1, 3.L and 3.R
\end{verbatim}
Going the other way, we assume @not (P ; Q)@.
\begin{verbatim}
 1      not (P ; Q)             given

 2      not P                   from the disjunction rules and 1
 3      not Q                   from the disjunction rules and 1
 4      (not P, not Q)          from the conjunction of 2 and 3
\end{verbatim}

The \emph{resolution} rule for disjunctions says that if one disjunct is
false then the other must be true:
\begin{verbatim}
 1      (P ; Q)                 given
 2      not P                   given

 3      (not P => Q)            equivalent to 1
 4      Q                       modus ponens over 2 and 3
\end{verbatim}
As a special case of this rule we see that @P@ must follow from
@(false ; P)@.

Aristotle's \emph{law of the excluded middle} states that the
disjunction of a proposition with itself must be true:
\begin{verbatim}
 1      (P ; not P)             given

 2      true                    trivial
 3      (P => true)             from the implication rules and 2
 4      (not P => true)         from the implication rules and 2

 5.L    P                       assuming the left hand side of 1
 6.L    true                    modus ponens over 5.L and 3

 5.R    not P                   assuming the right hand side of 1
 6.R    true                    modus ponens over 5.R and 4

 7      true                    from 1, 6.L and 6.R
\end{verbatim}

Implication is transitive:
\begin{verbatim}
 1      (P => Q)                given
 2      (Q => R)                given

 3.L    not P                   assuming antecedent of 1 is false
 4.L    (P => R)                from implication rules and 3.L

 3.R    Q                       assuming consequent of 1 is true
 4.R    R                       modus ponens over 3.R and 2
 5.R    (P => R)                from implication rules and 4.R

 6      (P => R)                from 1, 2, 4.L and 5.R
\end{verbatim}

\subsection{Cavilling Vilely}

\XXX{Check this isn't copyright!  It's been doing the Cambridge Tripos
rounds for years, at least.}

Consider the following somewhat contorted problem statement:
\begin{quote}
If Anna can cancan or Kant can't cant, then Greville will cavil vilely.
If Greville will cavil vilely, Will will want.
But Will \emph{won't} want.
So can Kant cant?
\end{quote}
How can we use propositional logic to decide the truth of the question?
Here's how.  We start by labelling the various propositions:
\begin{description}
\item let @a@ stand for ``Anna can cancan'';
\item let @k@ stand for ``Kant can cant'';
\item let @g@ stand for ``Greville will cavil vilely'';
\item and let @w@ stand for ``Will will want''.
\end{description}

Our axioms are obtained by translating the statement of the problem into
sentences in propositional logic:
\begin{verbatim}
 1      ((a ; not k) => g)      given
 2      (g => w)                given
 3      not w                   given
\end{verbatim}
Our \emph{goal} is to find a proof of either @k@ or @not k@; we can
obtain a proof of the former as follows (and since the problem statement
and logic are consistent, this means we can't prove the latter):
\begin{verbatim}
 4      not g                   from the implication rules and 3 and 2
 5      not (a ; not k)         from the implication rules and 4 and 1
 6      (not a, k)              De Morgan's law applied to 5
 7      k                       projecting the right hand side of 6
\end{verbatim}
So Kant \emph{can} cant after all!

\subsection{On Obtaining Proofs}

Finding a proof or disproof of a formula is not straightforward.  Since
this book only introduces logic, an in-depth treatment of proof
procedures is beyond its scope.  However, it behooves us to say a few
words on the subject.

The key idea in obtaining a proof is to work backwards from the goal.
One has to look at the inference rules and the axioms of the problem
domain and search for steps that will break down the goal into simpler
subgoals.  This procedure continues until the subgoals are either
trivial (deducible without axioms) or are themselves in the set of
axioms.  The tricky part is that one may need to handle disjunctions (it
is sometimes simpler to convert implications into their equivalent
disjunctive forms) by proving that the goal is provable regardless of
which disjunct is true.  Finding proofs is not easy and takes some
practice before one becomes proficient.

Note that if you don't \emph{know} that a particular formula is true or
false then one has to simultaneously look for proofs of both
possibilities: in a consistent theory you cannot find a disproof of a
theorem or a proof of a non-theorem!

While complete proof algorithms exist for deciding the truth of
arbitrary formulae in the propositional calculus (see the
\XXX{Davis-Putnam method}, for instance), unfortunately its more
expressive cousin, the predicate calculus, does not admit this property
(that is, there is no way to construct an algorithm to decide whether an
arbitrary predicate calculus formula is true or false.)  Mercury can be
viewed as implementing a correct, but incomplete, proof procedure for
first order logic.  But don't worry about that for now.

\subsection{Informal Shortcuts}

We often want to sketch out a proof quickly before writing it down in a
formal fashion (if, indeed, we decide we need this level of rigour.)  In
such situations it is acceptable to take a ``term rewriting'' approach:
given @P <=> Q@ we can always substitute @P@ for @Q@ and vice versa;
given @P => Q@ we may substitute @P@ for @Q@ in the antecedent of
implications and @Q@ for @P@ in the consequent of implications.  Below
is a handy selection of ``rewriting'' rules:
\begin{verbatim}
(P => Q)             <=>  (not Q => not P)
(P => Q)             <=>  (not P ; Q)
((P, Q) => R)        <=>  (P => (Q => R))

not true             <=>  false
not false            <=>  true
not not P            <=>  P

(P, not P)           <=>  false
(P, false)           <=>  false
(P, true)            <=>  P

(P ; not P)          <=>  true
(P ; true)           <=>  true
(P ; false)          <=>  P

((P => Q), (Q => R))  =>  (P => R)
(P, Q)                =>  P
(not P, (P, Q))       =>  Q
\end{verbatim}

\subsection{An Interesting Observation}

The reader may find it interesting to note that one can construct all
the rules of inference of propositional logic using just implication and
falsity:
\begin{verbatim}
not P                <=>  (P => false)
(P ; Q)              <=>  (not P => Q)
(P, Q)               <=>  ((P => not Q) => false)
\end{verbatim}

\section{Predicate Logic}

The expressive power of propositional logic is rather limited.  The key
problem is that it makes no provision for making general statements such
as, ``The square of an odd number is also an odd number.''  Instead one
is forced into writing down an (in this case) infinite number of
propositions of the form ``$1$ is an odd number'', ``$1^2$ is an odd
number'', ``$3$ is an odd number'', ``$3^2$ is an odd number'' and so
forth.  The relationship between odd numbers and this property of their
squares is not explicitly represented, other than as a correspondence
amongst the set of axioms.

Mercury programs are restricted predicate calculus theories; it is
actually very helpful to bear this in mind when reasoning about them.

\subsection{Language}

The language of predicate calculus is slightly more complicated than
that for propositional logic, introducing as it does logical variables,
terms and parameterised predicates.
\begin{itemize}
\item \emph{Terms} represent objects in the problem domain, such as
numbers, people, and so forth.  Terms can be structured -- a list of
terms would itself be a perfectly good term.
\item \emph{Logical variables} range over terms and appear as parameters
to predicates.
\item \emph{Predicates} represent properties of terms.  Unlike
propositions, predicates may be parameterized.  For example, we might
have a predicate @odd(X)@ to denote @X@ being an odd number, or a
predicate @parent(A, B)@ to denote @A@ being a parent of @B@.  We
normally assume the usual equality relation, @=@.
\end{itemize}
Sticking with Mercury syntax, predicates and term names start with
@lower case@ letters (unless we are using symbols for conveience, such
as @123@ or @*@ or list notation and such like.)  Variable names start
with @Upper Case@ letters.

Formulae are composed in much the same way as for the propositional
calculus, with the addition of two ways of introducing logical
variables.  Taking @P@ and @Q@ to stand for arbitrary logical formulae:
\begin{itemize}
\item @not P@ stands for the \emph{negation} ``@P@ is false'';
\item @(P, Q)@ stands for the \emph{conjunction} ``@P@ and @Q@'' meaning
\emph{both} @P@ and @Q@ are true;
\item @(P ; Q)@ stands for the \emph{disjunction} ``@P@ or @Q@'' meaning
\emph{at least one of} @P@ and @Q@ is true;
\item @(P => Q)@ stands for the \emph{implication} ``if @P@ then @Q@'';
\item @(some [X] P)@ stands for the \emph{existential
quantification} ``for some value of @X@ @P@'';
\item @(all [X] P)@ stands for the \emph{universal
quantification} ``for all values of @X@ @P@'';
\end{itemize}

We allow several variables to appear in the list of a quantification
with the understanding that
\begin{verbatim}
(some [X, Y, Z] P)
\end{verbatim}
really means
\begin{verbatim}
(some [X] (some [Y] (some [Z] P)))
\end{verbatim}

We will allow ourselves to use a pattern matching shorthand in
definitions (@<=>@) and implications (@=>@ and @<=@), so rather than
writing
\begin{verbatim}
(all [X] (p(X) <= X = 42))
\end{verbatim}
we simply write
\begin{verbatim}
p(42)
\end{verbatim}
to mean the same thing.

Returning briefly to our ``square numbers'' example, we would write that
as
\begin{verbatim}
(all [X, Y] ((odd(X), square(X, Y)) => odd(Y)))
\end{verbatim}
which reads as ``for all values of @X@ and @Y@, if @X@ is odd and @Y@ is
the square of @X@, then @Y@ is odd.''

\subsection{Rules of Inference}

For the most part, the rules of inference for the predicate calculus
appear identical to those for the propositional calculus.  The important
difference is the rules that dictate how variables are handled.

We can always infer @true@, which is the negation of @false@.
The double negation of a proposition is the same as asserting that
proposition.
\begin{verbatim}
                                not not P       P
-----           ----------      ----------      ----------
true            not false       P               not not P
\end{verbatim}
(If we omit the double negation rule then we have an intuitionistic
logic.)

We can form the conjunction of any two true propositions.
We can reorder conjunctions.
We can project a conjunct from a conjunction.
Conjunction with any false proposition is also false.
\begin{verbatim}
P
Q               P, Q            P, Q            not P
-----           -----           -----           ----------
P, Q            Q, P            P               not (P, Q)
\end{verbatim}

We can form a disjunction of any other proposition with a true
proposition.
We can reorder disjunctions.
If a disjunction is false then so are its disjuncts.
\begin{verbatim}
                                                not P
P               P ; Q           not (P ; Q)     not Q
------          ------          ------------    ------------
P ; Q           Q ; P           not P           not (P ; Q)
\end{verbatim}

If the antecedent of an implication is true then the consequent must
also be true (\emph{modus ponens}).
If the consequent of an implication is false then the antecedent must
also be false (\emph{modus tollens}).
We can introduce an implication using any proposition at all for a true
consequent or a false antecedent:
\begin{verbatim}
P               not Q
P => Q          P => Q          Q               not P
-------         -------         -------         -------
Q               not P           P => Q          P => Q
\end{verbatim}

We can substitute any term at all for a universally quantified
variable.
The dual of universal quantification is existential quantification.
\begin{verbatim}
all [X] P       some [X] P
----------      --------------------
P[a/X]          not (all [X] not P)

P[a/X]          not (all [X] not P)
-----------     --------------------
some [X] P      some [X] P
\end{verbatim}
where @P[a/X]@ means @P@ with the term @a@ replacing all free
occurrences of @X@ therein.  A free variable is one not in the scope of
any enclosing quantification.  For instance, @X@ is free in @odd(X)@,
but not in @some [X] odd(X)@.

\XXX{Check the above sequents.  I think there's a bug.}

\subsection{Finding the Length of a List}

Here's an example of how we can use logic to work out the length of a
list.
\begin{verbatim}
all [X, Y]
    length(Xs, N) <=>
        (
            (Xs = [], N = 0)
        ;
            some [X0, Xs0, N0]
                (   Xs = [X0 | Xs0],
                    length(Xs0, N0),
                    N  = N0 + 1
                )
        )
\end{verbatim}
which, incidentally, corresponds to the Mercury definition
\begin{verbatim}
length([],         0     ).
length([X0 | Xs0], N0 + 1) :- length(Xs0, N0).
\end{verbatim}

First, let's see how we can prove @length([a, b], 2)@ (remember, we
construct this proof bottom-up):
\begin{verbatim}
 1      [] = []
 2      0  = 0
 3      ([] = [], 0 = 0)
 4      (([] = [], 0 = 0) ; ...)
 5      length([], 0)

 6      [b]    = [b | []]
 7      0 + 1  = 0 + 1
 8      ([b] = [b | []], length([], 0), 0 + 1 = 0 + 1)
 9      some [X0, Xs0, N0] ([b] = [X0 | Xs0], ...)
   ...
10      length([b], 0 + 1)

11      [a, b]    = [a | [b]]
12      0 + 1 + 1 = 0 + 1 + 1
13      ([a, b] = [a | [b]], length([b], 0 + 1), 0 + 1 + 1 = 0 + 1 + 1)
14      some [X0, Xs0, N0] ([a, b] = [a | [b]], ...)
   ...
15      length([a, b], 0 + 1 + 1)
\end{verbatim}
and provided we accept that @0 + 1 + 1 = 2@ then our work here is done.

We haven't actually \emph{calculated} anything here; rather we've just
proved something we strongly suspected was true anyway.  What we really
want to do, as logic programmers, is find some way of working out, for
instance, what @N@ satisfies @length([a, b, c], N)@.

Just as we can take a term rewriting approach to solving problems in
propositional logic, we can use similar shortcuts in predicate logic.
In particular, this will avoid the need for us to pluck the right terms
from thin air when looking for a proof of a formula containing variables.
We use the following ``tricks'' while working backwards from the
top-level goal:
\begin{itemize}
\item existentially quantified variables are left in the 
\end{itemize}
\XXX NEED TO SORT THIS BIT OUT.





De Morgan's laws extend to cover quantification, giving us the following
identities:
\begin{align*}
\All{X}{p}
& \Eqv \Not{\Some{X}{\Not{p}}} \\
\Some{X}{p}
& \Eqv \Not{\All{X}{\Not{p}}} \\
\end{align*}

Since the names of quantified variables do not matter, we can rename
them at will.

\subsection{Using Quantifiers}

From $\All{X}{p}$ we can deduce $p[t/X]$ for any term $t$.  The
expression $p[t/X]$ denotes the application of the \emph{substitution}
$[t/X]$ to $p$ -- that is, all free occurrences of $X$ in $p$ are
replaced with $t$.  Substitution works as follows:
\begin{align*}
\text{(Over terms)} \\
X[t/X]
& = t \\
Y[t/X]
& = Y \text{ provided $X$ and $Y$ are different variables} \\
f(t_1, t_2, \ldots, t_n)[t/X]
& = f(t_1[t/X], t_2[t/X], \ldots, t_n[t/X]) \\
\text{(Over formulae)} \\
a(t_1, t_2, \ldots, t_n)[t/X]
& = a(t_1[t/X], t_2[t/X], \ldots, t_n[t/X]) \\
\Not{p}[t/X]
& = \Not{(p[t/X])} \\
(p \Conj q)[t/X]
& = (p[t/X] \Conj q[t/X]) \\
(p \Disj q)[t/X]
& = (p[t/X] \Disj q[t/X]) \\
(p \Imp q)[t/X]
& = (p[t/X] \Imp q[t/X]) \\
(\All{X}{p})[t/X]
& = \All{X}{p} \\
(\All{Y}{p})[t/X]
& = \All{Y}{p[t/X]} \text{ provided $X$ and $Y$ are different variables} \\
(\Some{X}{p})[t/X]
& = \Some{X}{p} \\
(\Some{Y}{p})[t/X]
& = \Some{Y}{p[t/X]} \text{ provided $X$ and $Y$ are different variables} \\
\end{align*}

From $p(t)$, for any term $t$, we can deduce $\Some{X}{p(X)}$.  Or, more
correctly, given $p[t/X]$ we conclude $\Some{X}{p}$.

We can simplify quantified formulae:
\begin{align*}
\All{X}{p \Conj q}
& \Eqv \All{X}{p} \Conj \All{X}{q}\\
\Some{X}{p \Conj q}
& \Eqv \Some{X}{p} \Conj \Some{X}{q}
\end{align*}

Provided $X \notin \FV(p)$ can rearrange quantifiers like so:
\begin{align*}
p \Conj \All{X}{q}
& \Eqv \All{X}{p \Conj q} \\
p \Conj \Some{X}{q}
& \Eqv \Some{X}{p \Conj q} \\
\end{align*}
The constraint is required because we do not want to inadvertently
\emph{capture} a free variable that happens to be called $X$ in $p$ in
the quantifier.  We can always move quantifiers out to the outermost
level by renaming variables as necessary.

\subsection{An Example: Schubert's Steamroller}

This rather knotty problem was devised by Mark E. Schubert \XXX{}.

We start off with a statement of the problem in plain English.
\begin{itemize}
\item Snails, caterpillars, birds, foxes and wolves are all animals.
\item Grain is a kind of plant.
\item Each species of animal eats all types of plants
or all species of smaller animals that eat some types of plants.
\item Wolves are bigger than foxes, foxes are bigger than birds, and
birds are bigger than caterpillars and snails.
\item Wolves don't eat foxes or grain.  Birds eat caterpillars, but not snails.
Caterpillars and snails like to eat plants.
\item \textbf{Is there an animal that eats a grain-eating animal?}
\end{itemize}

Now let's translate the axioms of the problem into logical formulae:
\begin{tabular}{rl}
    &  (Axioms.) \\
(1a) & $\Animal(\Snail)$ \\
(1b) & $\Animal(\Caterpillar)$ \\
(1c) & $\Animal(\Bird)$ \\
(1d) & $\Animal(\Fox)$ \\
(1e) & $\Animal(\Wolf)$ \\
\\
(2) & $\Plant(\Grain)$ \\
\\
(3) & $\All{X}{\Animal(X) \Imp \Herbivorous(X) \Disj \Carnivorous(X)}$ \\
(3a) & $\All{X}{\Herbivorous(X) \Eqv$ \\
     & $\qquad \All{Y}{\Eats(X, Y) \Bimp \Plant(Y)}}$ \\
(3b) & $\All{X}{\Carnivorous(X) \Eqv$ \\
     & $\qquad \All{Y}{\Eats(X, Y) \Bimp
                    \BiggerThan(X, Y) \Conj
                    \Some{Z}{\Plant(Z) \Conj \Eats(Y, Z)}}}$ \\
                    \\
(4a) & $\BiggerThan(\Wolf, \Fox)$ \\
(4b) & $\BiggerThan(\Fox, \Bird)$ \\
(4c) & $\BiggerThan(\Bird, \Caterpillar)$ \\
(4d) & $\BiggerThan(\Bird, \Snail)$ \\
\\
(5a) & $\Not{\Eats(\Wolf, \Fox)}$ \\
(5b) & $\Not{\Eats(\Wolf, \Grain)}$ \\
(5c) & $\Eats(\Bird, \Caterpillar)$ \\
(5d) & $\Not{\Eats(\Bird, \Snail)}$ \\
(5e) & $\Herbivorous(\Caterpillar)$ \\
(5f) & $\Herbivorous(\Snail)$ \\
\end{tabular}

The goal is then $\Some{X, Y}{\Eats(X, Y) \Conj \Eats(Y, \Grain)}$.

It turns out that the answer to the conundrum is that foxes eat birds
who eat grain (it's probably easier to write a computer program called a
\emph{theorem prover} to work this out than to do so by trying out each
combination by hand\ldots)  But how can we prove this?  Here's how:

\begin{tabular}{rl}
& (Deduce that wolves are carnivorous.) \\
(6) & $\Not{\Eats(\Wolf, \Grain)} \Conj \Plant(\Grain)$
\\ & --- by (5b) and (2) \\
(7) & $\Some{Y}{\Not{\Eats(\Wolf, Y)} \Conj \Plant(Y)}$
\\ & --- from (6) \\
(8) & $\Not{\All{Y}{\Eats(\Wolf, Y) \Disj \Not{\Plant(Y)}}}$
\\ & --- from (7) \\
(9) & $\Not{\All{Y}{\Eats(\Wolf, Y) \Bimp \Plant(Y)}}$
\\ & --- from (8) \\
(10) & $\Not{\Herbivorous(\Wolf)}$
\\ & --- from (9) and definition of $\Herbivorous$ (3a) \\
(11) & $\Carnivorous(\Wolf)$
\\ & --- by (10) and (3) via resolution \\
\\
& (Deduce therefore that foxes are carnivorous.) \\
(12) & $\All{Y}{\Eats(\Wolf, Y) \Bimp
            \BiggerThan(\Wolf, Y) \Conj
            \Some{Z}{\Plant(Z) \Conj \Eats(Y, Z)}}$
\\ & --- by (11) and (3b) via modus ponens \\
(13) & $\All{Y}{\Not{\Eats(\Wolf, Y)} \Imp
            \Not{\BiggerThan(\Wolf, Y)} \Disj
            \Not{\Some{Z}{\Plant(Z) \Conj \Eats(Y, Z)}}}$
\\ & --- contrapositive of (12) \\
(14) & $\Not{\BiggerThan(\Wolf, \Fox)} \Disj
            \Not{\Some{Z}{\Plant(Z) \Conj \Eats(\Fox, Z)}}$
\\ & --- by (13) and (5a) via modus ponens \\
(15) & $\Not{\Some{Z}{\Plant(Z) \Conj \Eats(\Fox, Z)}}$
\\ & --- by (14) and (4a) via resolution \\
(16) & $\All{Z}{\Not{\Eats(\Fox, Z)} \Bimp \Plant(Z)}$
\\ & --- from (15) \\
(17) & $\Not{\Eats(\Fox, \Grain)}$
\\ & --- by (16) and (2) via modus ponens \\
(18) & $\Not{\Eats(\Fox, \Grain)} \Conj \Plant(\Grain)$
\\ & --- by (17) and (2) \\
(19) & $\Some{Y}{\Not{\Eats(\Fox, Y)} \Conj \Plant(Y)}$
\\ & --- by (18) \\
(20) & $\Not{\All{Y}{\Eats(\Fox, Y) \Bimp \Plant(Y)}}$
\\ & --- by (19) \\
(21) & $\Not{\Herbivorous(\Fox)}$
\\ & --- by definition of $\Herbivorous$ (3a) \\
(22) & $\Carnivorous(\Fox)$
\\ & --- by (21) and (3) via resolution \\
\end{tabular}
So we've identified that the foxes eat all animals that eat some kind of
plant.  All we have to do now is show that birds eat grain, and hence
that foxes eat birds, and we have proved the goal.

\begin{tabular}{rl}
(23) & $\All{Y}{\Eats(\Snail, Y) \Bimp \Plant(Y)}$
\\ & --- by definition of $\Herbivorous(\Snail)$ (3a) \\
(24) & $\Eats(\Snail, \Grain)$
\\ & --- by (23) and (2) via modus ponens \\
(25) & $\Plant(\Grain) \Conj \Eats(\Snail, \Grain)$
\\ & --- by (24) and (2) \\
(26) & $\Some{Z}{\Plant(Z) \Conj \Eats(\Snail, Z)}$
\\ & --- by (25) \\
(27) & $\Not{\Eats(\Bird, \Snail)} \Conj
        \BiggerThan(\Bird, \Snail) \Conj
        \Some{Z}{\Plant(Z) \Conj \Eats(\Snail, Z)}$
\\ & --- by (26), (5d) and (4d) \\
(28) & $\Some{Y}{
            \Not{\Eats(\Bird, Y)} \Conj
            \BiggerThan(\Bird, Y) \Conj
            \Some{Z}{\Plant(Z) \Conj \Eats(Y, Z)}}$
\\ & --- by (27) \\
(29) & $\Not{\All{Y}{
            \Eats(\Bird, Y) \Bimp
            \BiggerThan(\Bird, Y) \Conj
            \Some{Z}{\Plant(Z) \Conj \Eats(Y, Z)}}}$
\\ & --- by (28) \\
(30) & $\Not{\Carnivorous(\Bird)}$
\\ & --- by definition of $\Not{\Carnivorous(\Bird)}$ (3b) \\
(31) & $\Herbivorous(\Bird)$
\\ & --- by (30) and (3) via resolution \\
(32) & $\All{Y}{\Eats(\Bird, Y) \Bimp \Plant(Y)}$
\\ & --- by definition of $\Herbivorous(\Bird)$ (3a) \\
(33) & $\Eats(\Bird, \Grain)$
\\ & --- by (33) and (2) via modus ponens \\
(34) & $\Plant(\Grain) \Conj \Eats(\Bird, \Grain)$
\\ & --- by (33) and (2) \\
(35) & $\Some{Z}{\Plant(Z) \Conj \Eats(\Bird, Z)}$
\\ & --- by (34) \\
(36) & $\BiggerThan(\Fox, \Bird) \Conj
        \Some{Z}{\Plant(Z) \Conj \Eats(\Bird, Z)}$
\\ & --- by (35) and (4b) \\
(37) & $\All{Y}{\Eats(\Fox, Y) \Bimp
            \BiggerThan(\Fox, Y) \Conj
            \Some{Z}{\Plant(Z) \Conj \Eats(Y, Z)}}$
\\ & --- by (22) and definition of $\Carnivorous(\Fox)$ (3b) \\
(38) & $\Eats(\Fox, \Bird)$
\\ & --- by (37) and (36) via modus ponens \\
(39) & $\Eats(\Fox, \Bird) \Conj \Eats(\Bird, \Grain)$
\\ & --- by (38) and (33) \\
(40) & $\Some{X, Y}{\Eats(X, Y) \Conj \Eats(Y, \Grain)}$
\\ & --- by (39) \\
QED \\
\end{tabular}

Well, that was hard work (in practice, when writing a paper, we would
omit many of the smaller steps in the above.)  Still, we should now have
some sort of feel for first order logic.

The point of this exercise is XXX HERE

\subsection{Aside on higher order programming?}

\XXX{We can treat closures not as higher order terms, but rather as
structures containing (amongst other things) \emph{names} of predicates
which can be interpreted by some special machinery that handles higher
order application.}

\section{Using Predicate Logic for Computation}

Horn clauses.

Proof procedures.

Clark completion.

Negation and the CWA.

Modes.




% % vim: ft=tex ff=unix ts=4 sw=4 et wm=8 tw=0

\chapter{Basic Syntax and Terminology}



\section{Introduction}

Describing Mercury takes a number of terms that are well understood
in the declarative programming community, but are rarely, if ever, used
in the more common imperative schools.  This chapter introduces the bulk
of the terms the reader may not be familiar with, but which occurr
frequently throughout the rest of this book.  The intention is to
provide a ``gloss'' to aid comprehension of what follows; more full
explanations will be given as the terms are used formally for the first
time.



\section{Declarative and Imperative}

Languages like C, C++, Java and so forth are all \emph{imperative}
programming languages.

Imperative programming considers a program to be a \emph{sequence of
instructions}.  Run-time control-flow decisions are made on the basis of
the contents of the \emph{mutable variables} that represent the
\emph{state} of the program.  Instructions fall into three categories:
those that update mutable variables with the result of some computation;
those that change the control flow of the program depending upon the
state of those variables; and those that perform input/output (IO)
operations.

Languages like Mercury and Haskell are \emph{declarative} programming
languages.

Declarative programming considers a program to be a \emph{computable
relationship} between its inputs and outputs.  There are no mutable
variables in declarative programming; the term \emph{variable} instead
refers to the name given to the result of some computation.  Some
relationships are simple enough to be computed directly, such as
@X = Y + Z@ where any two of the values are known.  Other relationships
have to be expressed by the programmer in terms of combinations of these
very basic relationships.

The ``moral'' motivation behind declarative programming is that the
resulting program is merely a refinement of the statement of the
problem, whereas it is actually very difficult to say exactly what
relationship a program is computing, making it harder to decide whether
the program contains a bug or not.

The practical motivation is that you generally get more bang for your
buck in terms of progress made per line of code written.  Moreover,
because declarative programs are essentially just statements of
mathematics, compilers can and do perform extensive checking and
verification tests on the code.  A large class of bugs that are part of
the day-to-day life of the imperative programmer simply cannot be
expressed in a declarative language; those that can be are usually
caught by the compiler, well before one needs to fire up the debugger.

\XXX{I think I'm selling this too hard.  The speech should probably go
in the introduction.}



\section{Predicates, Modes and Determinism}

In logic, a \emph{predicate} is simply the name given to a relationship
between some number of variables.  For example, @father(C, F)@ might
represent the relationship of @F@ being the father of @C@.  @C@ and @F@
are said to be the \emph{parameters} of @father@.

Given any bindings for @F@ and @C@ we can ask whether @father(C, F)@ is
true or not.  Moreover, we may be able to give the bindings for only a
subset of the variables.  Say we fix @F@, we can ask what the possible
bindings for @C@ are that would make @father(C, F)@ true.  Conversely,
we might ask what the possible values of @F@ are for a given @C@.
Indeed, we may go one step further and specify \emph{neither} @F@ nor
@C@ and simply ask for all possible pairwise bindings with the
appropriate property.

We use the term \emph{determinism} to refer to the number of possible
solutions a predicate can have in a particular \emph{mode} (i.e. given
fixed values for certain the variables.)  A predicate is
\emph{deterministic} if there is exactly one solution (set of bindings)
for the remaining variables, otherwise it is \emph{nondeterministic}.
(We will see that in Mercury it is useful to further subdivide the
determinism categories.)

Informally, we refer to the arguments with given values as inputs and
those whose value is not fixed as outputs.

For example, @father@ is deterministic if we fix only @C@ since every child
has exactly one father.  However, if we fix only @F@ we may get any
number of possible bindings for @C@ depending upon how many children @F@
has had, hence this mode of @father@ is nondeterministic.

We use the term ``procedure'' (or sometimes overload ``mode'') to refer
to a predicate restricted to a given set of argument modes.

\subsection{Syntax}

Mercury understands several determinisms: @erroneous@, @semidet@, @det@,
@nondet@, @multi@, @cc_nondet@ and @cc_multi@.  These will be explained
in detail in \XXX{}; for now it suffices to know that @semidet@ and
@nondet@ can fail (i.e. have no solutions), @semidet@ may have a single
solution, @det@ has exactly one solution (i.e. it describes a
\emph{functional} relationship -- see below), and @nondet@ and @multi@
may have more than one solution.

The two most important built-in modes are @in@ and @out@, corresponding
to input and output arguments respectively.  Modes are discussed in
\XXX{}.

The mode declarations (one per procedure) for the @father@ predicate
described above would look like this:
\begin{verbatim}
    % father(Child, Father).
    %
:- mode father(in,  out) is det.
:- mode father(out, in ) is nondet.
:- mode father(in,  in ) is semidet.
\end{verbatim}

\XXX{I'm not sure how/when/where to introduce syntax in this chapter.
It may be best to merge this chapter with the main Mercury-as-it-is-
written chapter.}



\section{Functions and Expressions}

A deterministic procedure is said to describe a \emph{functional}
relationship between its inputs and its outputs (that is, the output is
uniquely determined for each possible input.)  For instance, the
sum @X = Y + Z@ is functional given any two of @X@, @Y@ or @Z@.

Special syntax exists for functions with a single (last) output
argument.  The main reason for this is that it supports \emph{functional
composition}, which allows us to avoid naming intermediate results.  For
example, rather than writing
\begin{verbatim}
    Tmp1 = A    + B,
    Tmp2 = Tmp1 - C,
    Tmp3 = Tmp2 * D,
    X    = Tmp3 / E
\end{verbatim}
we can simply write
\begin{verbatim}
    X = (((A + B) - C) * D) / E
\end{verbatim}
(in fact the precedence and associativity of the operators @+@, @-@,
@*@, @/@ etc. has been chosen so that parentheses can often be
avoided.)

An \emph{expression} describes either a simple value or the
\emph{application} of a function to other expressions.



\section{Goals}

A goal describes either a simple \emph{predicate call} or XXX HERE!



Procedures.

Goals.

Terms.

Variables.

Quantifiers.

Conjunction, disjunction, negation, conditional.

...




% % vim: ft=tex ff=unix ts=4 sw=4 et tw=76

\chapter{Types}

Mercury is both strongly and statically typed.  Strong typing ensures
that a program does not inadvertently compare apples with oranges, while
the purpose of static typing is to inform the programmer \emph{at
compile time} if a given program may attempt such a thing.  Types also
exist to provide a systematic way to structure data.

\XXX{Are the following paragraphs too pejorative?  Relevant this
early?  Needed at all?  Examples required?}

By way of comparison with other languages, C is not
strongly typed, since C allows unchecked casts between values of
different types (a \emph{cast} is a way of telling the compiler to
treat a particular value as if it had a different type.)

Python is strongly typed, but not statically type checked: every time a
Python program performs an addition, for instance, the arguments have to
be tested to verify that they are indeed numbers.

Java, on the other hand, \emph{is} both strongly and statically typed.
Unfortunately, though Java's type system is weak: \emph{checked} run time
casts are frequently necessary, meaning that many type errors will only
be spotted when a running program aborts with an exception.

Mercury does have these problems.  All type errors in
a Mercury program are identified at compile time, making it harder to
ship buggy programs.  An expressive type system makes Mercury programs more
efficient, since no type checks are necessary at run time.  The extra
information also allows the compiler to perform optimizations that would
otherwise be impossible.

There are three different sorts of type in Mercury: discriminated
unions, equivalence types and abstract types.  A \emph{discriminated
union} is a collection of values.  An
\emph{equivalence type} is simply a different name for another type.  An
\emph{abstract type} is one whose definition is hidden from its users.

Because Mercury's type system is rich there is a danger of confusing the
reader with too much, too soon.  Explanation of the more esoteric aspects of
the type system, namely existentially quantified types and type classes, is
deferred to chapter \XXX{}.

\section{The Primitive Types}

Mercury's built-in types are @int@, @float@, @char@, @string@, tuples and
the predicate and function types.

\subsection{Integers}

The @int@ type represents the integers -- @123@, @-9@, @42@ and so forth -- that
will fit into a machine word on the target computer (the range is from
about $-2$ billion to $+2$ billion on a 32-bit machine.)

All the core operations one can perform on @int@s, including the basic
arithmetic functions, are defined in the @int@ module in the Mercury
standard library.

Like most languages, Mercury has syntax for hexadecimal, octal, and binary
numbers, as well as for the integer codes for characters.  The interested
reader can find the details in the Mercury Reference Manual \XXX{}.

\subsection{Floating Point Numbers}

Floating point values such as @2.718@ and @3.0e8@ ($3.0\times10^8$) are
encoded by the @float@ type.  (The decimal point must always
appear in a floating point number and be preceded and followed by at
least one digit.)

The core @float@ operations are defined in the @float@ module in the
Mercury standard library.  Mathematical constants such as $\pi$ and $e$, and
things like trignometric functions, are defined in the @math@ module in the
Mercury standard library.

\subsection{Characters}

The @char@ type represents single characters.  Unfortunately, due to the
Mercury's heritage, there is no special syntax for single character
values.  Some characters, @a@, @b@, @c@, @1@, @2@, @3@ and so forth,
need no special quoting.
For others (space, tab and newline, for instance), one has to use
single quotes and write @' '@, @'\t'@ and
@'\n'@ -- single quote and backslash are @'\''@ and
@'\\'@ respectively.
Finally, some characters need to be enclosed in parentheses since
otherwise they might be confused with infix operators: @(+)@ and
@(*)@, for example.
The safest method is to use both parentheses \emph{and} quoting, as in
@('/')@.  We apologise for the inconvenience.

The core @char@ operations are defined in the @char@ module in the
Mercury standard libarary.

\subsection{Strings}

Mercury strings are immutable sequences of characters.  Examples include
@"bodacious"@ and @"new\nline"@.  To embed certain characters in a
string -- double quotes, backslashes, tabs and newlines
et cetera -- one has to \emph{escape} them with a preceding backslash:
@\"@, @\\@, @\t@ and @\n@.

If a string contains an explicit newline, as in
\begin{myverbatim}
    "there's a newline in here
there it is"
\end{myverbatim}
then the newline character is actually part of the string, just as
if we had written
\begin{myverbatim}
    "there's a newline in here\nthere it is"
\end{myverbatim}
However, if the \emph{last} character on a line is a non-escaped backslash
then the trailing newline is not taken to be part of the string, hence
\begin{myverbatim}
    "I see \
no newline"
\end{myverbatim}
is the same as
\begin{myverbatim}
    "I see no newline"
\end{myverbatim}

\XXX{I don't really want to mention the double-double quotes in a string
thing.}

It is even possible to include characters in a string by giving the
appropriate character code, although the reader is referred to the Mercury
Reference Manual \XXX{} for details.

The core @string@ operations are defined in the @string@ module in the
Mercury standard library.

\subsection{Tuples}

Tuples are the only primitive compound type in Mercury.  A tuple is a
vector of values between braces:
\begin{myverbatim}
    {"the base of natural logarithms", 'e', 2.718}
\end{myverbatim}
Tuples are useful for parcelling up data in situations where it would be
annoying to have to create a special-purpose type to do the job.
Tuples can have any number of arguments.

There is no tuple-specific library module: one can construct and
deconstruct them, and that's it.

\subsection{Function and Predicate Types}

Functions and predicates are first class citizens in the Mercury type
system: they can be passed as arguments, returned as results
and stored in other data structures.  This aspect of Mercury is
discussed in detail in chapter \XXX{} on higher order programming.

Function and predicate types look just like the corresponding @func@ and
@pred@ declarations with the name is omitted.
The @func@ declaration for @math.sqrt/1@, for example, is given as
\begin{myverbatim}
:- func sqrt(float) = float.
\end{myverbatim}
The type of @math.sqrt/1@ is therefore @func(float) = float@.

The @pred@ declaration for @set.contains/2@, which tests whether a
particular value is a member of a set, is declared as
\begin{myverbatim}
:- pred contains(set(T), T).
\end{myverbatim}
so @set.contains/2@ has the type @pred(set(T), T)@.
The @T@ is a \emph{type parameter} (about which we will say more
in a little while) meaning we can substitute any type at all for @T@ and
the predicate will still work.

Although we haven't yet covered higher order programming (see chapter
\XXX{}), we should still be able to understand higher order type
signatures.  @string.words/2@, for instance, has the type
\begin{myverbatim}
    func(pred(char), string) = list(string)
\end{myverbatim}
indicating that its first argument is a predicate over characters.
(Note that the \emph{type} of a function or predicate says nothing about
its \emph{mode} -- which arguments are inputs and its determinism and so
forth.  Modes are the subject of chapter \XXX{}.)

\section{Discriminated Union Types}

One can think of a discriminated union as being a set of values, each of
which is distinguishable from the others.  At their core, virtually all
Mercury types are distributed unions.
(Equivalence types are simply a means of giving convenient names to
complicated types and abstract types are used to hide the
definition of a type.)
\XXX{Should I qualify this by saying that types implemented in foreign
code aren't strictly DUs?}

We can define a new type to represent the suits in a deck of playing
cards like this:
\begin{myverbatim}
:- type suit ---> spades ; hearts ; diamonds ; clubs.
\end{myverbatim}
The new type is called @suit/0@ and it has four 
\emph{data constructors}, or possible values: @spades@, @hearts@,
@diamonds@ and @clubs@.  The goal
\begin{myverbatim}
    X = spades
\end{myverbatim}
will, assuming @X@ is uninstantiated, bind @X@ to the value @spades@ which
is of type @suit/0@.

Data constructors can take arguments, as illustrated by the following type
for binary trees of @int@s:
\begin{myverbatim}
:- type int_tree ---> empty ; branch(int, int_tree, int_tree).
\end{myverbatim}
The code for insertion and the membership test can be written like
this:
\begin{myverbatim}
:- func insert(int, int_tree) = int_tree.

insert(X, empty          ) = branch(X, empty, empty).

insert(X, branch(Y, L, R)) =
    (      if X =< Y then branch(Y, insert(X, L), R)
      else /* X  > Y */   branch(Y, L, insert(X, R))
    ).

:- pred int_tree `contains` int.
:- mode in       `contains` in is semidet.

branch(X, L, R) `contains` Y :-
    (   Y = X
    ;   Y < X,  L `contains` Y
    ;   Y > X,  R `contains` Y
    ).
\end{myverbatim}
(Observe the C-style @/*@ comment @*/@ in the definition of @insert/2@.)

As we've seen before, a data constructor in a program denotes either a
construction or a deconstruction -- exactly which is determined by the
modes and the context.  In @insert/2@, occurrences of @branch/3@ and
@empty/0@ in the head are deconstructions, while occurrences in the body are
constructions.

The definition for @contains/2@ uses a plain disjunction, not a switch:
it reads ``a @branch(X, L, R)@ contains @Y@ \emph{iff} @Y = X@
\emph{or} @Y < X@ and @L@ contains @Y@ \emph{or} @Y > X@ and @R@
contains @Y@.''
If any arm of the disjunction succeeds then the predicate as a
whole succeeds.  A clause for the @empty@ case is not necessary since it
would always fail.

It is quite common for a type to have a single constructor of the same
name:
\begin{myverbatim}
:- type date ---> date(int, int, int).  % Year, month, day.
\end{myverbatim}
Mercury will never get the type and the data constructor confused since
the type name can only appear in type declarations and the constructor
name can only appear in clauses.

\subsection{Data Constructors with Named Fields}

Like most languages, Mercury supports data structures with named fields.
Consider a type used to keep track of the tally of votes:
\begin{myverbatim}
:- type tally ---> tally(ayes :: int, nayes :: int).
\end{myverbatim}
naming the fields of the @tally/2@ data constructor @ayes/0@ and
@nayes/0@ respectively.  We can still construct and deconstruct @tally/2@
values as if it had been defined without named fields.  However, having
named fields also allows us to refer to them \emph{without} the need for an
explicit deconstruction.

\subsubsection{Accessing Fields}

If we have @X = tally(12, 7)@ then we can refer to the values of the fields
of the value bound to @X@ using @X ^ ayes@ and @X ^ nayes@:
\begin{myverbatim}
    X ^ ayes  = 12,
    X ^ nayes =  7
\end{myverbatim}
The expression @X ^ ayes@ is just syntactic sugar for @ayes(X)@.  The
function @ayes/1@ is automatically constructed from the field name in the
type definition and will be defined as
\begin{myverbatim}
ayes(tally(A, _)) = A.
\end{myverbatim}

\subsubsection{Updating Fields}

We can ``update'' a particular field like this:
\begin{myverbatim}
    Y = ( X ^ ayes := 13 )
\end{myverbatim}
giving @Y = tally(13, 7)@.  This, of course, does not affect the value
of @X@: we still have @X = tally(12, 7)@.  
@X ^ ayes := 13@ should be read as ``the value of @X@ with @13@
substituted in the @ayes@ field.''

The expression @X ^ ayes := 13@ is just syntactic sugar for
@'ayes :='(X, 13)@.  The function @'ayes :='/2@ is also automatically
constructed from the type definition.  (Anything at all can be used as a
functor name if it is enclosed in @'@single quotes@'@, which is why the
extra characters in the name do not lead to trouble.)  The definition of
the automatically created @'ayes :='/2@ would be
\begin{myverbatim}
'ayes :='(tally(_, B), A) = tally(A, B).
\end{myverbatim}

Updating several fields in one go is straightforward:
\begin{myverbatim}
    Z = (( Y ^ ayes  := Y ^ ayes  + 2 )
             ^ nayes := Y ^ nayes + 3 )
\end{myverbatim}
giving @Z = tally(15, 10)@.

\subsubsection{Fields Within Fields}

Consider the following:
\begin{myverbatim}
:- type car
    --->    car(
                make            :: string,
                registration    :: string,
                owner           :: person
            ).

:- type person
    --->    person(
                name            :: string,
                date_of_birth   :: date,
                address         :: string
            ).
\end{myverbatim}
and the bindings
\begin{myverbatim}
    Fred = person(
               "Fred Bloggs",
               date(1965, 7, 4),
               "11 Strangetrousers Terrace"
           ),
    Car  = car( 
               "Ford Escort",
               "PQR 123",
               Fred
           )
\end{myverbatim}
We can obtain the @name@ of the @owner@ of @Car@ with
\begin{myverbatim}
    Car ^ owner ^ name = "Fred Bloggs"
\end{myverbatim}
@Car ^ owner ^ name@ is the same as writing @(Car ^ owner) ^ name@.

If Fred moves house and we need to update his address then we can write
\begin{myverbatim}
    Car1 = (Car ^ owner ^ address := "42 Strawberry Fields")
\end{myverbatim}
which is the same as if we'd written
\begin{myverbatim}
    Owner  = Car ^ owner,
    Owner1 = (Owner ^ address := "42 Strawberry Fields"),
    Car1   = (Car   ^ owner   := Owner1                )
\end{myverbatim}
So the parentheses matter.  If we just wanted an updated version of the
@owner@ field of @Car@, rather than an updated version of @Car@ itself,
we'd write
\begin{myverbatim}
    Owner1 = ((Car ^ owner) ^ address := "42 Strawberry Fields")
\end{myverbatim}
which this time is the same as having written
\begin{myverbatim}
    Owner  = Car ^ owner,
    Owner1 = (Owner ^ address := "42 Strawberry Fields")
\end{myverbatim}

\subsubsection{Mixing Named and Unnamed Fields}

It is not necessary to name every field in a data constructor:
\begin{myverbatim}
:- type person
    --->    person(
                name            :: string,
                date_of_birth   :: date,
                address         :: string,
                children        :: list(child),
                /* secrets */      list(secret)
            ).
\end{myverbatim}
The @list(secret)@ field is not named and, unlike the named fields, can
therefore only be accessed by deconstruction and ``updated'' by
construction.

\subsubsection{User-Defined Field Access Functions}

It can be useful to explicitly define functions to be used for field access,
either to provide ``virtual'' fields that are computed rather than stored
(C$\sharp$ calls such things \emph{properties}), or to provide sanity checks
on accesses and updates, or to define ``indexed fields''.

Say we want to add a ``virtual'' field to the @person/0@ type, computing the
number of children a particular individual has.
(Virtual field access functions do not represent real fields, of course, and
so don't appear as field names in the type definition.)
To do this we simply write
\begin{verbatim}
:- func person ^ num_children = int.

Person ^ num_children = list.length(Person ^ children).
\end{verbatim}

We can add a sanity check to the @children@ field's update function
ensuring that a person never \emph{loses} children when the list is updated:
(here we are just providing our own definition for the @children/2@ field
access function; Mercury will not construct a definition for a field access
or update function if we have already provided one)
\begin{verbatim}
:- func (person ^ children := list(child)) = person.

(person(A, B, C, D0, E) ^ children := D) = Person :-
    ( if all [Child] (
            list.member(Child, D0) => list.member(Child, D)
      ) 
      then Person = person(A, B, C, D, E)
      else exception.throw("you can't unhave children!")
    ).
\end{verbatim}
(The condition of the conditional goal reads ``for all @Child@, \emph{if}
@Child@ is a member of @D0@ \emph{then} @Child@ is a member of @D@.'')

We can also add ``virtual, indexed'' fields.  Say, for example, we wish to
access a @person@'s @children@ by number.  Then we might write
\begin{verbatim}
:- func person ^ child(int) = child.

Person ^ child(N) = list.index0_det(Person ^ children, N).
\end{verbatim}
where @list.index0_det(Children, N)@ returns the @N@th member of @Children@,
(counting the first as zero, in traditional computer science style) or
throws an exception if @N@ is out of range.

(The expression @Person ^ child(N)@ is syntactic sugar for
@child(N, Person)@.)

\XXX{Should I mention @map.elem/2@ etc. here as examples?}

Virtul indexed fields may have any number of arguments.  For example, the
two-dimensional array module, @array2d@ in the Mercury standard library,
defines @Array2D ^ elem(Row, Column)@ to access elements of the array.

\subsubsection{Restrictions on Field Names}

At the time of writing, Mercury requires that no field name be defined twice
in the same module in order to avoid problems with ambiguity.  The following
are therefore all errors.

This example is illegal because it uses the same field name twice in the
same data constructor:
\begin{verbatim}
:- type foo ---> foo(a :: int, a :: int).
\end{verbatim}
The next example uses the same field name in two different data
constructors and is therefore illegal, despite the fact that they belong to
the same type:
\begin{verbatim}
:- type foo ---> foo1(a :: int)
            ;    foo2(a :: int).
\end{verbatim}
Finally, one cannot use the same field name twice, even in different types:
\begin{verbatim}
:- type foo ---> foo(a :: int).
:- type bar ---> bar(a :: int).
\end{verbatim}
There is no problem with different \emph{modules} using the same field name
since the module qualified forms of the name will be different.

(The names @foo@, @bar@, @baz@ and @quux@ are conventionally used by
computer scientists when they can't think of anything better to use in an
example.)

\subsubsection{Semideterministic Field Access}

Consider the following type:
\begin{myverbatim}
:- type data ---> empty ; datum(payload :: int, rest :: data).
\end{myverbatim}
Given an @X@ of type @data/0@, the expression @X ^ payload@, may
\emph{fail}.  The field access @X ^ payload@ can only succeed if @X@ is
bound to a @datum/2@ value -- if @X@ is bound to @empty@ then there is no
@payload@ field.

Field accesses for field names that do not occur in all the data
constructors of a type are therefore \emph{semideterministic}.  Be aware of
his point, as it can from time to time lead to confusing error messages from
the compiler.

\XXX{Do I need to say anything about using pattern matching to resolve the
issue?}

\subsection{Parameterised Discriminated Union Types}

So far we've only described ``concrete'' types.  It turns out to be very
useful to generalise over whole families of types.  The binary
tree \emph{structure} used in @int_tree/0@, for instance, would also work
fine for strings, floating
point numbers and pretty much anything else.  Rather than defining
nearly identical types (\eg @string_tree/0@ and @float_tree/0@) every time
we need them, it is far better to \emph{parameterise} our definition, thus:
\begin{verbatim}
:- type tree(T) ---> empty ; branch(T, tree(T), tree(T)).
\end{verbatim}
A value of type @tree(T)@, then, is either @empty@ or a @branch(X, Y, Z)@
where @X@ is of type @T@ and @Y@ and @Z@ are of type @tree(T)@.

Since we are now working over arbitrary types rather than plain @int@s, we
need to use \emph{generic} comparison predicates for the insert and look-up
operations:
\begin{verbatim}
:- func insert(T, tree(T)) = tree(T).

insert(X, empty          ) = branch(X, empty, empty).

insert(X, branch(Y, L, R)) =
    (      if X @=< Y then branch(Y, insert(X, L), R)
      else /* X @>  Y */   branch(Y, L, insert(X, R))
    ).

:- pred tree(T) `contains` T.
:- mode in      `contains` in is semidet.

branch(X, L, R) `contains` Y :-
    O = ordering(X, Y),
    (   O = (=)
    ;   O = (<),    R `contains` Y
    ;   O = (>),    L `contains` Y
    ).
\end{verbatim}
The predicates, \verb!@<!, \verb!@=<!, \verb!@>=!, and \verb!@>!, 
compare two terms in Mercury's \emph{standard ordering}.
It doesn't matter what that ordering is for our purposes here, although the
curious reader can find out more from the Mercury Reference Manual
\XXX{Put this documentation in the Refence Manual!}.
The function
@ordering(X, Y)@ returns @(=)@ if @X = Y@, @(<)@ if \verb!X @< Y!, and
@(>)@ if \verb!X @> Y!.

We can substitute any type we like for the parameter @T@ in @tree(T)@:
@tree(int)@ is a binary tree of integers, @tree(string)@ is a binary tree of
strings, and so on.  The implementations for @insert/2@ and @contains/2@
will work regardless of which type is substituted for @T@, something we
could not have managed if we'd stuck to using @int_tree/0@, @string_tree/0@
and so forth.
Moreover, Mercury will tell us at compile time if we ever attempt to insert
a @string@, say, into a @tree(int)@ (compare this with Java, say, where such
a problem would only become apparent when the running program threw an
exception.)

A type may have more than one parameter.  We might want to
generalise our @tree/1@ type to store arbitrary key-value mappings to form a
dictionary structure:
\begin{verbatim}
:- type dict(K, V)
    --->    empty
    ;       branch(K, V, dict(K, V), dict(K, V)).
\end{verbatim}
The insert and lookup operations are similar to their @tree/1@ counterparts:
\begin{verbatim}
:- func insert(K, V, dict(K, V)) = dict(K, V).

insert(Ka, Va, empty) =
    branch(Ka, Va, empty, empty).

insert(Ka, Va, branch(Kb, Vb, L, R)) = Tree :-
    O = ordering(Ka, Kb),
    (   O = (=),    Tree = branch(Ka, Va, L, R)
    ;   O = (<),    Tree = branch(Kb, Vb, insert(Ka, Va, L), R)
    ;   O = (>),    Tree = branch(Kb, Vb, L, insert(Ka, Va, R))
    ).

:- pred lookup(dict(K, V), K,  V).
:- mode lookup(in,         in, out) is semidet.

lookup(branch(Ka, Va, L, R), Kb, Vb) :-
    O = ordering(Ka, Kb),
    (   O = (=),    Vb = Va
    ;   O = (<),    lookup(R, Kb, Vb)
    ;   O = (>),    lookup(L, Kb, Vb)
    ).
\end{verbatim}
It is a requirement that any type variables appearing in a data constructor
in a type definition must be parameters of the type.  The following
violates the rule and would be reported as an error:
\begin{verbatim}
:- type awful ---> mistake(T).
\end{verbatim}
The justification is that Mercury has to be able to work out what type @X@
has in @mistake(X)@ in every deconstruction.
This definition,
were it legal, says that @X@ could be anything at all
-- there is, in general, simply no way for the compiler to work out the
actual type ahead of time.
(Chapter \XXX{} explains existentially quantified types which \emph{can} be
used to construct heterogeneous collections.)

\XXX{Should I mention the shadow-types ``pattern''?}

\section{Equivalence Types}

Imagine we are writing a personnel database relating people to lists of
various attributes and that we have already defined the types @person/0@ and
@attribute/0@.
The @dict/2@ type defined above is a likely candidate for the job.
Specifically, we'd want to use
\begin{verbatim}
    dict(person, list(attribute))
\end{verbatim}
This is something of a mouthful.  Typing this everywhere we needed it would be
tiring and, worse, a maintenance problem if we later decide that, say,
@set(attribute)@ is a better idea than @list(attribute)@.

The Mercury solution to the problem is to use an equivalence type:
\begin{verbatim}
:- type persattrs == dict(person, list(attribute)).
\end{verbatim}
and hereafter we can write @persattrs@ as shorthand for the type on the
right hand side of the @==@ symbol.  It makes no difference to Mercury
whether you use @persattrs@ or the full type name.

A common idiom is to use equivalence types to give use-specific names
to primitive types.  For instance
\begin{verbatim}
:- type name         == string. % Must be non-empty.
:- type age          == int.    % Must be non-negative.
:- type num_children == int.    % Must be non-negative.
\end{verbatim}
@type@, @pred@ and @func@ declarations that use these names
appropriately become much more readable.

Equivalence types may also be parameterised.  For example, the following
describes a suitable representation for graph structures with labelled
vertices:
\begin{verbatim}
:- type graph(T) == map(vertex(T), list(vertex(T))).
\end{verbatim}
(We assume the type @vertex/1@ has been defined elsewhere.)
\XXX{Do we need more explanation of this type?  Or a simpler example?}

Note that, as with discriminated union types, any type variable appearing on
the right hand side of the @==@ must also appear as a parameter on the left
hand side.  Thus the following is in error:
\begin{verbatim}
:- type woeful == list(T).
\end{verbatim}

\section{Abstract Types}

It's not always a good to reveal the \emph{definition} of a type to its
users.  For one thing, doing so means one cannot later change the
definition of the type without the risk of breaking other programs that use
it.  It is a good engineering principle to hide 
implementation detail from users wherever possible.

The @dict/2@ dictionary data type we defined earlier is a good candidate
for a library module that we can re-use over and over again.  The
implementation details of @dict/2@ are not important to the users of the
module: they will only be interested in the functionality on offer.
Furthermore, we want the freedom to change the definition of @dict/2@ at
some later point if we find a more efficient representation -- and we want
to be able to do so without requiring existing users to make changes to
their code.

The solution to the problem is to make @dict/2@ an \emph{abstract type}.
\begin{myverbatim}
:- module dict.
:- interface.


:- type dict(K, V).     % Abstract type.

:- func new_dict = dict(K, V).

:- func insert(K, V, dict(K, V)) = dict(K, V).

:- pred lookup(dict(K, V), K,  V  ).
:- mode lookup(in,         in, out) is semidet.


:- implementation.


:- type dict(K, V)      % Definition of abstract type.
    --->    empty
    ;       branch(K, V, dict(K, V), dict(K, V)).

new_dict = empty.

...
\end{myverbatim}
The line @:- type dict(K, V).@ declares @dict/2@ to be an abstract type:
an abstract type is just one whose type declaration omits the definition.
The full type definition instead appears in the implementation section.

Since the users of an abstract type have no idea how it is defined, they can
neither create values of that type directly via construction nor examine
them with pattern matching (deconstruction).  The only way to manipulate
such values is to use the operations exported by the module defining the
abstract type.

This module includes the function @new_dict/0@ so that users can create new,
empty @dict/2@ values in the first place (functions of arity zero are just
fixed values.)
Other than that, users can only extend @dict/2@ values with @insert/3@ and
examine them with @lookup/3@.
The following is therefore wrong:
\begin{myverbatim}
:- import_module dict.

this_wont_compile :-
    Dict0 = empty,
    Dict1 = branch("Gateau", 'x', Dict0, Dict0),
    ...
\end{myverbatim}
because the implementation section of the @dict@ module is the \emph{only}
place that knows what the @dict/2@ data constructors are.  This, however, 
is fine:
\begin{myverbatim}
:- import_module dict.

this_will_compile :-
    Dict0 = new_dict,
    Dict1 = dict.insert("Gateau", 'x', Dict0),
    ...
\end{myverbatim}

\section{Explicit Type Qualification}

The types of the \emph{head variables} in a clause (\ie those appearing in
the clause head) are given in the corresponding @pred@ or @func@
declaration.  The compiler deduces the types of the \emph{local} variables
via a process known as \emph{type inference}.  Very occasionally, type
inference alone is not sufficient to unambiguously decide the type of a
local variable.

\subsection{Name Ambiguity}

Sometimes these problems arise due to overloading.  Say we have a module
@a@ exporting a predicate called @num/1@,
\begin{myverbatim}
:- module a.
:- interface.

:- pred num(int).
:- mode num(out) is multi.

:- implementation.

num(1).
num(2).
num(3).
\end{myverbatim}
and a module @b@ also exporting a predicate called @num/1@,
\begin{myverbatim}
:- module b.
:- interface.

:- pred num(float).
:- mode num(out) is multi.

:- implementation.

num(1.414).
num(2.718).
num(3.141).
\end{myverbatim}
and a module that imports both @a@ and @b@, defining a predicate @p/0@
testing whether there are solutions for @num/1@ whose sum is also a
solution:
\begin{myverbatim}
:- import int, float, a, b.

:- pred p.
:- mode p is semidet.

p :-
    num(X),
    num(Y),
    num(X + Y).
\end{myverbatim}
Since the code here does not module qualify @num/1@, the compiler cannot
decide if the programmer means @a.num/1@ and @int.(+)/2@ or @b.num/1@ and
@float.(+)/2@.

There are two ways to solve the problem: either one can explicitly module
qualify the names (not always convenient) or one can add explicit type
qualifiers to local variables and/or expressions to provide the compiler
with sufficient information to solve the problem.  Explicit type
qualification connects an expression to a type via the @with_type@ keyword.
The type ambiguity in @p/0@ above can be fixed with
\begin{myverbatim}
p :-
    num(X `with_type` int),
    num(Y `with_type` int),
    num((X + Y) `with_type` int).
\end{myverbatim}
although that would be overkill.  In fact only one type qualification is
necessary, for instance
\begin{myverbatim}
p :-
    num(X),
    num(Y),
    num((X + Y) `with_type` int).
\end{myverbatim}
The compiler will reason that since the result of @X + Y@ must be an
@int@, then what is meant here is @int:(+)/2@.  This means that @X@ and @Y@
must also have type @int@ (since the type of @int:(+)/2@ is
@func(int, int) = int@) and therefore that the calls to @num/1@ must refer
to @a.num/1@ (and not @b.num/1@ which has type @pred(float)@.)

\subsection{Type Ambiguity}

Here is an example of type ambiguity that arises for different reasons:
\begin{myverbatim}
main(!IO) :-
    io.read(Result, !IO),
    (
        Result = ok(X),
        io.print(X, !IO)
    ;
        Result = eof
    ;
        Result = error(_, _),
        throw(Result)
    ).
\end{myverbatim}
The predicates @io.read/3@ and @io.print/3@ read and write values of
virtually \emph{any} type.  However, the compiler needs to be able to work
out the type of @X@ before it can compile @main/2@ (chapter \XXX{} on
run-time type information explains why this is the case.)  Our program,
however, gives no clue as to what type @X@ is supposed to have and the
compiler will issue an error message of the form
\begin{myverbatim}
oops.m:017: In predicate `oops.main/2':
oops.m:017:   warning: unresolved polymorphism.
oops.m:017:   The variables with unbound types were:
oops.m:017:       X :: T
oops.m:017:       Result :: (io.read_result(T))
\end{myverbatim}
\XXX{I think the compiler should use @`with\_type`@ rather than @::@.}

Let's say we intended @X@ to be an @int@.
To resolve this problem we need to add an explicit type qualifier to @X@ --
it doesn't matter where, since @X@ can never change its type -- and it's
usually a good idea to pick the first occurrence:
\begin{myverbatim}
main(!IO) :-
    io.read(Result, !IO),
    (
        Result = ok(X `with_type` int),
        io.print(X, !IO)
    ;
        ...
    ).
\end{myverbatim}

\section{Types with User Defined Equality}

\XXX{Should almost certainly appear later in the book.}

\section{Types with User Defined Comparison}

\XXX{Should almost certainly appear later in the book.}

\section{Conclusion}

\XXX{Some kind of summary.}

% % vim: ft=tex ff=unix ts=4 sw=4 et wm=8 tw=0

\chapter{Predicates}

The key computational unit in Mercury is the predicate.  A
predicate describes a relationship between its arguments; it is
also possible to provide a procedural view of Mercury predicates,
which is how they can be used in a programming language.

Predicates cover not only tests and functions, as found in other
languages, but also procedures with multiple return values and
even non-deterministic relationships, in which there may be
several different solutions for a given set of input arguments.

\section{Introduction}

A predicate is defined by a set of \emph{clauses}, where each
clause takes the form
\begin{verbatim}
Head :- Body.
\end{verbatim}
This should be read as saying ``@Head@ is true if @Body@ is true'' with
the set of clauses forming an exhaustive specification of the
predicate.

If the body of a predicate is empty (taken to mean just @true@), one can
just write
\begin{verbatim}
Head.
\end{verbatim}
Clauses like this are called \emph{facts}.

The body of a predicate clause is a \emph{goal} -- a logical formula
that must be satisfied in order for the head to hold.

A very simple example of a predicate is
\begin{verbatim}
:- pred max(int, int, int).         % A, B, Max.
:- mode max(in,  in,  out) is det.

max(A, B, Max) :-
    ( if B < A then Max = A else Max = B ).
\end{verbatim}
This predicate takes two integers, @A@ and @B@, and succeeds
binding @Max@ to @A@ if @A@ is greater than @B@ or binding @Max@ to @B@
otherwise.  The part of the mode declaration @is det@ means
that this is a deterministic predicate: it will always succeed
and has just one solution for any given pair of inputs.

(This sort of deterministic predicate with a single output is
called a \emph{function}.  Functions are so common that Mercury
includes special syntax and conventions to simplify working
with them.  More of this in a later section \XXX{}.)

For a more interesting example, consider the following small
genealogical database of people defined by the @person@ type:
\begin{verbatim}
:- type person
    --->    arthur  ; bill      ; carl
    ;       alice   ; betty     ; cissy.
\end{verbatim}
We start by defining predicates that we can use to decide if a
particular person is male or female.  These predicates are
labelled semidet because they have at most one solution
(success) for a given argument, but might also fail.
\begin{verbatim}
:- pred male(person).
:- mode male(in) is semidet.

male(arthur).
male(bill).
male(carl).

:- pred female(person).
:- mode female(in) is semidet.

female(Person) :- not male(Person).
\end{verbatim}
The definition for @female/1@ states that @female(Person)@ succeeds
iff\footnote{\emph{Iff}: if and only if} @male(Person)@ fails.  (Recall
that variables in Mercury are distinguished by starting with a capital
letter.)
\begin{verbatim}
:- pred father(person, person).     % Child, Father.
:- mode father(in,     out) is semidet.

father(betty, arthur).
father(carl,  bill).
father(cissy, bill).

:- pred mother(person, person).     % Child, Mother.
:- mode mother(in,     out) is semidet.

mother(bill,  alice).
mother(carl,  betty).
mother(cissy, betty).
\end{verbatim}
The predicates @father/2@ and @mother/2@ take their first argument
as an input and return a result in the second.  The
determinism is semidet again because they are not exhaustive:
some inputs can cause them to fail (\eg there is no solution
for @father(arthur, X)@.)
\begin{verbatim}
:- pred parent(person, person).     % Child, Parent.
:- mode parent(in,     out) is nondet.

parent(Child, Parent) :- father(Child, Parent).
parent(Child, Parent) :- mother(Child, Parent).
\end{verbatim}
This simply says that the parent of a child is either the
father or the mother.  The determinism is nondet because for a
given @Child@ argument this predicate may fail or may have more
than one solution:
\begin{itemize}
\item both @father/2@ and @mother/2@ can fail
(\eg if @Child = arthur@);
\item just one of @father/2@ or @mother/2@ could succeed
(\eg if @Child = bill@);
\item both @father/2@ and @mother/2@ could succeed
(\eg if @Child = cissy@).
\end{itemize}

(How failure and multiple solutions affect the execution of a
Mercury program will be explained below.)
\begin{verbatim}
:- pred ancestor(person, person).   % Person, Ancestor.
:- mode ancestor(in,     out) is nondet.

ancestor(Person, Ancestor) :-
    parent(Person, Ancestor).

ancestor(Person, Ancestor) :-
    parent(Person, Parent),
    ancestor(Parent, Ancestor).
\end{verbatim}
We can now define an ancestor of a Person as being either
\begin{itemize}
\item a parent of that Person or
\item an ancestor of a parent of that Person.
\end{itemize}

Since @parent/2@ is nondet, so is @ancestor/2@.

This table explains the difference between the number of
solutions a predicate can have for a given determinism:

\begin{tabular}{lll}
\textbf{Determinism}       & \multicolumn{2}{l}{\textbf{Number of Solutions}} \\
            & \textbf{Min} & \textbf{Max} \\
\hline \\
@semidet@   & 0            & 1 \\
@nondet@    & 0            & 1 or more \\
@det@       & 1            & 1 \\
@multi@     & 1            & 1 or more \\
\end{tabular}

(There are a few other determinisms, but we don't need to
consider them just yet \XXX{}.)

The compiler will check that your determinism declarations are
correct.

One interesting thing about this database is that there's no
reason why it shouldn't be run in ``reverse''.  That is, for any
father or mother, we should be able to deduce who their
children are and for any ancestor we should be able to work
out who their descendants are.

To gain this extra functionality we have only to add the
required extra mode declarations; there is no need to touch
the definitions themselves!  The extra mode declarations are
\begin{verbatim}
:- mode father(out, in) is nondet.   % Infer Child from Father.
:- mode mother(out, in) is nondet.   % Infer Child from Mother.
:- mode parent(out, in) is nondet.   % Infer Child from Parent.
:- mode ancestor(out, in) is nondet. % Infer Person from Ancestor.
\end{verbatim}
Notice that @father/2@ and @mother/2@ are nondet in this mode
rather than semidet: looking at the definitions we see that
@father(Child, bill)@ has multiple solutions @Child = carl@ and
@Child = sissy@, while @mother(Child, betty)@ also has solutions
@Child = carl@ and @Child = sissy@.

There is no reason why a predicate cannot have several output
arguments or even no input arguments.  For example, we
might go on to add
\begin{verbatim}
:- pred parents(person, person, person).    % Child, Mother, Father.
:- mode parents(in,     out,    out   ) is semidet.
:- mode parents(out,    in,     in    ) is nondet.

parents(Child, Mother, Father) :-
    mother(Child, Mother),
    father(Child, Father).
\end{verbatim}
The first mode of @parents/3@ is semidet because both @mother/2@
and @father/2@ are semidet when called with @Child@ as an input
and both must succeed for @parents/3@ to succeed.

The second mode is nondet for similar reasons, but exhibits an
interesting property.  At first glance you might think that
something will go wrong here: when @parents/3@ is called in the
second mode, initially @Mother@ and @Father@ are inputs and @Child@
is an output.  However, if the call to @mother/2@ succeeds, then
@Child@ will end up being bound to the identifier for some
person.  This appears that we would then be trying to call
@father/2@ with \emph{both} arguments as inputs, but there is no such
mode declaration for @father/2@.  There is no need to worry,
however -- the compiler will recognise the situation and treat
the definition of @parents/3@ like this as far as the second
mode is concerned:\footnote{The compiler isn't doing any special analysis here;
this is just a natural consequence of conversion to
super-homogeneous normal form, which will be explained later
on \XXX{}.}
\begin{verbatim}
:- mode parents(out, in, in) is nondet.

parents(Child, Mother, Father) :-
    mother(Child, Mother),
    father(X,     Father),
    X = Child.
\end{verbatim}
So here @father/2@ is being called in mode @(out, in) is nondet@ and
@parents/3@ succeeds if the result @X@ is bound to the same value
as @Child@.

In effect the compiler has deduced that @father/2@ has the
implied mode
\begin{verbatim}
:- mode father(in, in) is semidet.
\end{verbatim}
In general, Mercury will reorder each clause body for each mode
declaration so that results are generated before they are needed -- each
mode of a predicate is referred to as a \emph{procedure}.

\section{Execution}
This section explains Mercury's execution model in more
detail.  In particular, the notions of failure, backtracking
and non-determinism are addressed.

Let's start by looking at some of the code for our
genealogical database again -- this time we're going to label
the clauses to help us see how programs execute in Mercury.
\begin{verbatim}

    :- pred father(person, person).     % Child, Father.
    :- mode father(in,     out   ) is semidet.
    :- mode father(out,    in    ) is nondet.

f1  father(betty, arthur).
f2  father(carl,  bill).
f3  father(cissy, bill).

    :- pred mother(person, person).     % Child, Mother.
    :- mode mother(in,     out   ) is semidet.
    :- mode mother(out,    in    ) is nondet.

m1  mother(bill,  alice).
m2  mother(carl,  betty).
m3  mother(cissy, betty).

    :- pred parent(person, person).     % Child, Parent.
    :- mode parent(in,     out   ) is nondet.
    :- mode parent(out,    in    ) is nondet.

p1  parent(Child, Parent) :- father(Child, Parent).
p2  parent(Child, Parent) :- mother(Child, Parent).

    :- pred ancestor(person, person).   % Person, Ancestor.
    :- mode ancestor(in,     out   ) is nondet.

a1  ancestor(Person, Ancestor) :-
        parent(Person, Ancestor).

a2  ancestor(Person, Ancestor) :-
        parent(Person, Parent),
        ancestor(Parent, Ancestor).

    :- pred parents(person, person, person).    % Child, Mother, Father.
    :- mode parents(in,     out,    out   ) is semidet.
    :- mode parents(out,    in,     in    ) is nondet.

s1  parents(Child, Mother, Father) :-
        mother(Child, Mother),
        father(Child, Father).
\end{verbatim}
First off, we'll consider the goal @parent(cissy, Parent)@.
Every time we see a goal we try expanding it according to each
clause definition in turn (this is what being referentially
transparent is all about).  If we get to a dead end we have to
\emph{backtrack} to the last point where we had a choice and try a
different clause (when we backtrack we also forget any
variable bindings we might have made one the way from the
/choice point/.)
\begin{verbatim}
        parent(cissy, Parent)
(by p1)     father(cissy, Parent)
(by f1)         false because betty \= cissy
(by f2)         false because carl  \= cissy
(by f3)         Parent = bill
\end{verbatim}
So one solution to @parent(cissy, Parent)@ is @Parent = bill@.  We
obtained this by first expanding the goal according to p1 and
then expanding the resulting goal @father(cissy, Parent)@
according to each of @f1@, @f2@ and @f3@ until we found one that
succeeded.

If the program later has to backtrack over this goal then we
have to forget the binding @Parent = bill@ and try the next
clause for @parent/2@ (since there are no more clauses for
@father/2@):
\begin{verbatim}
        parent(cissy, Parent)
(by p1)     mother(cissy, Parent)
(by m1)         false because bill \= cissy
(by m2)         false because carl \= cissy
(by m3)         Parent = betty
\end{verbatim}
Now, say we wanted to ask the question ``which ancestor of
@carl@, if any, is also a parent of @bill@?''  Here's how things
would proceed:\footnote{We have to supply fresh local vars in each
 expansion -- for instance, @Ancestor@ in the expansion of @a2@ has
 been replaced with a new variable @Z@.}
\begin{verbatim}
        ancestor(carl, X), parent(bill, Y), X = Y
(by a1)     parent(carl, X), parent(bill, Y), X = Y
(by p1)         father(carl, X), parent(bill, Y), X = Y
(by f1)             false because betty \= carl
(by f2)             X = bill, parent(bill, Y), X = Y
(  =  )             parent(bill, Y), bill = Y
(by p1)                 father(bill, Y), bill = Y
(by f1)                     false because bill \= betty
(by f2)                     false because bill \= carl
(by f3)                     false because bill \= cissy
(by p2)                 mother(bill, Y), bill = Y
(by m1)                     Y = alice, bill = Y
(  =  )                     bill = alice
(  =  )                     false because bill \= alice
(by m2)                     false because bill \= carl
(by m3)                     false because bill \= cissy
(by p2)         mother(carl, X), parent(bill, Y), X = Y
(by m1)             false because carl \= bill
(by m2)             X = betty, parent(bill, Y), X = Y
(  =  )             parent(bill, Y), betty = Y
(by p1)                 father(bill, Y), betty = Y
(by f1)                     false because bill \= betty
(by f2)                     false because bill \= carl
(by f3)                     false because bill \= cissy
(by p2)                 mother(bill, Y), betty = Y
(by m1)                     Y = alice, betty = Y
(  =  )                     betty = alice
(  =  )                     false because alice \= betty
(by m2)                     false because bill \= carl
(by m3)                     false because bill \= cissy
(by a2)     parent(carl, Z), ancestor(Z, X), parent(bill, Y), X = Y
(by p1)         father(carl, Z), ancestor(Z, X), parent(bill, Y), X = Y
(by f1)             false because carl \= betty
(by f2)             Z = bill, ancestor(Z, X), parent(bill, Y), X = Y
(  =  )             ancestor(bill, X), parent(bill, Y), X = Y
(by a1)                 parent(bill, X), parent(bill, Y), X = Y
(by p1)                     father(bill, X), parent(bill, Y), X = Y
(by f1)                         false because bill \= betty
(by f2)                         false because bill \= carl
(by f3)                         false because bill \= cissy
(by p2)                     mother(bill, X), parent(bill, Y), X = Y
(by m1)                         X = alice, parent(bill, Y), X = Y
(  =  )                         parent(bill, Y), alice = Y
(by p1)                             father(bill, Y), alice = Y
(by f1)                                 false because bill \= betty
(by f2)                                 false because bill \= carl
(by f3)                                 false because bill \= cissy
(by p2)                             mother(bill, Y), alice = Y
(by m1)                                 Y = alice, alice = Y
(  =  )                                 alice = alice
(  =  )                                 true
\end{verbatim}
So our program concludes that a solution to
\begin{verbatim}
    ancestor(carl, X), parent(bill, Y), X = Y
\end{verbatim}
is
\begin{verbatim}
    X = alice, Y = alice
\end{verbatim}
As we indicated earlier in the definition of @parents/3@, in
practice we'd be more likely to write the goal as just
\begin{verbatim}
    ancestor(carl, X), parent(bill, X)
\end{verbatim}
and let Mercury work out where the implicit unification should
go.

\Aside{The above example might give the impression that Mercury is a term
rewriting system.  This is not true (although conceivably a very
slow Mercury implementation might use such a technique\ldots)  What
happens behind the scenes is much more efficient, albeit
computationally equivalent.}

In practice, ninety per cent or so of the code one writes is
deterministic.  However, there are times (as in the example above) when
non-determinism can be used to write very clear, concise programs.
Support for backtracking and unification is what distinguishes Mercury
from purely functional languages.

\section{Recursion}

Imperative languages like C and Java provide a whole slew of
mechanisms for supporting iterative (looping) control flow --
while loops, repeat-until loops, for loops and so forth.

Declarative languages typically provide but one mechanism for
such things: \emph{recursion}.  A recursive predicate is simply one
defined in terms of itself \XXX{Don't forget to mention mutual
recursion}.  (Later on we will discover \emph{higher order} predicates
and observe that the standard library supplies predicates that stand in
for the common iterative mechanisms found in imperative languages.)

Some simple examples:
\begin{verbatim}
:- func factorial(int) = int.

factorial(N) =
    ( if N  = 0 then 1 else N * factorial(N - 1) ).


:- func fibonacci(int) = int.

fibonacci(N) =
    ( if N =< 2 then 1 else fibonacci(N - 1) + fibonacci(N - 2) ).
\end{verbatim}
These are examples of what is sometimes called ``middle
recursion'' where the recursive call is \emph{not} the last thing
that the predicate (or function) does when looping.  Here,
calls to @factorial/1@ finishes with a multiplication and calls
to @fibonacci/1@ finish with an addition.

Although middle recursion is easy to read, it incurrs some
performance penalty in that each iteration has to record
something extra on the stack in order to finish up the
computation.\footnote{That said, the compiler can in fact turn some
middle recursive predicates into equivalent tail recursive
predicates that do not incurr a stack overhead \XXX{}.}

\XXX{Include a side-bar or something explaining stack frames.}

\emph{Tail recursion} describes the situation where the last thing a
predicate does is call itself.  Since there is nothing left to
do after the call, the compiler can reuse the current call's
stack frame for the recursive call, allowing the predicate to
execute using only a fixed amount of stack space.  For
example, tail recursive implementations of the above
functions are
\begin{verbatim}
factorial(N) = fac(N, 1).

fac(N, X) =
    ( if N = 0 then X     else fac(N - 1, N * X) ).


fibonacci(N) = fib(1, 1, N).

fib(FN_2, FN_1, N) =
    ( if N =< 2 then FN_1 else fib(FN_1, FN_1 + FN_2, N - 1) ).
\end{verbatim}
Tail recursive code like this should execute just as quickly
and efficiently as a for or while loop in an imperative
language.

\section{Unifications}

Mercury has but two basic atomic (\ie indivisible) types of
goal: unifications and calls.

A unification is written @X = Y@.  A unification can fail if @X@
and @Y@ are not unifiable.

Two values are unifiable if they are ``structurally similar'' --
that is, where you see a data constructor in one, you either
see the same data constructor in the other (and the
corresponding arguments are also structurally similar) or a
variable, which will end up being bound to the corresponding
term on the other side if the unification is successful.\footnote{Unlike Prolog, Mercury forbids the aliasing of
variables whereby a partially instantiated data structure
may contain the same unbound variable in two different places.
This will be explained fully in a later chapter.  \XXX{}}

Unifications, therefore, can be used to bind variables to
values, test to see if a variable is bound to a particular
constructor, unpack the arguments of a constructor or all of
the above.

In the following examples we assume that variables are
initially unbound:

\begin{tabular}{rcll}
         @123@ & @=@ & @123@ &
                -- Succeeds \\
           @X@ & @=@ & @123@ &
                -- Binds @X = 123@ \\
         @123@ & @=@ & @234@ &
                -- Fails \\
           @X@ & @=@ & @foo(1, 2, 3)@ &
                -- Binds @X = foo(1, 2, 3)@ \\
@foo(X, Y, Z)@ & @=@ & @foo(1, 2, 3)@ &
                -- Binds @X = 1@, @Y = 2@, @Z = 3@ \\
@foo(X, Y, 4)@ & @=@ & @foo(1, 2, 3)@ &
                -- Fails (@4 \= 3@) \\
@foo(X, 2, 3)@ & @=@ & @foo(1, 2, Z)@ &
                -- Binds @X = 1@, @Z = 3@ \\
\end{tabular}

The complex unifications are most easily understood by first
converting them into \emph{super homogeneous normal form} (which is
what the compiler does.)  In SHNF, the left hand side of each
unification is a variable and the right hand side is either
another variable or a functor \XXX{have I explained this term?}, all of
whose arguments are variables.  For example

\begin{verbatim}
    foo(X, 2, 3) = foo(1, 2, Z)
\end{verbatim}
becomes
\begin{verbatim}
    V_1 = X, V_2 = 2, V_3 = 3, V_4 = foo(V_1, V_2, V_3),
    V_5 = 1, V_6 = 2, V_7 = Z, V_8 = foo(V_5, V_6, V_7),
    V_4 = V_8
\end{verbatim}
where @V_1@\ldots@V_8@ are all temporary variables
introduced in the conversion to SHNF.

\section{Calls}
The other kind of primitive goal supported by Mercury is the
predicate call, @p(X1, ..., Xn)@, which we have already seen in
the examples above.

One way to picture the evaluation of a call is to think of it
as expanding into the different clause bodies for the
predicate definition while looking for solutions.

Two relatively important built-in predicates are @true@ and
@false@ \XXX{}.  @true@ always succeeds and @false@ always fails.\footnote{@false@ is often written as @fail@, which is a hangover
from Mercury's Prolog roots where it was sometimes more useful
to think in procedural terms.}

\section{Conjunction}

A goal of the form @G1, G2, ..., Gn@ is called a \emph{conjunction}
with the separating commas read as ``and''.  A conjunction
succeeds iff a consistent solution to each of the sub-goals @G1@,
@G2@, \ldots, @Gn@ can be found by the program.

(The compiler may have to reorder the sequence of goals in a
clause in order to satisfy the mode constraints.  Although one
can set a flag to force the compiler to do no more reordering
than is necessary, in general this will mean that certain
optimizations will not be possible.  The upshot of this is
that one should avoid writing code that assumes a particular
evaluation order other than that dictated by the mode
constraints.)

A conjunction executes by trying each of the goals in order.
If a goal fails then the program backtracks to the nearest
preceding choice-point (\ie non-deterministic goal that may
have other solutions).

\section{Negation}

A goal of the form not @G@ succeeds iff @G@ has no solutions.  @G@
may be a compound goal, in which case it should be enclosed in
parentheses to avoid syntactic precedence problems.

The sub-goal @G@ is said to be in a \emph{negated context} and as
such cannot bind any variables visible outside the negation
(since the only way not @G@ can succeed is if @G@ fails, in which
case it will not produce any variable bindings.)

Note that not not @G@ is equivalent to @G@ and hence may bind
variables if it succeeds (while not not @G@ would be an odd
thing to write, some of the code transformations the compiler
performs can generate such things; getting the modes right for
such things requires that the compiler observe this
simplification rule.)

\section{If-Then-Else Goals}

Mercury's if-then-else construct looks like this:
\begin{verbatim}
    ( if ConditionGoal then YesGoal else NoGoal )
\end{verbatim}
Note that the else part is \emph{not} optional.  \XXX{Except in DCG
code...  But we probably don't need to mention this.}

You may also see if-then-elses written as
\begin{verbatim}
    ( ConditionGoal -> YesGoal ; NoGoal )
\end{verbatim}
although the author finds this style less appealing.

While the parentheses are not always required, it is a very
good idea to include them in order to avoid confusing
syntactic precedence errors.  One common exception is a chain
of if-then-elses where only the top level of parentheses are
necessary:
\begin{verbatim}
    ( if      ConditionGoal1 then YesGoal1
      else if ConditionGoal2 then YesGoal2
      else if ConditionGoal3 then YesGoal3
      ...
      else                        NoGoal
    )
\end{verbatim}
Extra parentheses are not required even if any of the @ConditionGoal@s,
@YesGoal@s or the @NoGoal@ are compound goals.

The if-then-else goal
\begin{verbatim}
    ( if ConditionGoal then YesGoal else NoGoal )
\end{verbatim}
is semantically equivalent to the disjunction
\begin{verbatim}
    ( ConditionGoal, YesGoal ; not ConditionGoal, NoGoal )
\end{verbatim}
but will be implemented more efficiently by the compiler (if
@ConditionGoal@ produces no solutions in the first disjunct then
there's no point in checking again that it has none in the
second disjunct.)

It's worth looking at a few examples to really understand how
if-then-else goals work.  Again, we assume that all variables
are initially unbound:
\begin{verbatim}
    ( if ( X = 1 ; X = 2 ) then ( X = 2 ; X = 4 ) else X = 5 )
\end{verbatim}
The above goal is @nondet@ (the condition is @multi@
and the then-goal is @nondet@, since @X@ will be bound at this point),
but has the single solution @X = 2@.
\begin{verbatim}
    ( if ( X = 1 ; X = 2 ) then ( X = 3 ; X = 4 ) else X = 5 )
\end{verbatim}
The above goal is also @nondet@ for the same reason, but in fact has no
solutions (the compiler can only be expected to perform a certain amount
of program analysis and will sometimes not be completely precise --
mode inference is actually undecidable in general, although this is not
a problem in practice.  \XXX{Check this is true!})
\begin{verbatim}
    ( if 1 = 2 then ( X = 3 ; X = 4 ) else ( X = 5 ; X = 6 ) )
\end{verbatim}
The above goal is @nondet@ because the condition is @semidet@ and the
then- and else-goals are
@multi@.  (Since the condition will fail, the
else-goal is evaluated, with solutions @X = 5@ and @X = 6@.)
\begin{verbatim}
    ( if 1 = 1 then ( X = 3 ; X = 4 ) else ( X = 5 ; X = 6 ) )
\end{verbatim}
Similary, the above goal is @nondet@ because the condition is @semidet@
and the then- and else-goals are
@multi@.  (Since the condition will succeed, the
else-goal is evaluated, with solutions @X = 3@ and @X = 4@.)

That said, in the vast majority of cases where the
condition-goal is semidet and the then- and else-goals are
deterministic, if-then-else goals will act in very much the
same way as similar structures in other programming languages.

Since the condition-goal is in a negated context in the else-arm
of the disjunctive form of an if-then-else, it cannot
produce any outputs that would be used in @NoGoal@ or anything
outside the if-then-else as a whole.  It can, however, produce
outputs that are only used in the @YesGoal@.  (The reason for
this restriction is slightly subtle and will be explained in
more detail later.  \XXX{It's to do with having mode
independent semantics.})

For example, the following somewhat contrived code violates
the constraint because @S@ is bound inside the condition and is
also visible outside the if-then-else goal.
\begin{verbatim}

:- pred prime_divisor(int, int).
:- mode prime_divisor(in,  out) is nondet.
...

:- pred prime_divisor_or_zero(int, int).
:- mode prime_divisor_or_zero(in,  out) is multi.

prime_divisor_or_zero(N, S) :-
    ( if   prime_divisor(N, S)
      then true
      else S = 0
    ).

\end{verbatim}
The correct way to write @prime_divisor_or_zero/2@ is
\begin{verbatim}
prime_divisor_or_zero(N, S) :-
    ( if   prime_divisor(N, D)
      then S = D
      else S = 0
    ).
\end{verbatim}
Note that if-then-else \emph{expressions} are slightly different;
the following is perfectly legal:
\begin{verbatim}
prime_divisor_or_zero(N, S) :-
    S = ( if prime_divisor(N, D) then D else 0 ).
\end{verbatim}

\section{Disjunction}

Just as conjunction lets you use ``and'' to construct goals,
disjunction lets you use ``or''.  A \emph{disjunctive} goal takes the
form @(G1 ; G2 ; ... ; Gn)@ with the separating semicolons being
read as ``or''.

A disjunction succeeds iff any of its disjunct sub-goals
succeeds.  A disjunction has as many solutions as all of its
disjuncts put together: if one disjunct fails or backtracking
exhausts all the solutions for one disjunct then execution
proceeds with another disjunct.  Again, the compiler is
generally free to reorder disjuncts, although this should not
have a visible impact on programs.  Disjunctions are typically
non-deterministic, although switches, mentioned earlier, are a
special case.

\XXX{Need examples?}

The clausal notation we have been using in the examples above
is in fact convenient syntactic sugar for writing top-level
disjunctions.  For example, the @ancestor/2@ predicate we
defined earlier
\begin{verbatim}

ancestor(Person, Ancestor) :-
    parent(Person, Ancestor).

ancestor(Person, Ancestor) :-
    parent(Person, Parent),
    ancestor(Parent, Ancestor).
\end{verbatim}
could equivalently be written as
\begin{verbatim}
ancestor(Person, Ancestor) :-
    (
        parent(Person, Ancestor)
    ;
        parent(Person, Parent),
        ancestor(Parent, Ancestor)
    ).
\end{verbatim}
(Indeed, this is how the compiler sees multi-clause
definitions.)

In general it is better style to use clausal form for
top-level disjunctions.

\section{Switches}
Mercury recognises particular forms of (@semi-@)@det@
disjunction which it can compile very efficiently.

The @string@ library module defines the following type:
\begin{verbatim}
:- type poly_type
    --->    f(float)
    ;       i(int)
    ;       s(string)
    ;       c(char).
\end{verbatim}
which can be used to form heterogeneous collections of the
primitive types (this is useful, amongst other things, for
supplying argument lists for formatted output.)

Say we wanted to write a predicate that would convert any
@poly_type@ value into a @string@.  Here's how we might write the
code (in practice we would use a function; here we use a
predicate for the purposes of illustration):
\begin{verbatim}
:- pred poly_type_to_string(poly_type, string).
:- mode poly_type_to_string(in, out) is det.

poly_type_to_string(f(F), S) :- float_to_string(F, S)
poly_type_to_string(i(I), S) :- int_to_string(I, S)
poly_type_to_string(s(S), S).
poly_type_to_string(c(C), S) :- char_to_string(C, S)
\end{verbatim}
this is equivalent to the single clause definition
\begin{verbatim}
poly_type_to_string(Poly, S) :-
    (   Poly = f(F), float_to_string(F, S)
    ;   Poly = i(I), int_to_string(I, S)
    ;   Poly = s(S)
    ;   Poly = c(C), char_to_string(C, S)
    ).
\end{verbatim}

The compiler knows that since @Poly@ is an input variable it
must be bound when the disjunction is evaluated.  The compiler
also sees that each arm of the disjunction unifies @Poly@
against a different data constructor.  The compiler therefore
deduces that at most one disjunct can succeed on a particular
call (and, since all @poly_type@ data constructors are tested
for, \emph{exactly} one must succeed.)  The compiler generates very
efficient code for so-called \emph{switch} constructs such as this.\footnote{The name \emph{switch} is used because of its similarity
to the C language construct of the same name.}

Switches are an elegant way of describing conditions based on
unification tests and are typically more efficient than the
equivalent if-then-else chains.

\section{Existential Quantification}

Sometimes we only need to know whether a solution exists, but
are not interested in the result.  For this we use existential
quantification, which looks like this:
\begin{verbatim}
    (some [X, Y, Z] G)
\end{verbatim}
A goal of this form will succeed iff there is a solution to @G@,
but any bindings for @X@, @Y@ and @Z@ will not be visible outside
the quantification -- it's rather like saying that @X@, @Y@ and @Z@ 
are local variables for the goal @G@.

Mercury has an rule that any variables in a clause that do not
also appear in the head are implicitly existentially
quantified, which means you never actually need to use
explicit existential quantification in your programs.

\section{Universal Quantification}

On the other hand, we may wish to know whether a particular
property holds for all solutions to a particular goal.  This
is where universal quantification is useful.

The goal
\begin{verbatim}
    (all [X, Y, Z] G)
\end{verbatim}
is equivalent to writing
\begin{verbatim}
    not (some [X, Y, Z] not G)
\end{verbatim}

\section{Implication}

Mercury has three types of goal for describing implicative
relationships between goals.\footnote{The translations are given by de Morgan's laws.}
\begin{itemize}
\item @(G1 => G2)@ is shorthand for @(not G1 ; G2)@;
\item @(G1 <= G2)@ is shorthand for @(G2 => G1)@; and
\item @(G1 <=> G2)@ is shorthand for @((G1 => G2), (G1 <= G2))@.
\end{itemize}

Note that parentheses are required around @G1@ and @G2@ if they
are not atomic goals; it is a good idea to also put
parentheses around the implication as a whole to avoid
ambiguity in the scope of the implication.)

Implication is most often used with universal quantification
to test for some general property.

\subsection{Examples}

This example uses the predicate @member(X, Xs)@ to non-deterministically
project members @X@ from the @list@ @Xs@ and the semidet predicate
@even(X)@ which succeeds iff @X@ is even.\footnote{The convention is to
name a @list@ of items, @X@, as @Xs@.}
\begin{verbatim}
:- pred all_even(list(int)).
:- mode all_even(in) is semidet.

all_even(Xs) :-
    all [X] ( member(X, Xs) => even(X) ).

\end{verbatim}

\begin{verbatim}

    % Two list can be interpreted as equivalent sets if
    % each contains the same members as the other.
    %
:- pred equivalent_sets(list(T), list(T)).
:- mode equivalent_sets(in, in) is semidet.

equivalent_sets(Xs, Ys) :-
    all [Z] ( member(Z, Xs) <=> member(Z, Ys) ).
\end{verbatim}
The auxiliary predicates @member/2@ and @even/1@ are defined as\footnote{The Mercury parser views anything in @`@backquotes@`@
as an infix operator.  This is the same as the rule used in Haskell.}
\begin{verbatim}
:- pred member(T, list(T)).
:- mode member(out, in) is nondet.

    % X is a member of a list if it is either the head of
    % that list or a member of the tail.
    %
member(X, [X | _ ]).
member(X, [_ | Xs]) :- member(X, Xs).


:- pred even(int).
:- mode even(in) is semidet.

even(X) :- X `mod` 2 = 0.
\end{verbatim}

\section{Higher Order Application}

\XXX{I'll talk about this later.}

\section{Anonymous and Singleton Variables}

Often one is not interested in a particular output variable
from a call or unification.  In these cases you can use the
special variable named @_@ (a single underscore) which stands
for a different anonymous or ``don't care'' variable every time
it appears.

Sometimes, however, it makes programs easier to read if you do
name don't care variables.  Since variables that only appear
once in a clause are usually the result of typographical
error, the compiler will issue a warning when it sees such
things.  To get around this problem, giving a variable a
name that starts with an underscore (\eg @_X@) tells the compiler that
you know this is a named don't care variable and that there's
no need to issue a warning.




% % vim: ft=tex ff=unix ts=4 sw=4 et wm=8 tw=0

\chapter{Functions}

Functions are @det@ (or @semidet@) relations with (at least) one output.

Essentially, a function is any relation with a single solution for
a given set of inputs.

Functions with single output values are so common that Mercury provides
special syntax to make working with them easier.  One of the key
advantages of functions is that they can be used as parts of
expressions, rather than having to have a separate goal computing each
subexpression in turn.  That is, one can use an expression as in
\begin{verbatim}
    X = (B + sqrt(B*B - 4*A*C)) / (2*A)
\end{verbatim}
rather than the verbose and somewhat opaque
\begin{verbatim}
    square(B, BSquared),
    multiply(4, A, FourA),
    multiply(FourA, C, FourAC),
    subtract(BSquared, FourAC, BSquaredMinusFourAC),
    sqrt(BSquaredMinusFourAC, Sqrt),
    add(B, Sqrt, Numerator),
    multiply(2, A, Denominator),
    divide(Numerator, Denominator, X)
\end{verbatim}

\section{Definition}

This example illustrates how functions are defined:
\begin{verbatim}
:- func length(list(T)) = int.
:- mode length(in) = out is det.

    % The length of an empty list is 0.
    % The length of a non-empty list is 1 for the head
    % plus the length of the tail.
    %
length([]      ) = 0.
length([_ | Xs]) = 1 + length(Xs).
\end{verbatim}
Like predicates, functions may be defined using several clauses
and make use of pattern matching.

Functions can be computed from goals, where the head and goal
are separated by @:-@ in the definition:
\begin{verbatim}
        % take(N, Xs) is the length min(N, length(Xs))
        % prefix of Xs.
        %
    :- func take(int, list(T)) = list(T).
    :- func take(in, in) = out is det.

    take(N, Xs) = Ys :-
        split(N, Xs, Ys, _).

        % drop(N, Xs) is the length max(0, length(Xs) - N)
        % suffix of Xs.
        %
    :- func drop(int, list(T)) = list(T).
    :- func drop(in, in) = out is det.

    drop(N, Xs) = Zs :-
        split(N, Xs, _, Zs).

        % split(N, Xs, Prefix, Suffix)
        %
    :- pred split(int, list(T), list(T), list(T)).
    :- mode split(in,  in,      out,     out    ) is det.

    split(N, Xs, Ys, Zs) :-
        ( if N > 0, Xs = [X | Xs0] then
            Ys = [X | Ys0],
            split(N - 1, Xs0, Ys0, Zs)
          else
            Ys = [],
            Zs = Xs
        ).
\end{verbatim}
By far the most common mode for a function is
\begin{verbatim}
:- mode f(in, in, ..., in) = out is det.
\end{verbatim}

If a function has (just) this sort of mode, then the mode
declaration can be ommitted and the compiler will simply assume
this mode is what is intended.  Hereafter we will omit unnecessary
mode declarations for functions.

\XXX{What exactly are the constraints on function
determinisms?  Remember to point out (somewhere) that
functions may also have multiple procedures.  See the ref.
manual section on Determinism.}

\section{Pattern Matching}

\XXX{Dealt with above.}

\section{Recursion}

\XXX{Dealt with above.}

\section{Conditional Expressions}

Conditional (if-then-else) expressions look like this:
\begin{verbatim}
    ( if ConditionGoal then YesExpr else NoExpr )
\end{verbatim}
where the usual rules for if-then-elses apply (in particular,
@ConditionGoal@ may bind output variables that are used in
YesExpr, but not elsewhere), except that the then and else
arms are \emph{expressions}, rather than goals.

Note that as with if-then-else goals, the else clause is \emph{not}
optional in a conditional expression (it would make no sense
not to have one.)

\section{* Partial (Semi-Deterministic) Functions}

\XXX{Leave for later.}

\section{Overview of Semidet Predicates}

\XXX{Deal with this in a later section.  It's sort-of advanced
stuff.}

\section{Polymorphism}

\XXX{Dealt with in section on types.}

\section{Infix Notation}

Mercury syntax supports a number of prefix, infix and postfix
operators, including all the usual arithmetic operators.  This
is just syntactic sugar, however, and there is no difference
between @X + Y@ and @+(X, Y)@ as far as the compiler is concerned.

If you want to use another name as an infix operator, you can
simply place it in @`@backquotes@`@:
\begin{verbatim}
    X `union` Y `union` Z
\end{verbatim}
is seen by the compiler as
\begin{verbatim}
    union(X, union(Y, Z))
\end{verbatim}
Backquoted symbols bind more tightly than anything else and
associate to the right.




% % vim: ft=tex ff=unix ts=4 sw=4 et wm=8 tw=0

\chapter{Input and Output}

One unfortunate consequence of being a pure declarative language
is that IO becomes somewhat more complex than is the case for
imperative languages.

\section{IO \emph{Is} a Side-Effect}

One problem is that performing IO necessarily has an effect on the
outside world that cannot be backtracked over or undone -- there is
no returning to an earlier state of the world!  This is in
contrast to the mathematically pure world that Mercury inhabits,
where there is no concept of a value (such as the state of the
world) ``changing'', only one of new such values being computed.

\section{Order is Important}

Another problem is that since logically there is no difference
between the goal @(G1, G2)@ and the goal @(G2, G1)@, we also need to
find some mechanism for ensuring that IO operations happen in the
intended order and are not mixed up as a consequence of the
compiler reordering goals for optimization purposes and so forth.

\section{Uniqueness}

A number of solutions to the IO problem have been adopted by
the pure, declarative languages, the main contenders being the
monadic approach (as exemplified by Haskell) and the
uniqueness approach (as exemplified by Clean and Mercury.)

The uniqueness approach works like this: we view a Mercury
program as a function from world states to world states.  The
top-level @main/2@ predicate of a Mercury program takes the
current world state as an input and computes a new world state
as its result.  Every IO operation does the same thing: takes
a world state as input and produces a new world state as
output -- notionally updated with the effects of the IO
operation (and the actions of the world at large between IO
operations).  The so-called IO state is opaque to the Mercury
program; it can only be queried via the operations defined in
the io module.\footnote{Of course, the Mercury program doesn't actually
pass the state of the world around in fact -- the IO state
abstraction serves both to ensure the properties we require
and to give a semantics to IO in Mercury.}

Example (we eschew state variable notation here for clarity of
exposition):
\begin{verbatim}
:- pred main(io, io).
:- mode main(di, uo) is det.

main(IO0, IO) :-
    io__write_string("pi = ", IO0, IO1),
    io__write_float(math__pi, IO1, IO2),
    io__nl(                   IO2, IO ).
\end{verbatim}
The type of the IO state is called just io and is defined
as an abstract type in the @io@ library module.  The
top-level predicate @main/2@ takes the initial IO state as a
@di@ mode argument (\emph{destructive input}), and produces another as a
@uo@ mode result (\emph{unique output}).  The three IO operations in the body
show how the initial IO state, @IO0@, is transformed into
the final IO state, @IO@.  So, @io__write_string/3@ destroys
@IO0@ and produces @IO1@ as a unique output.  Next,
@io__write_float/3@ destroys @IO1@ and produces @IO2@ as a
unique output.  Finally, @io__nl@ (which writes out a
newline) destroys @IO2@ and produces @IO@ as a unique output.

In order to ensure that old versions of the world state cannot
be accessed after an IO operation, the IO state is \emph{unique} --
this means that there can only ever be one live reference to
it.\footnote{A live reference is one that will be used
later on in computation.}  The old version of the IO state
is said to be clobbered by an IO operation -- the compiler will
report an error if the old version is still live after the IO
operation in question.  Similarly, it is impossible to make a
copy the IO state.\footnote{It is possible to ``fork'' the IO
state, this is necessary to support concurrency.  Concurrency
is dealt with in a later chapter. \XXX{}}

Example:
\begin{verbatim}
:- pred main(io, io).
:- mode main(di, uo) is det.

main(IO0, IO) :-
    io__write_string("Hello, ",  IO0, IO1),
    io__write_string("world!\n", IO0, IO ).
\end{verbatim}
Here @IO0@ is used twice; the compiler spots the bug and
rejects the program with
\begin{verbatim}
In clause for `main(di, uo)':
  in argument 2 of call to predicate `io:write_string/3':
  unique-mode error: the called procedure would clobber
  its argument, but variable `IO0' is still live.
\end{verbatim}
The uniqueness constraint is sufficient to ensure that IO
operations happen in a strict order, specified by the
programmer, and that it is impossible to backtrack over IO or
refer to a dead IO state.

Note that uniqueness is not a property reserved for IO states:
it is used to implement destructively updated arrays, the
store data type which allows the construction of arbitrary
pointer graphs, hash tables and so forth.  Uniqueness allows
the compiler to use a safe form of destructive update of data
structures: there is no reason why a dead unique object cannot
be reused to create a new live unique object (since the old
value can never be accessed), which is exactly what happens
for the data types just mentioned.

\section{Stylistic Considerations}

Since passing the IO state around everywhere is a little
cumbersome and also quite restrictive (it can only be passed
into det predicates), Mercury programmers naturally find
themselves separating applications into two parts: the part
that handles all the IO and the part that handles all the
interesting processing.  This is good style in general, and
although one might find it slightly annoying not to be able to
insert print statements willy-nilly as is the case with impure
languages, one soon finds that the discipline imposed pays
real dividends in terms of reusability, maintainability, ease
of debugging and so forth.

\section{* Determinism Restrictions}

Since IO operations cannot be backtracked across, the IO state
(and, indeed, unique objects in general) cannot be passed to
non-deterministic predicates -- that is, only deterministic
predicates can take unique IO states as arguments.\footnote{This is not strictly true; there is another
determinism, cc-multi, which is compatible with uniqueness.}

\section{* State Variable Syntax}

Having to name and pass two variables around for every IO
operation quickly becomes tiresome.  Mercury has a special
\emph{state variable} syntax for just this purpose.  The idea is to
write code that looks a little more like what one would write
in an imperative language, but which is transformed by the
compiler into pure Mercury.  A state variable argument @!X@
stands for two real arguments, @!.X@ and @!:X@, which in turn
stand for the ``current'' and ``next'' values of the state variable
@X@, respectively.  Occurrences of @!.X@ and @!:X@ are converted by
the compiler into appropriately numbered variables.

For example, the following code
\begin{verbatim}
    % Writes out a list of strings, separated by
    % commas.
    %
:- pred write_strings(list(string), io, io).
:- mode write_strings(in,           di, uo) is det.

write_strings([],            !IO).

write_strings([S1],          !IO) :-
    io__write_string(S1,     !IO).

write_strings([S1, S2 | Ss], !IO) :-
    io__write_string(S1,     !IO),
    io__write_string(", ",   !IO),
    write_strings(Ss,        !IO).
\end{verbatim}
is transformed by the compiler into\footnote{Note that the pred and mode declarations reflect
the fact that @!IO@ is actually two arguments, not one.}
\begin{verbatim}
    % Writes out a list of strings, separated by
    % commas.
    %
:- pred write_strings(list(string), io, io).
:- mode write_strings(in,           di, uo) is det.

write_strings([],            IO0, IO) :-
    IO = IO0.

write_strings([S1],          IO0, IO) :-
    io__write_string(S1,     IO0, IO).

write_strings([S1, S2 | Ss], IO0, IO) :-
    io__write_string(S1,     IO0, IO1),
    io__write_string(", ",   IO1, IO2),
    write_strings(Ss,        IO2, IO ).
\end{verbatim}
Henceforth we will use state variable syntax rather than
explicitly numbered IO states.

\section{Common IO Operations}

The @io@ library module defines a plethora of useful IO
operations and as usual with libraries, the reader is
encouraged to take some time to peruse the interface section.
Here we will present some basic IO operations to help get the
ball rolling.

\subsection{Output}

Output is generally simpler to deal with than input,
because, by and large, there are no error codes to deal
with.

The @io@ library module includes predicates for the output
of the basic types:
\begin{verbatim}
:- pred io__write_string(string, io, io).
:- mode io__write_string(in,     di, uo) is det.

:- pred io__write_char(char, io, io).
:- mode io__write_char(in,   di, uo) is det.

:- pred io__write_int(int, io, io).
:- mode io__write_int(in,  di, uo) is det.

:- pred io__write_float(float, io, io).
:- mode io__write_float(in,    di, uo) is det.
\end{verbatim}
However, it's typically easier to use the more general
formatted output predicate:
\begin{verbatim}
:- pred io__format(string, list(poly_type), io, io).
:- mode io__format(in,     in,              di, uo) is det.
\end{verbatim}
The @string@ argument describes how the output is to be formatted, very
similar in spirit and detail to what one would pass to C's @printf()@.
The @list@ argument is a type safe means of passing the parameters to be
formatted.

Using @io__format/4@ one might write
\begin{verbatim}
:- pred write_record(string, int, float, io, io).
:- mode write_record(in,     in,  in,    di, uo) is det.

write_record(Name, Age, Children, !IO) :-
    io__format("%s is %d years old and has %f children.\n",
               [s(Name), i(Age), f(Children)], !IO).
\end{verbatim}
and then we could call
\begin{verbatim}
    write_record("Joe Bloggs", 43, 2.4, !IO)
\end{verbatim}
and the program would write out
\begin{verbatim}
Joe Bloggs is 43 years old and has 2.4 children.
\end{verbatim}
The implementation of @write_record/5@ is much simpler than the
functionally equivalent
\begin{verbatim}
:- pred write_record(string, int, float, io, io).
:- mode write_record(in,     in,  in,    di, uo) is det.

write_record(Name, Age, Children,           !IO) :-
    io__write_string(Name,                  !IO),
    io__write_string(" is ",                !IO),
    io__write_int(Age,                      !IO),
    io__write_string(" years old and has ", !IO),
    io__write_float(Children,               !IO),
    io__write_string(" children.\n",        !IO).
\end{verbatim}

In fact, @io__format/4@ is quite a bit more powerful, in
that the @%@ formatting specifications can include
details as to the style of formatting, precision,
justification and so forth.

\XXX{I'm not sure I should mention @io\_\_print/3@ this early.}
Another useful predicate the @io@ library module provides is
\begin{verbatim}
:- pred io__print(T,  io, io).
:- mode io__print(in, di, uo) is det.
\end{verbatim}
@io__print/3@ is used to print a representation of arbitrary
Mercury values.  Be aware, though, that if you try to
print the results of expressions, the compiler may ask you
to supply more type information to help resolve what
exactly it is you are printing (\ie the expression or the
value of the expression.)  This is subtle stuff and will
be dealt with in a later chapter. \XXX{}

\subsection{Input}

Input is marginally more complex than output since
typically on of three things can happen:
\begin{enumerate}
\item we successfully read a value of the required type from
  the input stream;
\item we hit the end-of-file;
\item an error of some kind occurs (\eg the input is
  malformed or the input source has gone away unexpectedly.)
\end{enumerate}

To this end the @io@ library module defines the following
type to report the results of input operations:
\begin{verbatim}
:- type io__result(T) ---> ok(T)
                      ;    eof
                      ;    error(io__error).
\end{verbatim}
In order, @ok(X)@ is returned if the input operation
succeeded, reading @X@; @eof@ is returned if the end of file
has been reached; and @error(ErrorCode)@ is returned if
something went wrong (the function @io__error_message/1@ can
be used to turn @ErrorCode@ into a printable error message.)

This arrangement forces the programmer to handle the error
cases.  There is still lively debate over whether error
codes or throwing exceptions is the best way to handle
errors for things like this.  A genuine advantage of the
error code approach is that you have to consider the error
cases from the outset, which, while requiring a little
more initial thought from the programmer, usually pays
real dividends later on.
\XXX{is this the right place to say this?  Should I
enlarge on the debate?  Probably no and yes
respectively...}

Two very basic input predicates are
\begin{verbatim}
:- pred io__read_char(io__result(char), io, io).
:- mode io__read_char(in,               di, uo) is det.

:- pred io__read_line_as_string(io__result(string), io, io).
:- mode io__read_line_as_string(in,                 di, uo)
            is det.
\end{verbatim}

The input predicates are also less comprehensive in the
sense that there are no predicates @io__read_int/3@,
@io__read_float/3@ or @io__read_string/3@.  The problem is
that it's not clear exactly what should be allowed to
terminate the input stream for an @int@, @float@ or @string@.
Instead, the library leaves issues of parsing up to
applications (there are several programs in the Mercury
extras distribution to help, including a lexer and parser
generators.)

Here's a simple program echoes the capitalised version of
letters in the input stream:
\begin{verbatim}
:- module capitalise.
:- interface.
:- import_module io.

:- pred main(io, io).
:- mode main(di, uo) is det.

:- implementation.
:- import_module char, exception.

main(!IO) :-
    io__read_char(R, !IO),
    (
        R = ok(C),
        io__write_char(char__to_upper(C), !IO),
        main(!IO)
    ;
        R = eof
    ;
        R = error(E),
        exception__throw(E)
    ).
\end{verbatim}




% % vim: ft=tex ff=unix ts=4 sw=4 et tw=76

\chapter{Modes}

REORGANISE CHAPTER AS FOLLOWS:
\begin{itemize}
\item Split into two chapters.
\emph{Basic Modes.}
\begin{itemize}
\item in, out
\item determinisms
\item di, uo
\end{itemize}
\emph{Advanced Modes.}
\begin{itemize}
\item modes and insts
\item mode definitions
\item inst definitions
\item SHNF
\item reordering
\item determinism inference
\item committed choice
\item mostly uniqueness
\item mode-specific clauses
\end{itemize}
\item \XXX{Move higher order insts to HO chapter.}
\item \XXX{In modules chapter, mention that there are no abstract insts or
modes.}
\end{itemize}

\XXX{Add a light-and-fluffy intro.  Explain that what's important is
in/out/di/uo and a basic grasp of determinism categories.  Thereafter,
one can return to this chapter on an as-needs basis.}

Mercury's mode system performs several jobs: it tells us whether a given
predicate argument is an input or output, whether it is unique on input
or output, and how many solutions a particular predicate call can have.

\section{Predicates and Procedures}

Consider the following predicate that computes the concatenation of two
lists:
\begin{myverbatim}
:- pred append(list(T), list(T), list(T)).
:- mode append(in,      in,      out    ) is det.
:- mode append(in,      out,     in     ) is semidet.
:- mode append(out,     out,     in     ) is multi.

append([],       Bs, Bs      ).
append([A | As], Bs, [A | Cs]) :- append(As, Bs, Cs).
\end{myverbatim}
The definition reads as follows: ``appending the empty list and the list
@Bs@ is just the list @Bs@; appending the non-empty list @[A | As]@ and
the list @Bs@ is the list whose head is @A@ and whose tail @Cs@ is
formed by appending @As@ and @Bs@.

The following are all solutions of @append/3@:
\begin{myverbatim}
    append([], [1, 2, 3], [1, 2, 3])
    append([1],   [2, 3], [1, 2, 3])
    append([1, 2],   [3], [1, 2, 3])
    append([1, 2, 3], [], [1, 2, 3])
\end{myverbatim}
but these are \emph{not}:
\begin{myverbatim}
    append([],     [], [1, 2, 3])
    append([1],   [3], [1, 2, 3])
    append([], [2, 3], [1, 2, 3])
\end{myverbatim}

Four mode declarations are given for @append/3@, each one describing a
different \emph{procedure}.  A procedure is a particular combination
of inputs and outputs for a given predicate.

It's easiest to understand the different procedures of @append/3@ if we
first convert its definition into \emph{super-homogeneous normal form}.
SHNF expands out all syntactic sugar and introduces new variables and
unifications so that each atomic goal is one of the following:
\begin{description}
\item [a predicate call,] @p(X, ...)@, in which each argument is a
variable;
\item [a construction or deconstruction,] @X = a(Y, ...)@, in which @a@
is a data constructor, each of whose arguments is a variable;
\item [an equality test or assignment,] @X = Y@, between two variables.
\end{description}
In a construction, @X@ is an output and @Y, ...@ are inputs.  In a
deconstruction, @X@ is an input and @Y, ...@ are outputs.  In an
equality test, both @X@ and @Y@ are inputs, while in an assignment
either @X@ is an input and @Y@ is an output or vice versa.  Assignments
and constructions always succeed, whereas deconstructions and equality
tests may fail.

The SHNF for @append/3@ is
\begin{myverbatim}
append(H1, H2, H3) :-
    (
        H1 = [],
        H2 = Bs,
        H3 = Bs
    ;
        H1 = [X1 | X2],  X1 = A,  X2 = As,
        H2 = Bs,
        H3 = [Y1 | Y2],  Y1 = A,  Y2 = Cs,
        append(Z1, Z2, Z3),  Z1 = As,  Z2 = Bs,  Z3 = Cs
    ).
\end{myverbatim}
(Although this \emph{looks} more complicated than the original
definition, SHNF means we only have to consider very simple kinds of
goal in each instance.  Any unnecessary unifications and so forth will
be optimized away by the compiler before generating an executable.)

The cardinal rule is that a goal requiring a particular variable as an
input can only be executed \emph{after} that variable has been
\emph{bound}, either because it is an input head variable, or it
previously appeared as an output argument of a predicate call, or it was
previously bound in a unification.

In order to satisfy the rule, the Mercury compiler has to reorder the
goals in conjunctions separately for each procedure.  (In other words,
the same predicate definition will be compiled for each
procedure.)
Reordering is sound because
Mercury is a logic programming language and, declaratively speaking, it
doesn't matter if we write ``@P@ and @Q@'' or ``@Q@ and @P@'' -- they
both mean the same thing.  The order matters when we try to execute
a program, but it doesn't affect the \emph{semantics} (\ie the meaning)
of the program.

Let's see how mode reordering works for each mode of @append/3@.

\subsection{The First Mode}

The first mode of @append/3@ is
\begin{myverbatim}
:- mode append(in, in, out) is det.
\end{myverbatim}
which tells us that calls to this procedure have head variables @H1@ and
@H2@ as inputs (\ie they start off already bound to something) while
@H3@ is an output (\ie it starts off \emph{free}, but will be bound to
something by the time the procedure finishes.)

A goal @append(As, Bs, Cs)@ in this mode reads ``given @As@ and @Bs@,
compute the @Cs@ that results from appending @As@ and @Bs@''.  Thus the
goal
\begin{myverbatim}
    append([1], [2, 3], Cs)
\end{myverbatim}
will compute the solution @Cs = [1, 2, 3]@.

Since disjuncts have no effect on one another, we can consider each
disjunct separately.

The first disjunct is
\begin{myverbatim}
    H1 = [],
    H2 = Bs,
    H3 = Bs
\end{myverbatim}
Since @H1@ is an input and @[]/0@ is a functor, the first goal @H1 = []@
must be a deconstruction and can be \emph{scheduled} immediately.

In the second goal, @H2 = Bs@, @H2@ is an input, but @Bs@ has not been
bound to anything (all local variables start off as free), hence this is
an assignment and afterwards @Bs@ will also be bound.  We can schedule
this goal straight away, too.

In the third goal, @Bs@ is now an input while @H3@ is not bound,
therefore @H3 = Bs@ is an assignment to @H3@.

So in this procedure no reordering is required for the first disjunct
and @H3@ finishes up bound as required.

The second disjunct is
\begin{myverbatim}
    H1 = [X1 | X2],  X1 = A,  X2 = As,
    H2 = Bs,
    H3 = [Y1 | Y2],  Y1 = A,  Y2 = Cs
    append(Z1, Z2, Z3),  Z1 = As,  Z2 = Bs,  Z3 = Cs
\end{myverbatim}
Since @H1@ is input, but @X1@ and @X2@ are not bound, @H1 = [X1 | X2]@
is a deconstruction.  Then @X1 = A@ and @X2 = As@ are assignments to @A@
and @As@ respectively.

Similarly, @H2@ is an input and @Bs@ is not bound, so @H2 = Bs@ is an
assignment to @Bs@

The third line needs some reordering.  The goal @H3 = [Y1 | Y2]@
contains no bound variables at this point, so it will have to be
scheduled later.  The unification @Y1 = A@ is an assignment to @Y1@,
since @A@ is now bound, and can be scheduled now.  The unification
@Y2 = Cs@, on the other hand, also only contains free variables and has
to be scheduled later.

The fourth line also needs reordering.  Neither @Z1@, @Z2@ nor @Z3@ are
bound at this point, so the call to @append/3@ cannot be scheduled here.
However, if we first schedule the assignments @Z1 = As@ and
@Z2 = Bs@ then we \emph{can} subsequently call @append(Z1, Z2, Z3)@
via
\begin{myverbatim}
:- mode append(in, in, out) is det.
\end{myverbatim}
(because @Z1@ and @Z2@ are now bound) and, as @Z3@ will be bound after
the call, follow this up with the assignment @Z3 = Cs@.

Returning to our two unscheduled goals, since @Cs@ is
now bound, we can schedule the assignment @Y2 = Cs@ and then the
construction @H3 = [Y1 | Y2]@.

The mode reordered version of the second disjunct is therefore
\begin{myverbatim}
    H1 = [X1 | X2],  X1 = A,  X2 = As,
    H2 = Bs,
    Z1 = As,  Z2 = Bs,  append(Z1, Z2, Z3),  Z3 = Cs
    Y1 = A,   Y2 = Cs,  H3 = [Y1 | Y2]
\end{myverbatim}
again leaving @H3@ bound as an output as required (we've moved the goal
@Y1 = A@ a little later than its earliest possible scheduling in order to
preserve as much of the structure of the original definition as
possible; the compiler almost certainly wouldn't bother with such
niceties!)

Putting our two disjuncts together, we get
\begin{myverbatim}
:- mode append(in, in, out) is det.

append(H1, H2, H3) :-
    (
        H1 = [],
        H2 = Bs,
        H3 = Bs
    ;
        H1 = [X1 | X2],  X1 = A,  X2 = As,
        H2 = Bs,
        Z1 = As,  Z2 = Bs,  append(Z1, Z2, Z3),  Z3 = Cs
        Y1 = A,   Y2 = Cs,  H3 = [Y1 | Y2]
    ).
\end{myverbatim}
All that remains is to verify that the @is det@ determinism declaration
is satisfied.  The @pred@ declaration tells us that @H1@ has type
@list(T)@, and so must be bound to either @[]@ or @[X1 | X2]@ for some
@X1@ and @X2@.  The top-level goal is a disjunction and for each of the
two possible bindings of @H1@ there is a disjunct starting with the
corresponding deconstruction.  The disjunction is therefore an
\emph{exhaustive switch} and, since the other goals in each disjunct are
all deterministic, the switch as a whole must be deterministic.  Hence
the @is det@ determinism declaration is satisfied.

(You start to see why we get the compiler to do all this hard work for
us\ldots)

% \subsection{The Second Mode}
% 
% OMITTED BECAUSE THE MODE IS INFERRED AS NONDET, BUT IN PRACTICE IS
% SEMIDET
% 
% The second mode is
% \begin{myverbatim}
% :- mode append(out, in, in) is nondet.
% \end{myverbatim}
% A goal @append(As, Bs, Cs)@ in this mode reads ``given @Bs@ and @Cs@,
% what @As@ (if any) can be appended with @Bs@ to get @Cs@?''  Hence
% \begin{myverbatim}
%     append(As, [3], [1, 2, 3])
% \end{myverbatim}
% will compute the solution @As = [1, 2]@ whereas
% \begin{myverbatim}
%     append(As, [1], [1, 2, 3])
% \end{myverbatim}
% will just fail.
% 
% To speed up the process of explanation we'll just present the correctly
% reordered disjuncts with comments to explain as we go along.
% 
% The first disjunct
% \begin{myverbatim}
%     H1 = [],
%     H2 = Bs,
%     H3 = Bs
% \end{myverbatim}
% is ordered as follows:
% \begin{myverbatim}
%                         % Start,          binds H2, H3
%     H1 = [],            % Construction,   binds H1
%     H2 = Bs,            % Assignment,     binds Bs
%     H3 = Bs             % Equality test
% \end{myverbatim}
% 
% The second disjunct
% \begin{myverbatim}
%     H1 = [X1 | X2],  X1 = A,  X2 = As,
%     H2 = Bs,
%     H3 = [Y1 | Y2],  Y1 = A,  Y2 = Cs
%     append(Z1, Z2, Z3),  Z1 = As,  Z2 = Bs,  Z3 = Cs
% \end{myverbatim}
% is reordered thus:
% \begin{myverbatim}
%                         % Start,          binds H2, H3
%     H2 = Bs,            % Assignment,     binds Bs
%     H3 = [Y1 | Y2],     % Deconstruction, binds Y1, Y2
%     Y1 = A,             % Assignment,     binds A
%     Y2 = Cs,            % Assignment,     binds Cs
%     Z2 = Bs,            % Assignment,     binds Z2
%     Z3 = Cs,            % Assignment,     binds Z3
%     append(Z1, Z2, Z3), % Call append(out, in, in) is semidet
%                         %                 binds Z1
%     Z1 = As,            % Assignment,     binds As
%     X1 = A,             % Assignment,     binds X1
%     X2 = As,            % Assignment,     binds X2
%     H1 = [X1 | X2]      % Construction,   binds H1
% \end{myverbatim}
% 
% The @is nondet@ determinism is justified because...

\subsection{The Second Mode}

\begin{myverbatim}
:- mode append(in, out, in) is semidet.
\end{myverbatim}
A goal @append(As, Bs, Cs)@ in this mode reads ``given @As@ and @Cs@,
what @Bs@ (if any) can @As@ be appended with to get @Cs@?''  The goal
\begin{myverbatim}
    append([1], Bs, [1, 2, 3])
\end{myverbatim}
will compute the solution @Bs = [2, 3]@, but
\begin{myverbatim}
    append([2], Bs, [1, 2, 3])
\end{myverbatim}
will fail.

To speed up the process of explanation we'll just present the correctly
reordered disjuncts with comments to explain as we go along.  At each
point we try to schedule the ``earliest'' goal appearing in the original
definition that can be scheduled.

The first disjunct
\begin{myverbatim}
    H1 = [],
    H2 = Bs,
    H3 = Bs
\end{myverbatim}
is ordered as follows:
\begin{myverbatim}
                        % Start,          binds H1, H3
    H1 = [],            % Deconstruction
    H3 = Bs             % Assignment,     binds Bs
    H2 = Bs,            % Assignment,     binds H2
\end{myverbatim}

The second disjunct
\begin{myverbatim}
    H1 = [X1 | X2],  X1 = A,  X2 = As,
    H2 = Bs,
    H3 = [Y1 | Y2],  Y1 = A,  Y2 = Cs
    append(Z1, Z2, Z3),  Z1 = As,  Z2 = Bs,  Z3 = Cs
\end{myverbatim}
is reordered thus:
\begin{myverbatim}
                        % Start,          binds H1, H3
    H1 = [X1 | X2],     % Deconstruction, binds X1, X2
    X1 = A,             % Assignment,     binds A
    X2 = As,            % Assignment,     binds As
    H3 = [Y1 | Y2],     % Deconstruction, binds Y1, Y2
    Y1 = A,             % Equality test
    Y2 = Cs             % Assignment,     binds Cs
    Z1 = As,            % Assignment,     binds Z1
    Z3 = Cs             % Assignment,     binds Z3
    append(Z1, Z2, Z3), % Call append(in, out, in) is semidet
                        %                 binds Z2
    Z2 = Bs,            % Assignment,     binds Bs
    H2 = Bs             % Assignment,     binds H2
\end{myverbatim}

The @is semidet@ determinism declaration is justified by the fact that
while the disjunction is a switch on @H1@, as before, the second disjunct
can \emph{fail} because of the deconstruction of @H3@, the equality test
@Y1 = A@ or the recursive call to the @semidet@ mode of @append/3@.

\subsection{The Third Mode}

\begin{myverbatim}
:- mode append(out, out, in) is multi.
\end{myverbatim}
A goal @append(As, Bs, Cs)@ in this mode reads ``given @Cs@, what @As@
and @Bs@ can be appended to get @Cs@?''  The goal
\begin{myverbatim}
    append(As, Bs, [1, 2, 3])
\end{myverbatim}
has the following possible solutions:
\begin{myverbatim}
    As = [],        Bs = [1, 2, 3]
    As = [1],       Bs =    [2, 3]
    As = [1, 2],    Bs =       [3]
    AS = [1, 2, 3], Bs =        []
\end{myverbatim}
each of which will be computed by Mercury on backtracking.
Every list can be decomposed in at least one way, so this predicate
has determinism @multi@, meaning any given call will have at least one
solution and possibly more.

The first disjunct
\begin{myverbatim}
    H1 = [],
    H2 = Bs,
    H3 = Bs
\end{myverbatim}
is ordered like this:
\begin{myverbatim}
                        % Start,          binds H3
    H1 = [],            % Assignment,     binds H1
    H3 = Bs,            % Assignment,     binds Bs
    H2 = Bs             % Assignment,     binds H2
\end{myverbatim}

The second disjunct
\begin{myverbatim}
    H1 = [X1 | X2],  X1 = A,  X2 = As,
    H2 = Bs,
    H3 = [Y1 | Y2],  Y1 = A,  Y2 = Cs
    append(Z1, Z2, Z3),  Z1 = As,  Z2 = Bs,  Z3 = Cs
\end{myverbatim}
is reordered this way:
\begin{myverbatim}
                        % Start,          binds H3
    H3 = [Y1 | Y2],     % Deconstruction, binds Y1, Y2
    Y1 = A,             % Assignment,     binds A
    Y2 = Cs,            % Assignment,     binds Cs
    Z3 = Cs,            % Assignment,     binds Z3
    append(Z1, Z2, Z3), % Call append(out, out, in) is multi
                        %                 binds Z1, Z2
    Z1 = As,            % Assignment,     binds As
    Z2 = Bs,            % Assignment,     binds Bs
    X1 = A,             % Assignment,     binds X1
    X2 = As,            % Assignment,     binds X2
    H1 = [X1 | X2],     % Construction,   binds H1
    H2 = Bs             % Assignment,     binds H2
\end{myverbatim}

Every call to this mode of @append/3@ creates a \emph{choice
point} -- a place where different disjuncts may each lead to a solution.
The first disjunct always succeeds (it consists entirely of
assignments.)  The second, however, can fail if the deconstruction of
@H3@ fails, but otherwise may have many solutions thanks to the
recursive call to the @is multi@ mode of @append/3@.

\subsection{Nondeterminism}

Since this is such an important point, let's spend a little time seeing
how nondeterminism works in practice.  Now that we understand mode
reordering, we can dispense with unfriendly SHNF and work with the
altogether fluffier original definition.
\begin{myverbatim}
:- mode append(out, out, in) is multi.

append([],       Bs, Bs      ).
append([A | As], Bs, [A | Cs]) :- append(As, Bs, Cs).
\end{myverbatim}
Consider the goal
\begin{myverbatim}
    append(As, Bs, [1, 2, 3])
\end{myverbatim}
For each call to @append/3@, Mercury creates a choice point for the
disjunction and tries one of the disjuncts.  Failure later on will cause
Mercury to \emph{backtrack} to the most recent choice point and try the
other disjunct.

The possible paths to a solution are illustrated in figure
\XXX{append:fig}
(local variables have been renamed @As1@, @As2@, @As3@ and so on to
avoid confusion).
Each of the coloured arrows on the left hand side leads from an
@append/3@ choice point to the result of taking one of the two
clauses.
The arrows on the right hand side show the data flow
from the third argument when taking the second clause.
\begin{figure}[ht]
\flushleft{\epsfbox{append.eps}}
\caption{Choice points and nondeterminism.}[append:fig]
\end{figure}

When execution chooses the first clause for @append(As, Bs, Cs)@
we get @As = [], Bs = Cs@ as the solution.  When
execution chooses the second clause we get the deconstruction
@Cs = [A | Cs1]@, followed by the call @append(As1, Bs, Cs1)@ and the
construction @As = [A | As1]@.

Following the ``choice point'' arrows in the figure backwards, we see for
instance that the second-to-last solution generated is
\begin{myverbatim}
    Bs  = [3],
    As2 = [],
    As1 = [2 | As2],
    As  = [1 | As1]
\end{myverbatim}
which is the same as saying @As = [1, 2], Bs = [3]@.

\subsubsection{An Example of Use}

\XXX{I think this may be rather too complicated.  Any suggestions?}

So how is nondeterminism useful?  Nondeterminism is most commonly used
to succinctly describe searching problems.  Say we are asked to write a
predicate that takes a list of integers and succeeds iff the list is in
ascending order.  We could take the laborious route of writing
\begin{myverbatim}
:- pred in_ascending_order(list(int)).
:- mode in_ascending_order(in       ) is semidet.

in_ascending_order([]).
in_ascending_order([_]).
in_ascending_order([A, B | List]) :-
    A < B,
    in_ascending_order([B | List]).
\end{myverbatim}
But this involves rather too much ``how'' compared to ``what'' and
requires some analysis on the part of the reader to understand the
definition.  An alternative approach is to rephrase the specification to
say a list is in ascending order \emph{if} it contains no consecutive
members @A@ and @B@ such that @not A < B@.  We can then state the
solution directly:
\begin{myverbatim}
in_ascending_order(List) :-
    not (has_consecutive_members(List, A, B), not A < B).

:- pred has_consecutive_members(list(T), T,   T  ).
:- mode has_consecutive_members(in,      out, out) is nondet.

has_consecutive_members(List, A, B) :-
    append(Prefix, Suffix, List),
    Suffix = [A, B | _].
\end{myverbatim}
(This version is admittedly no shorter; we'll remedy this shortcoming in
a little while.)

The definition of @has_consecutive_members/3@ says that ``@List@
contains the consecutive members @A@ and @B@ \emph{if} it can be split
into @Prefix@ and @Suffix@ where @Suffix@ is a list with at least two
members, starting with @A@ and @B@.''

@has_consecutive_members/3@ works by exploiting the
@append(out, out, in)@ @is multi@ procedure.  For example, consider the
goal
\begin{myverbatim}
    has_consecutive_members([1, 2, 3, 4], A, B).
\end{myverbatim}
The possible solutions for the body of @has_consecutive_members/3@
are
\begin{myverbatim}
    Prefix = [],     Suffix = [1, 2, 3, 4], A = 1, B = 2
    Prefix = [1],    Suffix =    [2, 3, 4], A = 2, B = 3
    Prefix = [1, 2], Suffix =       [3, 4], A = 3, B = 4
\end{myverbatim}
(The solutions to the @append/3@ subgoal giving @Suffix = [4]@ and
@Suffix = []@ cause the deconstruction
@Suffix = [A, B | _]@ to fail.)

Let's return to the definition of @in_ascending_order/1@.  Since a goal
of the form @not P@ means ``@P@ has no solutions'', Mercury must try all
possible ways of finding a solution for @P@ before deciding whether
@not P@ succeeds or fails.

We'll use a couple of examples to illustrate the principle.
From the definition, the goal @in_ascending_order([1, 2, 3, 4])@ is
equivalent to
\begin{myverbatim}
    not (
        has_consecutive_members([1, 2, 3, 4], A, B),
        not A < B
    )
\end{myverbatim}
As we've just seen, the solutions for the @has_consecutive_members/3@
subgoal are
\begin{myverbatim}
    A = 1, B = 2
    A = 2, B = 3
    A = 3, B = 4
\end{myverbatim}
Since @not A < B@ is false in each case, the conjunction as a whole has
no solution and is
also false.  Therefore the \emph{negated conjunction} is \emph{true}
and @[1, 2, 3, 4]@ is in ascending order.

What about @in_ascending_order([1, 2, 4, 3])@?  This is equivalent to
\begin{myverbatim}
    not (
        has_consecutive_members([1, 2, 4, 3], A, B),
        not A < B
    )
\end{myverbatim}
The possible solutions for the @has_consecutive_members/3@ subgoal are
\begin{myverbatim}
    A = 1, B = 2
    A = 2, B = 4
    A = 4, B = 3
\end{myverbatim}
@not A < B@ is false for the first two subgoal solutions, but
\emph{true} for the third.  Since the conjunction as a whole has a
solution, its negation is \emph{false}.  Hence @[1, 2, 4, 3]@ is not in
ascending order.

(Note that for a goal @P, Q@, a Mercury program does not compute all
the solutions to @P@ \emph{before} considering @Q@.  Rather, it tests
@Q@ as soon as @P@ succeeds, only backtracking into
@P@ for a different solution if @Q@ fails.  \XXX{I don't really want to
get into a discussion of termination semantics etc. here.})

\subsubsection{Tidying Up}

The double negation in the definition of @in_ascending_order/1@ is
unfortunate.  Another way of stating the predicate specification is that
a list is in ascending order if every consecutive pair of members @A@
and @B@ satisfies @A < B@.  Mercury has syntactic sugar to allow us to
write this directly:
\begin{myverbatim}
in_ascending_order(List) :-
    all [A, B] (has_consecutive_members(List, A, B) => A < B).
\end{myverbatim}
This reads as ``@List@ is in ascending order \emph{if} for all @A, B@,
if @A, B@ are consecutive members of @List@ then @A < B@.''

The goal @all [X, Y, Z] P@ is shorthand for
@not some [X, Y, Z] (not P)@ -- that is, ``@P@ is true for all
@X, Y, Z@'' is the same as saying ``there is no @X, Y, Z@ for which @P@
is not true.''

The goal @some [X, Y, Z] Q@ can be shortened to just @Q@ since Mercury
assumes that any variables that appear just in @Q@ are existentially
quantified.  \XXX{Remember to mention quantification in some earlier
chapter.}

The goal @R => S@ is shorthand for @not (R, not S)@ -- that is,
``if @R@ is true then @S@ is true'' is the same as saying ``it is not
the case that @R@ can be true and @S@ false''.

Combining the above we get
\begin{myverbatim}
in_ascending_order(List) :-
    not not not (
        has_consecutive_members(List, A, B),
        not A < B
    ).
\end{myverbatim}
Finally, a goal @not not P@ is equivalent to just @P@ -- ``it's not the
case that P is not true'' is just the same as saying ``P is true'' -- so
we end up with our original definition,
\begin{myverbatim}
in_ascending_order(List) :-
    not (
        has_consecutive_members(List, A, B),
        not A < B
    ).
\end{myverbatim}

Next, we turn our attention to @has_consecutive_members/3@.  The
original definition was
\begin{myverbatim}
has_consecutive_members(List, A, B) :-
    append(Prefix, Suffix, List),
    Suffix = [A, B | _].
\end{myverbatim}
We can shorten this two ways.  First, because @Prefix@ appears only once,
we can replace it with the ``don't care'' variable, @_@.  Second, @=/2@ in
mercury really does denote equality which means, this being a declarative
language where we can always replace equals with equals, we can ``push''
the unification up into the call to @append/3@, giving
\begin{myverbatim}
has_consecutive_members(List, A, B) :-
    append(_, [A, B | _], List),
\end{myverbatim}

\section{Modes and Insts}

\XXX{I'm not going to mention uniqueness until a later section.}

So far we have only looked at the @in@ and @out@ modes without
really saying what they mean.  We've also been somewhat cavalier in our
use of terminology, using ``mode'' to refer to both possible modes of a
predicate (its procedures) and to refer to whether an argument in an
input or an output.  In this section, we use mode in the latter sense.

For the most part, @in@ and @out@ are all that are necessary.  However,
any program that performs IO, employs higher order programming techniques,
or uses subtyping will also need some more specialised modes.

\subsection{Modes}

A \emph{mode} describes the transition between instantiation states, or
\emph{insts}, for a variable when a unification or procedure call 
executes successfully.  (A goal that fails has no effect on anything.)
An \emph{inst} describes the possible bindings for a variable
at a particular instant in time.

The two most basic insts are @free@ and @ground@.  A free variable has
not yet been bound to anything whereas a ground variable \emph{is}
bound to something, but we don't know what, exactly.

The built-in modes @in@ and @out@ are therefore defined like this:
\begin{myverbatim}
:- mode out == (free   >> ground).
:- mode in  == (ground >> ground).
\end{myverbatim}
These are mode \emph{definitions} (as opposed to mode \emph{declarations}
which are associated with procedures.)  The left and right hand sides of
the @>>@ symbol are the ``before'' and ``after'' insts, respectively.

The mode definition for @out@ says that an @out@ mode argument must be
free before the goal, but will be ground afterwards (\ie it will
become bound \emph{if the goal succeeds.})  The definition for @in@ says
that an @in@ mode argument must be ground before the goal is executed
and will still be ground afterwards.

\subsubsection{Parametric Modes}

Just as types can have parameters, modes can be generalised over insts.
For instance, the built-in parametric modes @in/1@ and @out/1@ are defined as
\begin{myverbatim}
:- mode out(I) == (free >> I).
:- mode in(I)  == (I    >> I).
\end{myverbatim}
so we could have defined @in/0@ and @out/0@ with
\begin{myverbatim}
:- mode out == out(ground).
:- mode in  == in(ground).
\end{myverbatim}

\subsection{Refined Insts}

One can define more refined insts than @ground@.  To illustrate,
consider the following predicate:
\begin{myverbatim}
:- pred head_tail(list(T), T,   list(T)).
:- mode head_tail(in,      out, out    ) is semidet.

head_tail([X | Xs], X, Xs).
\end{myverbatim}
The goal @head_tail([1, 2, 3], Head, Tail)@ has the solution
@Head = 1,@ @Tail = [2, 3]@, whereas the goal
@head_tail([], Head, Tail)@ will fail.

Now, say that at a particular point in our program we have the goal
@head_tail(List, Head, Tail)@ and we \emph{know} that @List@ will not be
empty.  It would be horrible to have to write
\begin{myverbatim}
    ( if head_tail(List, Head, Tail) then
        ...rest of the program...
      else
        exception.throw("this wasn't supposed to happen!")
    )
\end{myverbatim}
to make this particular goal deterministic rather than
semideterministic.

A more elegant solution is to define an inst for non-empty lists:
\begin{myverbatim}
:- inst non_empty_list ---> [ground | ground].
\end{myverbatim}
This says that a variable with inst @non_empty_list@ will be bound to
a @[|]/2@ functor, both of whose arguments will be ground.

Next, we specify another procedure for @head_tail/3@:
\begin{myverbatim}
:- mode head_tail(in(non_empty_list), out, out) is det.
\end{myverbatim}

The new mode declaration for @head_tail/3@ states
that if the first argument is a non-empty list input, and the other two
arguments are outputs, then @head_tail/3@ is guaranteed to succeed.  The
compiler can verify this because it knows that if the first argument
(call it @List@) to @head_tail/3@ is non-empty then the deconstruction
@List = [X | Xs]@ \emph{must} succeed.

Now the sordid conditional goal wrapper around our
call to @head_tail/3@ is no longer necessary.  Which is super.

\subsection{Parametric Insts}

We can also have parametric insts.  Consider the following:
\begin{myverbatim}
:- inst list(I) ---> [] ; [I | list(I)].
\end{myverbatim}
This says that a variable with inst @list(I)@ (for some inst @I@) will
be bound to either @[]@ or to a @[|]/2@ functor whose first argument has
inst @I@ and whose second argument has inst @list(I)@ (notice the
similarity to the type definition for lists.)

Now we can write @list(non_empty_list)@, for instance, to refer to any
list of non-empty lists.

\section{Determinism}

We need to examine the various kinds of determinism in more detail.

Let's start by reviewing the determinism categories we've encountered so
far, plus one or two others that turn out to be useful:

\begin{tabular}{l|l|l}
Category    & Failure       & Solutions \\
\hline \hline
@failure@   & always fails  & has no solutions \\
@semidet@   & may fail      & has at most one solution \\
@det@       & cannot fail   & has exactly one solution \\
@nondet@    & may fail      & may have several solutions \\
@multi@     & cannot fail   & may have several solutions \\
@erroneous@ & \emph{abnormal termination} \\
\end{tabular}

We haven't mentioned @failure@ and @erroneous@ before.

@failure@ is the determinism category for goals that cannot succeed.
This might seem like an odd thing to have, but it's rather like having
zero in mathematics: without it there's a gaping hole in the scheme.
The built-in predicate @false/0@ has determinism @erroneous@ and is
equivalent to @not true@.

@erroneous@ is the determinism category for goals like
@exception.throw/1@ which do not exit normally.  @exception.throw/1@
neither fails nor succeeds.  Instead, an \emph{exception} occurs which
returns control to the most recent \emph{exception handler} on the call
stack.  \XXX{Do I need to explain ``call stack''?}  This is explained
fully in chapter \XXX{} on exceptions.

How do we work out the determinism category for a goal (and hence a
procedure)?  It turns out we only have to understand five cases:
atomic goals, conjunction and disjunction, negation and conditional
goals.  From these we can generalise to any goal.

\subsection{Atomic Goals}

An atomic goal is either a unification or a procedure call.  The
determinism of a procedure call is given by the corresponding predicate
mode declaration.

Unifications come in four flavours:
\begin{description}
\item [constructions and assignments] always have exactly one
solution and hence have determinism category @det@;
\item [equality tests and deconstructions] either fail or succeed once,
giving them a determinism category of @semidet@.
\end{description}

\subsection{Conjunctions}

Consider the goal @P, Q@.

If either @P@ or @Q@ has determinism @failure@ then the conjunction
as a whole has determinism @failure@.

Similarly, if either @P@ or @Q@ has determinism @erroneous@ then the
conjunction as a whole has determinism @erroneous@.

Otherwise the determinism of the conjunction is worked out as follows:
if either of @P@ or @Q@ can fail then the conjunction as a whole can
fail.  If either of @P@ or @Q@ may have several solutions then the
conjunction as a whole may have several solutions.

The following table gives the determinism category of a conjunction from
the determinisms of its conjuncts:

\begin{center}
\begin{tabular}{l|llll}
              & @semidet@   & @det@       & @nondet@    & @multi@ \\
\hline
@semidet@     & @semidet@   & @semidet@   & @nondet@    & @nondet@ \\
@det@         & @semidet@   & @det@       & @nondet@    & @multi@ \\
@nondet@      & @nondet@    & @nondet@    & @nondet@    & @nondet@ \\
@multi@       & @nondet@    & @multi@     & @nondet@    & @multi@ \\
\end{tabular}
\end{center}

\subsection{Disjunctions}

Consider the goal @(P ; Q)@.

If @P@ has determinism @failure@ then the disjunction as a whole has
the same determinsm of @Q@ -- and vice versa.

Similarly, if @P@ has determinism @erroneous@ then the disjunction as a
whole has the same determinsm of @Q@ -- and vice versa.

Otherwise, if both @P@ and @Q@ can fail then the disjunction as a whole
can fail and if either of @P@ or @Q@ may have several solutions then the
disjunction as a whole may have several solutions.

The following table gives the determinism category of a disjunction from
the determinsms of its disjuncts:

\begin{center}
\begin{tabular}{l|llll}
            & @semidet@   & @det@       & @nondet@    & @multi@ \\
\hline
@semidet@   & @semidet@   & @multi@     & @nondet@    & @nondet@ \\
@det@       & @multi@     & @det@       & @multi@     & @multi@ \\
@nondet@    & @nondet@    & @multi@     & @nondet@    & @multi@ \\
@multi@     & @nondet@    & @multi@     & @multi@     & @multi@ \\
\end{tabular}
\end{center}

\subsection{Negations}

Consider the goal @not P@.

If @P@ has determinsm @failure@ then the negation has determinsm @det@.

If @P@ has determinsm @erroneous@ then the negation has determinsm
@erroneous@ as well.

Otherwise, if @P@ cannot fail then the negation has determinsm
@failure@, otherwise it has determinsm @semidet@.

The following table gives the determinism category of a negation from
the determinism of its subgoal:

\begin{center}
\begin{tabular}{l|l}
@P@           & @not P@ \\
\hline \hline
@semidet@     & @semidet@ \\
@det@         & @failure@ \\
@nondet@      & @semidet@ \\
@multi@       & @failure@ \\
\end{tabular}
\end{center}

\subsection{Conditional Goals}

Consider the goal @( if P then Q else R )@.

This is semantically equivalent to @(P, Q ; not P, R)@.  We can use this
form to work out the determinism of the conditional goal as a whole from
the determinism of @P@, @Q@ and @R@.

\section{Polymorphic Modes}

\XXX{Do I want to mention this here?  At all?}

\section{Uniqueness}

\XXX{I think this should probably go after the insts section and before
the determinism section.}

So far, the only unique structure we've seen has been the @io.state@ used by
IO operations.  Uniqueness, however, is needed by several other structures
such as arrays and stores that we'll come to in later chapters \XXX{}.

The key characteristic of these structures is that, for reasons of either
efficiency or correctness, values of these types can only ever have one
\emph{live} reference at a time.  Liveness, in this case, means that the
referring variable in question can be used \emph{at most once} in subsequent
computation (except for one special case that will be explained shortly.) As
a consequence, a value cannot be unique at a particular point in a program
if there are two or more live variables that refer to it or if the
computation can be backtracked over.

There are three main argument modes associated with unique values:
\begin{itemize}
\item @di@ -- destructive input arguments must be unique at the start of the
call and are \emph{dead} (\ie no longer available) afterwards;
\item @ui@ -- unique input arguments must be unique at the start of the call
and are still unique afterwards;
\item @uo@ -- unique output arguments are like ordinary @out@ arguments
except that they are guaranteed to be unique.
\end{itemize}
\XXX{Fix @ui@ modes.}

These built-in modes are defined as follows:
\begin{verbatim}
:- mode di == (unique >> dead  ).
:- mode ui == (unique >> unique).
:- mode uo == (free   >> unique).
\end{verbatim}

The inst @unique@ is like @ground@, except that every part of the
corresponding value (if it has structure) is also required to be unique.

Insts describing values whose top-level functor is unique, but whose
arguments may not be, are defined as follows:
\begin{myverbatim}
:- inst unique_list(I) == unique( [] ; [I | unique_list(I)] ).
\end{myverbatim}
This defines a new inst, @unique_list/1@ whose top-level functor is unique
and bound to either a @[]/0@ or a @[|]/2@ value with arguments of inst @I@
and @unique_list(I)@ respectively.

\XXX{It's a shame that we don't have a simpler syntax for unique(...)
in the same was as we do for bound(...).}

\XXX{Add top\_unique inst which only applies to the top-level functor.}

\XXX{The ui mode doesn't yet work.}

\XXX{Nested uniqueness doesn't yet work.}

\XXX{I don't want to say too much here until we've got uniqueness sorted
out.}

\subsection{Uniqueness and Reuse}

\XXX{Reuse doesn't yet work.}

\section{Higher Order Modes}

\XXX{This will go in the HO chapter.}

\section{Committed Choice} 

\XXX{Do I want to mention this here?  At all?}

\section{Mostly Uniqueness}

\XXX{Do I want to mention this here?  At all?}

\section{Conclusion}

Modes are helpful documentation for the programmer.

Modes can be used to implement subtyping.

Modes allow the compiler to generate efficient code by reordering for
each procedure.

% vim: ft=tex ff=unix ts=4 sw=4 et wm=8 tw=72

\chapter{Modules}

A Mercury program is made up of one or more \emph{modules} that are
compiled separately and then linked together to make a working
executable.  Modules serve three purposes: they allow code files to
be compiled and developed independently of the rest of the program; they
group sets of names into \emph{namespaces}; and they
control the \emph{visibility} of names.

\XXX{Should I add something along the lines that Mercury's module system
is simpler and, perhaps curiously, considerably more expressive than the
public/private/protected wobble used in Java/C$\sharp$/etc?}

\section{Module Structure}

Mercury modules look like this:
\begin{myverbatim}
:- module modulename.
:- interface.

    ...interface section...

:- implementation.

    ...implementation section...
\end{myverbatim}
A module called @modulename@ needs to go in a file called
@modulename.m@.  Consequently each file can contain only one module.
\XXX{I know this isn't strictly true when we talk about submodules, but
I really don't want to split hairs at this point.}

\subsection{Exported Names and Private Names}

Consider a file @a.m@ containing
\begin{myverbatim}
:- module a.
:- interface.

:- type foo ---> ...

:- func f(foo) = foo.

:- implementation.

:- type bar ---> ...

f(X) = ...

:- pred p(bar, foo).
:- mode p(in,  out) is det.

p(Y, Z) :- ...
\end{myverbatim}
Module @a@ \emph{exports} the names defined or declared in its interface
section.  In this case, @a@ exports the type @foo@ and the function
@f/1@.

Clauses defining predicates and functions are not allowed in the
interface section because we want to hide implementation detail from the
module's users.  The definition for @f/1@ therefore has to go in
@a@'s implementation section.

The names declared in @a@'s implementation section are \emph{private}.
In other words, they are not usable anywhere outside the implementation
section for @a@.  Here, the type @bar@ and the predicate @p/2@ are
private to @a@.

\subsection{Importing Modules}

Now consider the file @b.m@:
\begin{myverbatim}
:- module b.
:- interface.
:- import_module a.

:- type baz ---> ...

:- pred q(a.foo, baz).
:- mode q(in,    in ) is semidet.

:- implementation.

q(X, Y) :- ...
\end{myverbatim}
The @import_module@ declaration in the interface section of module @b@
allows @b@ to use names defined in module @a@ (in both the interface and
implementation sections).

Module @b@ exports the type @baz@ and the predicate @q/2@ whose first
argument is of type @a.foo@ (\ie the type @foo@ defined in module @a@)
and whose second argument is of type @baz@ (defined in module @b@).

Now let's look at file @c.m@:
\begin{myverbatim}
:- module c.
:- interface.
:- import_module b.

:- type quux ---> ...

:- func g(quux) = b.baz.

:- implementation.
:- import_module a.

g(X) = ...

:- pred r(a.foo, b.baz).
:- mode r(in,    out  ) is multi.

r(Y, Z) :- ...
\end{myverbatim}
Module @c@ imports module @b@ in its interface section and can therefore
use names exported by @b@ in both its interface and implementation
sections.  The function declaration for @g/1@ in the interface section
refers to @b.baz@, as does the predicate declaration for @r/2@ in the
implementation section.

The declaration for @r/2@ in the implementation section also refers to
@a.foo@.  Imports are not transitive, so module @c@ still has to
explicitly import module @a@ in order to use @a.foo@, regardless of the
fact that module @b@ imports module @a@ in its interface section.  Note
that the scope of the import declaration in the implementation section
extends only over that section and not to the interface section.

For economy of typing, it is possible to condense several import
declarations, such as
\begin{myverbatim}
:- import_module a.
:- import_module b.
:- import_module c.
\end{myverbatim}
into one if you wish:
\begin{myverbatim}
:- import_module a, b, c.
\end{myverbatim}

\section{Mutually Dependent Modules}

Mercury has no problem with module @a@ importing module @b@ while at the
same time module @b@ imports module @a@.

The Mercury standard library modules @float@ and @math@, for example,
are mutually dependent.  The @float@ module defines the basic operations
on floating point numbers and uses the type @math.domain_error@ to
report things like division by zero.  The @math@ module defines more
complex floating point operations, such as the trignometric functions,
but necessarily depends upon the operations defined in the @float@
module.

\section{Overloading and Module Qualified Names}

Reusing the same name withing the same module.

Reusing the same name as exported by a different module.

\section{The Top-Level Module}

Every program must have a top-level module that exports a @main/2@
predicate with the following signature:
\begin{myverbatim}
:- pred main(io, io).
:- mode main(di, uo) is det.
\end{myverbatim}
(where @io@ is a non-module qualified synonym for @io.state@.)
@main/2@ is the starting point for any Mercury program.

If the top-level module is called @foo@, say, then one can compile the
program into an executable with just
\begin{myverbatim}
> mmc --make foo
\end{myverbatim}
This will recompile only the modules that have been changed since the
last time they were compiled, along with any modules that import them.
In this way, recompilation time for large projects is kept to a minimum.

Chapter \XXX{} discusses the Mercury compiler and related tools in
detail.



\section{Abstract Types}



\section{Submodules}



\subsection{Nested Submodules}



\subsection{Separate Submodules}



\section{Conclusion}

% % vim: ft=tex ff=unix ts=4 sw=4 et wm=8 tw=0

\chapter{Higher Order Programming}

Predicates and functions are first class values, just like values of
type @int@, @string@, @list(char)@ and so forth.  They can be
passed around and constructed just like any other kind of value.  Code
that manipulates predicate values is said to be \emph{higher order}.

\XXX{Should say somewhere that equality and ordering are undefined on
higher order values.}

What is special about predicate values is that they can be
\emph{applied} to arguments in order to carry out tests or compute other
values.

This turns out to be a surprisingly useful and flexible way to program.
Indeed, it is one of the key reasons why one should consider a typical
declarative programming language in preference to the more common
imperative languages.

\section{Example: the Map Function}

It is very common to want to apply a function to each member of a @list@
to obtain the @list@ of transformed values.  That is, given a function
@F@ and a @list@ @[X1, X2, X3, ...]@, we want to compute the @list@
@[F(X1), F(X2), F(X3), ...]@.

We can achieve this with the higher order function @map@:
\begin{verbatim}
:- func map(func(T1) = T2, list(T1)) = list(T2).

map(_, []      ) = [].
map(F, [X | Xs]) = [F(X) | map(F, Xs)].
\end{verbatim}
The first argument @F@ is declared to be a function itself computing
values of type @T2@ from values of type @T1@, where @T1@ and @T2@ can be
anything.  The second argument is declared to be a @list(T1)@ and the
result of the @map@ operation is a @list(T2)@.

The first clause states that if the second argument is the empty list
@[]@, then so is the result.

The second clause states that if the second argument is a @list@ with head
@X@ and tail @Xs@, then the result is the @list@ whose head is computed by
applying @F@ to @X@, namely as @F(X)@, and whose tail is computed by
@map(F, Xs)@.

It follows that
\begin{verbatim}
map(F, [X1, X2, X2, ...]) = [F(X1), F(X2), F(X3), ...]
\end{verbatim}
as required.  Since @map/2@ is polymorphic, it will work for any @F@
with the appropriate signature -- there is no need to recode it for each
particular function we wish to map over a @list@.

\section{Example: the Foldr Function}

\XXX{Should probably reference ``Why Functional Programming Matters''.}

Say we need to write a function @sum@ that will compute the sum of a
@list@ of @int@s.  We could reason as follows: the sum of a list whose
head is @X@ and whose tail is @Xs@ is just @X@ plus the sum of the @Xs@.
Since we would like @sum(Xs ++ Ys) = sum(Xs) + sum(Ys)@, for any @Xs@
and @Ys@, we would conclude that the sum of the empty list must be @0@.
Hence
\begin{verbatim}
:- func sum(list(int)) = int.

sum([]      ) = 0.
sum([X | Xs]) = X + sum(Xs).
\end{verbatim}
Observe that @sum([X1, X2, X3]) = X1 + (X2 + (X3 + 0))@.

By a similar process we could arrive at a function @prod@ that computed
the product of a @list@ of @int@s:
\begin{verbatim}
:- func prod(list(int)) = int.

prod([]      ) = 1.
prod([X | Xs]) = X * prod(Xs).
\end{verbatim}
Observe this time that @prod([X1, X2, X3]) = X1 * (X2 * (X3 * 1))@.

At this point we see that the definitions of @sum@ and @prod@ are almost
identical, the only difference being the result for the empty list and
the function used to combine the head of the list with the result of
processing the tail.

What would be most useful would be an higher order function that
generalised this pattern so that we only have to get it right once and
can reuse it thereafter for @list@ processing functions with a similar
pattern to @sum@ and @prod@.  We call this function @foldr@ and it takes
two arguments: @A@ is the value returned when the @list@ in question is
empty and @F@ is the function that computes the combination of the head
of the @list@ with the result of processing its tail.
\begin{verbatim}
:- func foldr(func(T1, T2) = T2, T2, list(T1)) = T2.

foldr(_, A, []      ) = A.
foldr(F, A, [X | Xs]) = F(X, foldr(F, A, Xs)).
\end{verbatim}
(The type signature for @foldr@ might look a little intimidating, but a
little careful consideration should make it obvious what's going on.)

\XXX{By the way, the standard library gets the argument order wrong as
far as Currying is concerned.  Can we fix it for v2 or are we stuck with
the current ordering?}

Now, with the aid of a couple of auxiliary functions,
\begin{verbatim}
:- func plus(int, int) = int.

plus(X, Y) = X + Y.

:- func times(int, int) = int.

times(X, Y) = X * Y.
\end{verbatim}
we can go on to define @sum@ and @prod@ in terms of @foldr@:
\begin{verbatim}
sum(Xs) = foldr(plus, 0, Xs).

prod(Xs) = foldr(times, 1, Xs).
\end{verbatim}
(The reason we had to define @plus@ is that the standard @int@ library
function @+@ has more than one mode -- given any two arguments to the
equation @X + Y = Z@ one can obtain the third, although this is not true
of integer multiplication (we define @times@ merely for symmetry) -- and
Mercury currently has no syntax for specifying a particular procedure of
a function or predicate nor the means to identify from context, in
higher order code in general, which is otherwise intended.) \XXX{This is
a complicated paragraph.}

@foldr@ is surprisingly general.  For instance, consider the definition
of the @list@ concatenation function, whose name is the infix operator
@++@:
\begin{verbatim}
:- func list(T) ++ list(T) = list(T).

[]       ++ Ys = Ys.
[X | Xs] ++ Ys = [X | Xs ++ Ys].
\end{verbatim}
Once we have @foldr@, there's no need for us to have to think about the
recursion any more.  We can instead write
\begin{verbatim}
Xs ++ Ys = foldr(cons, Ys, Xs).

:- func cons(T, list(T)) = list(T).

cons(X, Xs) = [X | Xs].
\end{verbatim}
(we have to define @cons@ because data constructors such as @[|]@ are
\emph{not} functions, not least because they can also be used as
``deconstructors'' in unifications and pattern matching.)

It is worth spending the time to become familiar with higher order
functions such as @foldr@.  Such functions enable you to solve the
general case once and then never have to expend mental or physical
effort duplicating the scheme for each specific application.

\section{Lambdas}

Sometimes it is a little painful to have to name each and every small
auxiliary function when writing higher order code.  Lambdas are
auxiliary predicate or function procedures that can be constructed on an
as-needs basis in a program and passed around just as if they had been
defined as separate predicates or functions in their own right.  The
main difference is that lambdas do not have names (they are sometimes
described as `anonymous') and therefore cannot be recursive.

We illustrate lambdas by recoding @sum@, @prod@ and @++@:
\begin{verbatim}
sum(Xs)  = foldr((func(A, B)  = A + B   ), 0, Xs).

prod(Xs) = foldr((func(A, B)  = A + B   ), 1, Xs).

Xs ++ Ys = foldr((func(A, As) = [A | As]), Ys, Xs).
\end{verbatim}
As with most coding short-cuts, lambdas can make code both more and less
legible.  As a general rule, lambdas are best kept brief and used in
situations where their purpose is obvious, or, in other words, if you
think an explanatory comment is justified, then avoid using a lambda.
Their use is justified in the above cases.

\XXX{Talk about lambdas with bodies.  They only get one clause.}

\XXX{Talk about predicate lambdas, including nondeterminism etc.}

\XXX{Say that it's fine to unify lambdas with variables etc.}

\XXX{Don't forget the scope rules.}

\section{Partial Application (Currying)}

Looking at the definition of @map@ once more,
\begin{verbatim}
map(_, []      ) = [].
map(F, [X | Xs]) = [F(X) | map(F, Xs)].
\end{verbatim}
we see exactly the same pattern of recursion captured by @foldr@, hence
we can recode it as
\begin{verbatim}
map(F, Xs) = foldr(apply_cons(F), [], Xs).

:- func apply_cons(func(T1) = T2, T1, list(T2)) = list(T2).

apply_cons(F, X, Ys) = [F(X) | Ys].
\end{verbatim}
using the auxiliary function @apply_cons@.  

At this point one may be prompted to ask, ``What is this strange
@apply_cons(F)@ appearing in the definition of @++@?  And besides,
@apply_cons@ takes three arguments.''

The expression @apply_cons(F)@ is a \emph{partial application} of
@apply_cons/3@, resulting in a \emph{closure} which is equivalent to
writing @func(A, Bs) = apply_cons(F, A, Bs)@.

\XXX{Are lambdas implemented any more efficiently than this behind the
scenese?}

(Note that in practice we would probably have just written
\begin{verbatim}
map(F, Xs) = foldr((func(X, Ys) = [F(X) | Ys]), [], Xs).
\end{verbatim}
and avoided the need for a named auxiliary function at all.)

\XXX{Watch out for procedure ambiguity using closures.}

\XXX{Mention that a fully applied func expression is applied whereas a
pred expression may not be.}

\XXX{Mention @call@ and @apply@.}

\XXX{What about restrictions on partial application?}

\section{Modes}
\section{* Monomorphism Restriction}
\section{* Monomoding Restriction}
\section{* Efficiency}




% \section{* Type Classes}
\subsection{OO Programming}
\subsection{Type Class Declarations}
\subsubsection{Method Signatures}
\subsubsection{Type Class Constraints}
\subsection{Instance Declarations}
\subsubsection{Method Implementations}
\subsubsection{Type Class Constraints}
\subsection{Existentially Quantified Types}
\subsubsection{Use}
\subsubsection{Why Output Only}
\subsection{...Constructor Classes}
\subsection{...Functional Dependencies}
\subsection{Restrictions and Explanations Thereof}
\subsubsection{On Type Class Definitions}
\subsubsection{On Instance Definitions}




% \include{std-util}
% \section{Lists}

Lists are perhaps the single most useful data structure in the
programmer's armoury.

The @list@ module in the Mercury standard library defines lists as
follows:
\begin{verbatim}
:- type list(T) ---> []
                ;    [T | list(T)].
\end{verbatim}
The notation @[A | B]@ is special syntactic sugar recognised by the
Mercury parser for @[|](A, B)@ -- that is, @[|]/2@ is the basic list
data constructor, with @[]@ standing for the empty list.

Moreover, an expression of the form @[A, B, C]@ is syntactic sugar for
@[A | [B | [C | []]]]@ which, of course, is identical to
@[|](A, [|](B, [|](C, [])))@.

Also, an expression of the form @[A, B, C | Xs]@ is syntactic sugar for
@[A | [B | [C | Xs]]]@ which, of course, is identical to
@[|](A, [|](B, [|](C, Xs)))@.

These are obviously singly linked lists.  \XXX{Discussion of the pros
and cons of the various list ADTs out there.}

\subsection{The Main List Operations}

The higher order functions @map@, @foldl@ and @foldr@ provide easy ways
of iterating over lists, either transforming them member by member or
somehow accumulating the result of processing each member in turn.

The section on higher order programming \XXX{} has already dealt with
@map@, @foldl@ and @foldr@ in some detail, so we merely summarise that
discussion here and concentrate on operations we have not previously
examined.

\subsubsection{Miscellany}

\begin{verbatim}
    % length([X1, X2, X3, ..., XN]) = N
    %
:- func length(list(T)) = int.

length(Xs) = foldl((func(_, N) = N + 1), Xs, 0).

    % reverse([X1, X2, X3, ..., XN]) = [XN, ..., X3, X2, X1]
    %
:- func reverse(list(T)) = list(T).

reverse(Xs) = foldl((func(X, Ys) = [X | Ys]), Xs, []).
\end{verbatim}

\subsubsection{Membership}

\XXX{Should have a section on equality and @compare@ and so forth.}

@member@ is used to decide membership of a @list@ under equality and to
non-deterministically project members from a @list@ (the argument ordering
is arguably unfortunate for higher order programming, but this is
historically how things have been done):
\begin{verbatim}
    % member(X, Xs) iff X is a member of Xs.
    %
:- pred member(T,   list(T)).
:- mode member(in,  in     ) is semidet.
:- mode member(out, in     ) is nondet.

member(X, [X | _ ]).
member(X, [_ | Xs]) :- member(X, Xs).
\end{verbatim}
From time to time one wants to access members of a @list@ by their index
(\ie distance from the start of the @list@).  There are two sets of
operations for doing so, depending upon whether it is most convenient to
give the head of the list an index of 1 or 0:
\begin{verbatim}
    % index0(Xs, I, X) iff 0 =< I < length(Xs) and
    % X is the I+1th member of Xs.  That is,
    % index0([X1, X2, X3], 1, X2) while
    % index0([X1, X2, X3], 3, _ ) fails.
    %
:- pred index0(list(T), int, T  ).
:- mode index0(in,      in,  out) is semidet.

    % index1(Xs, I, X) iff 1 =< I =< length(Xs) and
    % X is the Ith member of Xs.  That is
    % index1([X1, X2, X3], 1, X1) while
    % index1([X1, X2, X3], 3, _ ) fails.
    %
:- pred index1(list(T), int, T  ).
:- mode index1(in,      in,  out) is semidet.

    % These functions correspond to the predicates above, but differ
    % in that an exception is thrown if the index is out of range.
    %
:- func index0(list(T), int) = T.
:- func index1(list(T), int) = T.
\end{verbatim}

\subsubsection{Mapping}

@map@ applies its function argument to each member of its @list@ argument
and returns the corresponding @list@ of results.
\begin{verbatim}
    % map(F, [X1, X2, X3, ...]) = [F(X1), F(X2), F(X3), ...]
    %
:- func map(func(T1) = T2, list(T1)) = list(T2).

map(_, []      ) = [].
map(F, [X | Xs]) = [F(X) | map(F, Xs)].
\end{verbatim}

A related function is @map_corresponding@ which is used to map a
function combining the corresponding members of \emph{two} lists
(@map_corresponding@ will throw an exception if the lists are of
different lengths.)
\begin{verbatim}
    % map_corresponding(F, [X1, X2, X3, ...], [Y1, Y2, Y3, ...]) =
    %       [F(X1, Y1), F(X2, Y2), F(X3, Y3), ...]
    %
:- func map_corresponding(func(T1, T2) = T3, list(T1), list(T2)) =
            list(T3).

map_corresponding(_, [],       []      ) = [].
map_corresponding(_, [],       [_ | _ ]) = <<throw exception>>.
map_corresponding(_, [_ | _ ], []      ) = <<throw exception>>.
map_corresponding(F, [X | Xs], [Y | Ys]) =
    [F(X, Y) | map_corresponding(F, Xs, Ys)].
\end{verbatim}

There is also a @map_corresponding3@ which works for functions of three
arguments:
\begin{verbatim}
:- func map_corresponding3(func(T1, T2, T3) = T4,
            list(T1), list(T2), list(T3)) = list(T4).
\end{verbatim}

\XXX{Should probably include a subsubsection on zipping and
interleaving.}

\subsubsection{Folding}

The two main folding operations are @foldl@ and @foldr@.  We have
already seen a definition of @foldr@ (the version supplied in the
Mercury standard library unfortunately uses a slightly different
argument ordering):
\begin{verbatim}
    % foldr(F, [X1, X2, X3], A) = F(X1, F(X2, F(X3, A)))
    %
:- func foldr(func(T1, T2) = T2, list(T1), T2) = T2.

foldr(_, [], A) = A.
foldr(F, [X | Xs], A) = F(X, foldr(Xs, A)).
\end{verbatim}
In many situations the function @F@ will be commutative or we will want
to process the @list@ starting with the leftmost member, in which case
it is more efficient to use the tail recursive @foldl@:
\begin{verbatim}
    % foldl(F, [X1, X2, X3], A) = F(X3, F(X2, F(X1, A)))
    %
:- func foldl(func(T1, T2) = T2, list(T1), T2) = T2.

foldl(_, [],       A) = A.
foldl(F, [X | Xs], A) = foldl(F, Xs, F(X, A))).
\end{verbatim}
As an example, here's how we could define the @reverse@ function, as
well as more efficient versions of the @sum@ and @prod@ functions
introduced in the section on higher order programming \XXX{}:
\begin{verbatim}
reverse(Xs) = foldl((func(X, Ys) = [X | Ys]), Xs, []).
sum(Xs)     = foldl((func(X, A ) = X + A   ), Xs, 0 ).
prod(Xs)    = foldl((func(X, A ) = X * A   ), Xs, 1 ).
\end{verbatim}

\subsubsection{XXX More To Come}

\subsection{General Advice}

\XXX{This probably deserves its own top-level section.}

While lists are easy to understand and work with, most programmers show
an unfortunate tendency to use lists when another data structure may be
more appropriate.  It is worth spending some time looking at the various
types provided by the Mercury standard library to see what is available.
The decision as to which data structure is best for a given situation is
one that can only be made in the light of experience, although as a rule
of thumb you cannot go far wrong by picking the data structure with the
most useful set of support functions for the problem in hand
(occasionally an @assoc_list@ may be preferable to a @map@, but in most
situations it won't be.)


% % vim: ft=tex ff=unix ts=4 sw=4 et wm=8 tw=0

\chapter{Association Lists}




% % vim: ft=tex ff=unix ts=4 sw=4 et wm=8 tw=0

\chapter{Maps}

The @map@ data type is provided by the @map@ module in the standard
Mercury library.

After @list@s, @map@s are perhaps the most widely used non-trivial
Mercury data type.  A @map@ is essentially a dictionary structure
(or \emph{associative array})
mapping \emph{keys} to \emph{values}.  The core operations involve
setting up a mapping from key to value, changing the mapping for a
given key, and looking up the value associated with a given key.

The @map@ data type is parameterised by the types of keys and values:
\begin{verbatim}
:- type map(K, V).
\end{verbatim}
Maps have efficient $O(log n)$ cost bounds on all the basic
operations for a @map@ containing $n$ mappings (the reader may be
interested to know they are currently implemented using 234-trees
\XXX{ref}).

Here is a small example of a very basic telephone directory
application using a @map@ to store the mapping from names to
phone numbers:
\begin{verbatim}
:- import_module map, list, string, std_util, assoc_list, exception.

:- pred main(io, io).
:- mode main(di, uo) is det.

main(!IO) :-
    io__read_line_as_string(Result, !IO),
    (
        Result = eof
    ;
        Result = error(_),
        throw(Result)
    ;
        Result = ok(Name0),

            % Chop off the trailing new line character and
            % convert to lower case.
            %
        Name = to_lower(substring(Name0, 0, length(Name0) - 1)),

        ( if   phone_book ^ elem(Name) = Number
          then io__format("%d\n", [s(Number)], !IO)
          else io__format("`%s' is not in the phone book.\n",
                    [s(Name0)], !IO)
        ),
        main(!IO)
    ).

    % We memoize the result of this function since we only
    % want to build it once.
    %
:- func phone_book = map(string, string).
:- pragma memo(phone_book/0).

phone_book = map__from_assoc_list([
        "ralph" -       "9873 1234",
        "tyson" -       "9873 4342",
        "fergus" -      "9873 1237",
        "zoltan" -      "9876 8754",
        ...
    ]).
\end{verbatim}
\XXX{Is this a bit too ``clever'' for a tutorial example?}
\XXX{There's probably a better example.}

\section{The Main Map Operations}

An empty map is constructed by calling the nullary @init@ function:
\begin{verbatim}
:- func init = map(K, V).
\end{verbatim}
There are predicates to decide whether a given map is empty or
contains a mapping for a particular key:
\begin{verbatim}
:- pred map__is_empty(map(K, V)).
:- mode map__is_empty(in) is semidet.

:- pred map__contains_key(map(K, V), K).
:- mode map__contains_key(in,        in) is semidet.
\end{verbatim}
We have two field-access like functions for lookup up values
associated with keys.  The first, @elem@, simply fails if there
is no mapping for the given key.  The second, @det_elem@, will
throw an exception if this is the case.
\begin{verbatim}
:- func elem(K, map(K, V)) = V is semidet.

:- func det_elem(K, map(K, V)) = V.
\end{verbatim}
So the expression @Map ^ elem(Key)@ denotes the value associated
with @key@, if any.

Mappings can be added or changed using the field assignment-like
functions @'elem :='@ and @'det_elem :='@.  The expression
@Map ^ elem(Key) := Value@ always succeeds, returning an updated
version of @Map@ overwriting the mapping for @Key@, if any, in @Map@
or adding a new mapping if @Map@ does not contain one.  The expression
@Map ^ det_elem(Key) := Value@ is similar, except that it will throw
an exception if @Map@ does not already contain a mapping for @Key@.
\begin{verbatim}
:- func 'elem :='(K, map(K, V), V) = map(K, V).

:- func 'det_elem :='(K, map(K, V), V) = map(K, V).
\end{verbatim}

\section{Bulk Initialisation and Update}

The @map@ module provides several means of initialising and updating
@map@s from other structures relating keys to values.

\subsection{Association Lists}

The simplest dictionary type is @assoc_list@.
\begin{verbatim}
:- func from_assoc_list(assoc_list(K,V)) = map(K,V).

:- func from_sorted_assoc_list(assoc_list(K,V)) = map(K,V).
\end{verbatim}
These two functions both construct a new map from the @Key - Value@
mappings in the @assoc_list@ argument.  Mappings are inserted in the
order in which the @Key - Value@ pairs appear in the @assoc_list@.

Example (the 1926 estimates of Dr Catherine Morris Cox):
\begin{verbatim}
    IQs = from_assoc_list([
        "Bobby Fischer"   - 187,
        "Galileo Galilei" - 185,
        "Rene Descartes"  - 180,
        "Immanuel Kant"   - 175,
        "Charles Darwin"  - 165,
        "Albert Einstein" - 160
    ])
\end{verbatim}
If the @assoc_list@ in question is already sorted with distinct keys in
ascending order then using @from_sorted_assoc_list@ \emph{may} be faster
(but there's certainly no point in separately sorting the @assoc_list@
just to use this function instead!)  \XXX{It's not clear to me that we
really want to mention this one at all, especially since they're
currently implemented identically.}

The following function allows one to make several updates to a @map@
from an @assoc_list@:
\begin{verbatim}
:- func set_from_assoc_list(map(K,V), assoc_list(K, V)) = map(K,V).
\end{verbatim}

Example (wild guesses):
\begin{verbatim}
    NewIQs = set_from_assoc_list(IQs, [
        "Ralph Becket"     - 253,
        "Fergus Henderson" - 211,
        "Albert Einstein"  - 161,   % Revised estimate.
        "Tyson Dowd"       -  86
    ])
\end{verbatim}

\subsection{Corresponding Pairs in Lists}

Occasionally one has keys and values in separate @list@s.  One can
construct a @map@ from them using the following function:
\begin{verbatim}
:- func from_corresponding_lists(list(K), list(V)) = map(K, V).
\end{verbatim}
The @list@s must be the same length; if they aren't then this 
function will throw an exception.

Example:
\begin{verbatim}
    QualityOfPet =
        from_corresponding_lists(
            [cat, dog, snake], [good, bad, inadvisable]
        )
\end{verbatim}
giving @QualityOfPet ^ elem(cat) = good@ and
@QualityOfPet ^ elem(dog) = bad@ and
@QualityOfPet ^ elem(snake) = inadvisable@.
\XXX{This example may also need a little work.}

\subsection{Other Maps}

One can use the following functions to merge @map@s together:
\begin{verbatim}
:- func merge(map(K, V), map(K, V)) = map(K, V).

:- func overlay(map(K,V), map(K,V)) = map(K,V).
\end{verbatim}
The @merge@ function will throw an exception if the two maps have a key
in common.  The @overlay@ function will not, taking the values for
common keys from the second @map@ argument in the resulting @map@.

Example:
\begin{verbatim}
    RoutesA = from_assoc_list([
        adelaide - melbourne,
        perth - adelaide,
        melbourne - canberra
    ]),
    RoutesB = from_assoc_list([
        canberra - sydney,
        sydney - brisbane,
        brisbane - darwin,
        melbourne - alice_springs
    ]),
    Routes = overlay(RoutesA, RoutesB)
\end{verbatim}
will result in @Routes ^ elem(perth) = adelaide@ and
@Routes ^ elem(sydney) = brisbane@ and
@Routes ^ elem(melbourne) = alice_springs@.
\XXX{Another bad example.  Should use this one for @multi\_map@.}

\section{Miscellaneous Operations}

\XXX{This section needs examples throughout.}

One can delete the mappings for one or a @list@ of keys using the
following:
\begin{verbatim}
:- func delete(map(K,V), K) = map(K,V).

:- func delete_list(map(K,V), list(K)) = map(K,V).

:- pred map__remove(map(K,V), K, V, map(K,V)).
:- mode map__remove(in, in, out, out) is semidet.

:- pred map__det_remove(map(K,V), K, V, map(K,V)).
:- mode map__det_remove(in, in, out, out) is det.
\end{verbatim}
The @delete@ and @delete_list@ functions simply ignore key arguments
that do not have mappings in the @map@ argument.

The predicate @map__remove@ will fail if the given key is not present,
otherwise it returns both the value mapping of the key and a version of
the @map@ that does not contain a mapping for the key.

The present @map__det_remove@ is similar, except that it will throw an
exception if the given @map@ does not have a mapping for the key in
question.

One can obtain a count of the number of mappings in a map with
\begin{verbatim}
:- func count(map(K, V)) = int.
\end{verbatim}

One can obtain (sorted) lists of keys and values stored in a @map@:
\begin{verbatim}
:- func keys(map(K, V)) = list(K).

:- func sorted_keys(map(K, V)) = list(K).

:- func values(map(K, V)) = list(V).
\end{verbatim}
The functions @keys@ and @sorted_keys@ are identical, except that the
latter guarantees to return the keys in ascending order.  The values
returned by @values@ are not guaranteed to be in any order.

The following convert @map@s to @assoc_list@s:
\begin{verbatim}
:- func to_assoc_list(map(K,V)) = assoc_list(K,V).
:- func to_sorted_assoc_list(map(K,V)) = assoc_list(K,V).
\end{verbatim}
@to_sorted_assoc_list@ returns the corresponding @assoc_list@ with keys
in ascending order.

XXX HERE!

% % vim: ft=tex ff=unix ts=4 sw=4 et wm=8 tw=0

\chapter{Arrays}

An array in Mercury is a mutable vector of values, indexed by position
starting from zero.  Arrays are unique in that they provide $O(1)$
access and update.  In order to make this property safe in a declarative
language, arrays have to be unique: a mutable array can only ever have
one live reference at any given moment.

\section{The Array Type, Modes and Insts}

Arrays are abstract types,
\begin{verbatim}
:- type array(T).
\end{verbatim}

The three main modes for @array@s are @array_di@, @array_uo@ and
@array_ui@ with the same meaning as @di@, @uo@ and @ui@ respectively,
but specialised for @array@s.  An @array@ of values with more complex insts
(e.g. arrays of predicates) must be passed around using the parametric
insts @array(I)@ and @uniq_array(I)@, the former being used for
immutable @array@s.  \XXX{Of course, even the @array@ access preds don't use
the parametric modes, so they're rather useless except for inst
casting, in which case we don't really need them anyway...}

\section{Initialising Arrays}

Since Mercury @array@s are polymorphic and there is no equivalent notion
of C's @NULL@ and so forth, @array@s must be initialised with a
programmer-supplied default value to be be placed in each @array@
cell.  The first argument is the number of cells the @array@ is to have:
\begin{verbatim}
:- func init(int, T) = aray(T).
\end{verbatim}
The empty @array@ is a special case (@array@s can be resized) and can be
constructed without an initialiser:
\begin{verbatim}
:- func make_empty_array = array(T).
\end{verbatim}
Finally, it is possible to initialise an array from a list:
\begin{verbatim}
:- func array(list(T)) = array(T).
\end{verbatim}

\section{Array Bounds}

\XXX{The current design pays lip-service to the notion that we may one
day have offset arrays.  I don't think we should mention this here,
since I think it was a mistake to mention it and not go the whole hog.}

Array cells are indexed starting from zero.  The highest numbered cell
index in an @array@ with @N@ cells is therefore @N - 1@.  The @size@
function returns the number of cells in an @array@; the @max@ function
return the highest numbered index of the array, or @-1@ if the array is
empty \XXX{the documentation for @max/1@ should reflect this}:
\begin{verbatim}
:- func size(array(T)) = int.
:- func size(array_ui) = out is det.
:- func size(in      ) = out is det.

:- func max(array(T)) = int.
:- func max(array_ui) = out is det.
:- func max(in      ) = out is det.
\end{verbatim}
It follows that @size(A) = max(A) + 1@.

There is a predicate to check that a given index lies within a given
@array@'s bounds:
\begin{verbatim}
:- pred in_bounds(array(T), int).
:- mode in_bounds(array_ui, in ) is semidet.
:- mode in_bounds(in,       in ) is semidet.
\end{verbatim}

\section{Access and Update}

Array members are accessed via the conventionally named functions:
\begin{verbatim}
:- func array(T) ^ elem(int) = T.
:- mode array_ui ^ elem(in ) = out is det.
:- mode in       ^ elem(in ) = out is det.

:- func ( array(T) ^ elem(int) := T  ) = array(T).
:- mode ( array_di ^ elem(in ) := in ) = array_uo is det.
\end{verbatim}
These functions check that the given index is within the array bounds
and will throw an exception if not.  These lookup operations are both
$O(1)$ in space and time.  \XXX{How good are C compilers at moving
bounds checks out of inner loops?}

\XXX{Should I mention the unsafe versions?}

There are two other predicates each for access and update that are
sometimes useful.  The pair, @semidet_lookup@ and @semidet_set@, merely
fail if the index is out of bounds, rather than throwing an exception:
\begin{verbatim}
:- pred semidet_lookup(array(T), int, T  ).
:- mode semidet_lookup(array_ui, in,  out) is semidet.
:- mode semidet_lookup(in,       in,  out) is semidet.

:- pred semidet_set(array(T), int, T, array(T)).
:- mode semidet_set(array_di, in, in, array_uo) is semidet.
\end{verbatim}

The second pair, @slow_set@ and @semidet_slow_set@, do not require the
input @array@ to be unique:
\begin{verbatim}
:- func slow_set(array(T), int, T ) = array(T).
:- mode slow_set(array_ui, in,  in) = array_uo is det.
:- mode slow_set(in,       in,  in) = array_uo is det.

:- pred semidet_slow_set(array(T), int, T,  array(T)).
:- mode semidet_slow_set(array_ui, in,  in, array_uo) is semidet.
:- mode semidet_slow_set(in,       in,  in, array_uo) is semidet.
\end{verbatim}
If the input @array@ is not unique, then destructive update (i.e. $O(1)$
update) is not possible so a unique copy -- $O(n)$ -- of the input
@array@ is made made and then updated.  As before, @semidet_slow_set@
will fail rather than throw an exception if the index argument is out of
bounds.

% \XXX{Why is the unique input argument @array_ui@ rather than @array_di@ ?
% Now that we have mode-specific clauses, we should not have a problem.}

\section{Utility Functions and Predicates}

\XXX{Here!}

\section{Implementing Multi-Dimensional Arrays}

As provided, Mercury arrays are one-dimensional.  Some applications
require arrays with two or more dimensions.  These kinds of arrays come
in two flavours: rectangular and ragged.

A rectangular @array@ with dimensions @W1 * W2 * ... * Wn@ is a
one-dimensional @array@ whose cells are indexed by @(X1, X2, ..., Xn)@
where @0 =< Xi < Wi@, mapping to the index
@(W1 * X1) + (W2 * X2) + ... + (Wn * Xn)@ in the one-dimensional
representation.

A ragged @array@ is one whose cells are themselves @array@s.  Hence the
cell whose index is given by @(X1, X2, ..., Xn)@ in an $n$-dimensional
ragged array is accessed via
@A ^ elem(X1) ^ elem(X2) ^ ... ^ elem(Xn)@.  Since an @array@'s size is
not part of its type, there is no requirement that each of the elements
of @A ^ elem(X1) ^ elem(X2) ^ ... ^ elem(Xj)@ (for some @j =< n@) have
the same size.  Accessing a cell in a ragged @array@ is usually a little
slower than accessing a cell in a rectangular @array@; however, ragged
@array@s may be much more economical in terms of space for sparse data.

Representing rectangular and ragged arrays each requires a little more
work.

\subsection{Rectangular Arrays}

\XXX{I've written the @table@ library module.  Get it checked in and
talk about it here.}

It is necessary to construct a special-purpose representation for each
number of dimensions.  Here is an example of how one might code up a
two-dimensional @array@ ADT:
\begin{verbatim}

\end{verbatim}

\subsection{Ragged Arrays}

\section{Caveats}

\XXX{They're not really unique at the moment.}

\XXX{Get nested uniqueness working.}

% % vim: ft=tex ff=unix ts=4 sw=4 et wm=8 tw=0

\chapter{Compiling Programs}

The Melbourne Mercury compiler comes with a suite of powerful tools to
simplify the compilation of Mercury programs for various purposes,
including building executables, shared and static libraries, linking
with foreign language object files and so forth.

This chapter aims to give a very brief introduction to the most
commonly used facilities of the compiler and its associated tools.

\section{The Mercury Directory}

But first, a useful tip.  If you create a subdirectory called @Mercury@
in the directory in which you will be running the compilation, then the
Mercury tools will assume (unless explicitly told otherwise) that all
intermediate files used in the compilation process (of which there are
many) should be placed in the Mercury directory.  This helps to avoid
cluttering up the working directory with files that should not be edited
by hand or, for the most part, be of any interest to the programmer.

The Mercury build tools are fairly smart and will go to some lengths to
avoid unnecessary recompilation.  However, there are rare occasions in
which they cannot work out that a particular intermediate file needs
updating, in which case one way of solving the problem is to simply
delete all the contents of the @Mercury@ directory, for example with 
@rm -rf Mercury/*@, and restart the build process from scratch.

\section{Building Executables}

\XXX{Make a comment somewhere at the start of the book that shell script
stuff lines will start with a @\$@ in the first column.}

The simplest way to build an executable from a Mercury program is just
this:
\begin{verbatim}
$ mmc --make foo
\end{verbatim}
where @foo@ is the name of the top-level Mercury module defining the
@main/2@ predicate.

The @--make@ option to @mmc@ will cause it to examine all the Mercury
dependencies of module @foo@ (i.e. the modules it imports, the modules
they import and so forth) and carry out any necessary compilation steps
for either @foo@ or any if its dependencies.  @mmc@ will try to do only
the minimum amount of work necessary to create an up-to-date executable
(i.e. it will avoid unnecessary recompilation of @foo@ or its
dependencies.)  There is no need to separately compile @foo@'s
dependencies -- @mmc --make@ will work it all out for you.

It can also be useful to specify @--smart-recompilation@.  With this
option turned on, @mmc@ will even avoid unnecessary
recompilation in the case where the interface of a dependency has
been changed, but not in a way that would make any difference (this
can be a \emph{real} timesaver for large projects.)

\section{Mmake}

Some projects may require various preprocessing steps or special options
to be set as part of the build process.  As a rule of thumb, any time
one needs more than @mmc --make@ then one should probably use the @mmake@
system.  If you need to use various other compilation options then @mmake@
is almost certainly the easiest tool to use.

\subsection{Mmakefiles}

@mmake@ is based upon the well known @make@ tool, common on Unix systems,
amongst others.  In order to use @mmake@, one first needs to write a
file called @Mmakefile@ with the necessary build instructions.

Here are the contents of a sample @Mmakefile@:
\begin{verbatim}
MAIN_TARGET = foo
depend: $(MAIN_TARGET).depend

GRADEFLAGS = --debug
\end{verbatim}
By and large, the order of the lines in an @Mmakefile@ doesn't make any
difference.  Each line in an @Mmakefile@ is either an assignment to an
variable or a dependency rule.

A variable assignment looks like @GRADEFLAGS = --debug@ where the variable
name appears to the left of the @=@ sign and the string to be assigned to
it appears on the right.  Any current environment variables are also
visible to an @Mmakefile@, so if you are going to run @mmake@ knowing that
an environment variable, @BAZ@, say, is going to be set, you can refer to
@BAZ@ in your @Mmakefile@ just as if it had been defined in the @Mmakefile@
itself.

Whenever @$(BAZ)@ appears in an @Mmakefile@, the string assigned to @BAZ@
is substituted instead.

If you need a @$@ sign to appear anywhere in an @Mmakefile@,
you need to write @$$@ instead: @mmake@ assumes that anything else
is introducing a variable substitution.

A simple dependency rule looks like this:
\begin{verbatim}
depend: $(MAIN_TARGET).depend
\end{verbatim}
The name of the \emph{target} appears to the left of the colon and the
names of the targets it depends on (there may be several, separated
with spaces) appear to the right.  Amongst other things, @mmake@
analyses the dependency rules and ensures the various targets are
up-to-date in bottom-up order (a target is deemed up-to-date if the
corresponding file is newer than those for its dependencies.)

Given the simple @Mmakefile@ above, if we invoke @mmake depend@ then
@mmake@ will start by building the target @foo.depend@ (since @foo@
is assigned to @MAIN_TARGET@.)  In this case @depend@ is a so-called
\emph{dummy target} since a file called @depend@ is never actually
created.  The target @foo.depend@ causes the Mercury build system to
examine the module @foo@ and compute its dependencies.  It is important
to build @foo.depend@ before attempting to compile @foo@.  Moreover,
any time the dependencies for @foo@ change (e.g. by changing the
interface of @foo@ or any of its dependencies) then @foo.depend@ must
be rebuilt for the compilation process to work properly.

Next, we can invoke just @mmake@ on its own, in which case it will
attempt to build the module named in @MAIN_TARGET@ (if it has a value)
as an executable.  Otherwise one can name an explicit target.  For
instance, one could simply do
\begin{verbatim}
$ mmake foo.depend
$ mmake foo
\end{verbatim}
without having to use an @Mmakefile@ at all!  (Of course, in this
case it's easier to just invoke @mmc --make foo@.)

Dependency rules can also include an explicit build procedure for a
target that overrides any default process that @mmake@ might have.
For example, say we needed to preprocess a given file, @bar.z@ say,
in order to obtain the requisite Mercury module, we would include
the following in the @Mmakefile@:
\begin{verbatim}
bar.m: bar.z
        cat bar.z | my_preprocessor > bar.m
\end{verbatim}
Now, when @mmake@ needs @bar.m@ and sees that either @bar.z@ is
newer than @bar.m@
or that @bar.m@ does not exist at all, it will run the commands
that appear on the lines below the dependency rule.  Commands
should be indented by a single hard tab character (ASCII code 9,
not spaces) and there should be no blank lines between commands
if there are more than one.

@mmake@ already knows how to compile ordinary Mercury code.  The main
reason for supplying an explicit build procedure is to handle foreign
code that will be linked with the output from the Mercury compiler to
construct the executable.

The main use for the @mmake@ system is to handle complex settings
for calls to the Mercury tools.  One invariably runs the
build process more than once while developing a program, so it
helps to only have to write down the various flags and settings
once in an @Mmakefile@.

\subsection{Module-Specific Options}

\subsection{Cleaning Up}

Targets clean and realclean.

\section{Compilation Grades}

Standard.

Debugging.

Profiling.

--use-grade-subdirs

\section{Linking Mercury Programs with Foreign Code}

\section{Building and Installing Libraries}

Tools know from installation where standard libs etc. live.

When installing and using libraries from elsewhere we have to pass
this information to the tools.


\section{Useful Compiler Flags and Environment Variables}

% % vim: ft=tex ff=unix ts=4 sw=4 et wm=8 tw=0

\chapter{Stores}

Efficient implementation of many algorithms depend upon being able to
create and traverse edges in a graph in @O(1)@ time.  Imperative
languages tend to use pointers (references into memory) for this
purpose, be they explicit as in C and C++ or implicit as in Java.

The only way to make edge addition \emph{and} traversal $O(1)$ in a pure
declarative language is to require that the graph structure be unique
(otherwise the best one can do is $O(\log n)$ access and update.)
Uniqueness means that a strict order of accesses and updates is imposed
on the graph and that there can never be any references to old versions
of the graph.  To this end, the Mercury standard library @store@
provides the @store@ datatype.

A value can be put into a store, and the store will return a
reference, called a @mutvar@ for \emph{mutable variable}, to the value
that can be used to retrieve it again at a later date.  Moreover, the
value that a @mutvar@ refers to can be changed within the store (that
is, the @mutvar@ can be made to refer to something else; the value in
the store is not changed in any way.)  We can compose complex graph
structures by including @mutvar@s in the values placed in the store.
Indeed, it is even possible to create cyclic systems of @mutvar@s.

Other useful advantage that @stores@ have include
\begin{itemize}
\item the fact that @mutvar@s are typed, so one cannot get any
surprises when derefencing,
\item there is no notion of the @NULL@ reference, as in C, so one's
program will never fail when dereferencing, and
\item @mutvar@s are tied to a particular @store@ -- you cannot
inadvertently use the @mutvar@ from one @store@ to access data in a
different @store@.
\end{itemize}

Stores are also extremely useful for holding data that needs to be
accessed and updated very quickly.

\section{The Store Type Class}

For pragmatic reasons, it is useful to be able to treat the @io.state@
also as a @store@ in its own right.  This is achieved by requiring the
@store@ module predicate arguments to be instances of the @store@
abstract type class defined in the @store@ module.  Only the @io.state@
and the abstract @store@ datatype are instances of this type class:
\begin{verbatim}
:- typeclass store(S).

:- type store(S).

:- instance store(io.state).
:- instance store(store(S)).
\end{verbatim}

\section{Creating Stores}

A @store@ is an abstract data type:
\begin{verbatim}
:- type store(S).
\end{verbatim}

A new store is created via a call to @store.new@:
\begin{verbatim}
:- some [S] pred new(store(S)).
:- mode          new(uo      ) is det.
\end{verbatim}
A store is unique and its type includes an existentially quantified type
parameter.  This is the mechanism by which we ensure that two different
stores cannot be confused: the compiler cannot prove whether two
@store@s with types @some [S1] store(S1)@ and @some [S2] store(S2)@
respectively do in fact have the same type, so it forbids any attempt to
treat the @store@s interchangably.  (This is an instance of using
\emph{shadow types} whereby the type of the values of interest includes
type parameters that do not refer to anything in the values of the type,
but are used to convey extra information via the type system -- in this
case, that @mutvar@s from different @store@s are not interchangable.)

\section{Creating References}

When a value is added to a @store@, a new @mutvar@ is returned.  The
@mutvar@ can then be used to retrieve the value from the @store@ or to
change the value in the @store@ to which it refers.  In this sense,
@mutvar@s are to all intents and purposes mutable variables, much like
variables in imperative programming languages.

Since both @io.state@s and @store@s are in the @store@ type class,
@mutvar@s are described thus:
\begin{verbatim}
:- type generic_mutvar(T, S).

:- type io_mutvar(T)       == generic_mutvar(T, io__state).
:- type store_mutvar(T, S) == generic_mutvar(T, store(S)).
\end{verbatim}
Here we see that each @mutvar@'s type is tied both to its @store@ and to
the type of values to which it refers.  This means the compiler will
spot and reject any program that erroneously attempts to use a @mutvar@
either with the wrong @store@ or to refer to values of the wrong type.

A @mutvar@ is created when a value is added to a @store@:
\begin{verbatim}
:- pred new_mutvar(T,  generic_mutvar(T, S), S,  S ) <= store(S).
:- mode new_mutvar(in, out,                  di, uo) is det.
\end{verbatim}

We can retrieve the referent of a @mutvar@:
\begin{verbatim}
:- pred get_mutvar(generic_mutvar(T, S), T,   S,  S ) <= store(S).
:- mode get_mutvar(out,                  out, di, uo) is det.
\end{verbatim}

We can change the referent of a @mutvar@:
\begin{verbatim}
:- pred get_mutvar(generic_mutvar(T, S), T,  S,  S ) <= store(S).
:- mode get_mutvar(out,                  in, di, uo) is det.
\end{verbatim}

\section{Creating Cyclic References}

Given that the above predicates only create a @mutvar@ when a value is
added to a @store@, they cannot be used to directly construct cyclic
structures (one could use something like the @maybe@ type from
@std_util@ to indicate a field where a @mutvar@ will go, but this would
be ugly and likely to lead to programs with erroneously
``uninitialised'' @mutvar@ fields.)

The @new_cyclic_mutvar@ predicate exists to solve this problem:
\begin{verbatim}
:- pred new_cyclic_mutvar(
            func(generic_mutvar(T, S)) = T,
            generic_mutvar(T, S),
            S, S
        ) <= store(S).
:- mode new_cyclic_mutvar(in, out, di, uo) is det.
\end{verbatim}
What happens is this: a new, unitialised @mutvar@ is constructed and
passed to the function argument, which can take it as a reference to the
value it is about to construct.  Since the function does not have access
to the @store@, it cannot dereference the @mutvar@, hence it is safe to
leave it uninitialised at this point.  Finally, the value returned by
the function is made the referent of the @mutvar@ which is itself
returned by @new_cyclic_mutvar@.

\section{Example: Rings}

Perhaps the simplest cyclic structure is the ring.  Here we show how to
implement singly-linked rings using @store@s.

\begin{verbatim}
:- type ring(T, S)
    --->    empty
    ;       ring(
                datum   :: T,
                nextref :: generic_mutvar(ring(T, S), S)
            ).


:- func new_ring = ring(T, S).

new_ring = empty.


:- pred next_in_ring(ring(T, S), ring(T, S), S,  S ) <= store(S).
:- mode next_in_ring(in,         out,        di, uo) is det.

next_in_ring(empty, empty, !S).

next_in_ring(R @ ring(_, _), Next, !S) :-
    get_mutvar(R ^ nextref, Next, !S).


:- pred insert_next(T,  ring(T, S), ring(T, S), S,  S ) <= store(S).
:- mode insert_next(in, in,         out,        di, uo) is det.

insert_next(X, empty, RB, !S) :-
    new_cyclic_mutvar(func(M) = ring(X, M), SelfRef, !S),
    get_mutvar(SelfRef, RB,                          !S).

insert_next(X, RA @ ring(_, _), RB, RefNext), !S) :-
    get_mutvar(RA ^ nextref, RC, !S),
    new_mutvar(RCRef,        RC, !S),
    RB = ring(X, RCRef),
    set_mutvar(RA ^ nextref, RB, !S).


:- pred remove_next(ring(T, S), ring(T, S), S,  S ) <= store(S).
:- mode remove_next(in,         out,        di, uo) is det.

remove_next(empty, empty, !S).

remove_next(RA @ ring(_, _), RC, !S) :-
    get_mutvar(RA ^ nextref, RB, !S),
    (
        RB = ring(_, _),
        ( if RA = RB then
            RC = empty
          else
            get_mutvar(RB ^ nextref, RC, !S),
            set_mutvar(RA ^ nextref, RC, !S)
        )
    ;
        RB = empty,
        RC = empty
    ).
\end{verbatim}

Observe in particular the code for @insert_next@ and @remove_next@.  In
the latter, if the @ring@ is not empty, we simply change the target of
@RA ^ nextref@ to be the next @ring@ item after that.

In @insert_next@ we have to obtain a \emph{new} @mutvar@ for the new
@ring@ item, @RB@.  We cannot simply copy the @nextref@ field for @RA@
since we subsequently change the target of @RA ^ nextref@ to be @RB@
itself.  \XXX{Give an example to show the subtle difference in, say, C?}

\section{Stores and Concurrency}

Need to use locks to ensure concurrent access.
\XXX{Say more.}

% % vim: ft=tex ff=unix ts=4 sw=4 et wm=8 tw=0

\chapter{* Exceptions}

There are two ways of dealing with error conditions that are detected at
run-time.  The first is for the operation which detected the error to
return an error code of some sort.  This is what happens in the @io@
library with predicates that return an @io.result@.  Return codes
generally have to be dealt with as soon as the operation returns; in
some situations this can mean that the underlying algorithm ends up
hidden beneath a slew of error handling code.

The other option is to \emph{throw an exception} (sometimes called
\emph{raising} an exception).  An exception causes execution to
immediately return to the nearest \emph{exception handler} in the call
stack.  The exception handler must decide what to do from that point on.
Exceptions are useful when the best place to handle an error (or rather,
an exceptional situation) is at some higher level of abstraction -- that
is, further up the call stack -- rather than having to worry about
unlikely events at every point in the program.

Exception based code in Mercury makes use of the modules @exception@ and
@std_util@, the latter because exceptions are ``transmitted'' as values
of type @univ@.

\section{An Example}

\XXX{I use an IO based example.  Perhaps it would be better to use a
non-IO example and afterwards mention @try\_io@ and relatives.}

In this example we want to ``translate'' one file into another.  The
obvious code using error code handling is unfortunately ugly:
\begin{verbatim}
:- import_module io, string, list.

main(!IO) :-
    io.open_input("input_file", OpenInputResult, !IO),
    (
        OpenInputResult = error(_),
        report_error("Couldn't open input_file", !IO)
    ;
        OpenInputResult = ok(Input),
        io.open_output("output_file", OpenOutputResult, !IO),
        (
            OpenOutputResult = error(_),
            io.close_input(Input, !IO),
            report_error("Couldn't open output_file", !IO)
        ;
            OpenOutputResult = ok(Output),
            translate(Input, Output, TranslateResult, !IO),
            io.close_output(Output, !IO),
            io.close_input(Input, !IO),
            (
                TranslateResult = error(_),
                report_error("Something went wrong in translation", !IO)
            ;
                TranslateResult = ok
            )
        )
    ).

:- pred report_error(string, io, io).
:- mode report_error(in,     di, uo) is det.

report_error(ErrMsg, !IO) :-
    io.stderr_stream(StdErr, !IO),
    io.format(, StdErr, "%s\n", [s(ErrMsg)], !IO),
    io.set_exit_status(1, !IO).

:- pred translate(input_stream, output_stream, io.res, io, io).
:- mode translate(in,           in,            out,     di, uo) is det.

translate(Input, Output, Result, !IO) :- ...
\end{verbatim}
We can write more concise code that handles all the possible errors in
one place by using exceptions:
\begin{verbatim}
:- import_module io, string, list, unit, exception, univ.

:- pred main(io, io).
:- mode main(di, uo) is cc_multi.

main(!IO) :-
    try_io(open_files_and_translate, Result, !IO),
    (
        Result = succeeded(_)
    ;
        Result = exception(Exception),

            % Here we use mode univ(out) = in is semidet.
            %
        ( if   Exception = univ(ErrMsg)
          then report_error(ErrMsg, !IO)
          else rethrow(Result)          % These are not the droids
                                        % we're looking for.
        )
    ).

:- pred open_files_and_translate(unit, io, io).
:- mode open_files_and_translate(out,  di, uo) is det.

open_files_and_translate(unit, !IO) :-
    open_in("input_file", Input, !IO),
    open_out("output_file", Input, Output, !IO),
    translate(Input, Output, !IO),
    io.close_output(Output),
    io.close_input(Input).

:- pred open_in(string, input_stream, io, io).
:- mode open_in(in,     in,           di, uo) is det.

open_in(File, Input, !IO) :-
    io.open_input(File, Result, !IO),
    (
        Result = ok(Input)
    ;
        Result = error(_),
        throw("Couldn't open " ++ File)
    ).

:- pred open_out(string, input_stream, output_stream, io, io).
:- mode open_out(in,     in,           out,           di, uo) is det.

open_out(File, Output, !IO) :-
    io.open_output(File, Result, !IO),
    (
        Result = ok(Output)
    ;
        Result = error(_),
        io.close_input(Input),
        throw("Couldn't open " ++ File)
    ).
\end{verbatim}
We can see that the code in the version that uses exceptions has
better structure, although the exception handler (the @try_io@ code) is
forced to use some advanced mechanisms, most notably @univ@ values.  The
code here \emph{is} slightly longer, but that is mainly an artefact of
using a toy example.  In a real application we would expect the
situation to be reversed (otherwise there would be little point in using
exceptions.)

It is important to observe that every exception-throwing site is
responsible for performing clean-up operations that are not dealt with
by the exception handler.  In this case we have to pass the @Input@
stream handle to the @open_out@ predicate so that we can close it if we
fail to successfully open the @Output@ stream (and, similarly, the
@translate@ predicate should close both streams before throwing an
exception.)



\section{Throwing Exceptions}

An exception @X@ is thrown by calling @throw(X)@ from the @exception@
module (@throw/1@ exists in both predicate and function versions and, in
the normal course of things, the compiler will be able to work out from
context which is meant.)

@throw/1@ has determinism @erroneous@, indicating to the compiler that
this predicate does not terminate normally (@erroneous@ can also be used
for predicates that do not return at all), hence it is not necessary to
tie-up all loose ends.  For instance, in
\begin{verbatim}
    p(A, X0, X1),
    (
        A = ok,
        q(X1, X2)
    ;
        A = ouch,
        throw(...)
    ),
    r(X2, X)
\end{verbatim}
we do not have to include the unification @X2 = X1@ in the second
disjunct of the switch, as would be required if @throw/1@ did not have
determinism @erroneous@ since @X2@ is required by the call to @r/2@.

Any value at all can be thrown as an exception; it is up to the
enclosing exception handler to decide if and how to handle matters.

If, in an exception handler, one has to pass the exception back up to an
exception handler even further up the call stack then one should use
@rethrow@.  The reason for this will be explained in the next section
\XXX{}.

\section{Handling Exceptions}

Exception handling works via a call to @try@ or one of its variants:
\begin{verbatim}
    % try(Goal, Result)
    %
:- pred try(pred(T),                exception_result(T)).
:- mode try(pred(out) is det,       out(cannot_fail)) is cc_multi.
:- mode try(pred(out) is semidet,   out             ) is cc_multi.
:- mode try(pred(out) is cc_multi,  out(cannot_fail)) is cc_multi.
:- mode try(pred(out) is cc_nondet, out             ) is cc_multi.
\end{verbatim}
where
\begin{verbatim}
:- type exception_result(T)
    --->    succeeded(T)
    ;       failed
    ;       exception(univ).

:- inst cannot_fail
    --->    succeeded(ground)
    ;       exception(ground).
\end{verbatim}
and @univ@ is the universal type defined in the @std_util@ standard
library module.

What happens operationally with a call @try(Goal, Result)@ is that the
closure @Goal@ is executed and @Result@ instantiated according to what
happened:
if @Goal@ succeeded, returning @X@, then @Result = succeeded(X)@;
if @Goal@ failed then @Result = failed@;
otherwise @Goal@ must have thrown an exception @E@, in which case
@Result = exception(U)@ where @U = univ(E)@ (i.e. the encapsulation of
the abitrary type of @E@ in the universal type @univ@.)

The code that follows the call to @try@ must decide what to do depending
upon @Result@.

If @Result = succeeded(X)@ or @Result = failed@ then we can proceed as
if we had simply called @Goal@.  Note that if @Goal@ has determinism
@det@ or @cc_multi@ then it cannot fail, in which case the argument mode
of @Result@ is @out(cannot_fail)@ which, in turn, means that we do not
need to test for @Result = failed@.

If @Result = exception(U)@ then we need to handle the exception.
Sometimes this can be as simple as performing some clean-up operations
and calling @rethrow(Result)@.  Usually, however, we are interested in
exactly what exception was thrown, which requires a checked, dynamic
(i.e.  run-time) \emph{cast} of @U@ from type @univ@ to the expected
type for @E@.

The @univ@ function in @std_util@ has the following signature:
\begin{verbatim}
:- func univ(T  ) = univ.
:- mode univ(in ) = out is det.
:- mode univ(out) = in is semidet.
\end{verbatim}
Going in the ``forwards'' direction, any value of any type can be
converted in a value of type @univ@, as one might expect.  In the
``reverse'' direction, however, a value of type @univ@ may or may not be
convertible into a value of some given type @T@.

So the code for handling an exception looks like this:
\begin{verbatim}
    try(Goal, Result),
    (
        Result = succeeded(X), ...
    ;
        Result = failed, ...
    ;
        Result = exception(U),
        ( if U = univ(E) then
                % E is the value of the exception in whatever
                % type which deal_with_exception/1 below expects.
                %
            deal_with_exception(E),
            ...
          else
                % Otherwise the exception is of a different type
                % to that which we were prepared to handle, so the
                % exception cannot be intended for us to handle.
                % Therefore we pass it on the the next exception
                % handler up the call stack.
                %
            rethrow(Result)
        )
    )
\end{verbatim}
Note that our if-then-else goal could have several arms if we are
prepared to deal with exceptions of several different types, but we must
still be prepared to @rethrow@ a result which we cannot handle -- simply
dropping such things on the floor means that higher-level exception
handlers cannot do their job.  This is particularly important for
higher order code that deals with exceptions.

\section{Effect On Determinism}

To throw an exception is to effect an abnormal exit from a procedure.
Thanks to the halting problem \XXX{} it is impossible for a compiler to
decide whether an arbitrary piece of code will throw an exception or
even what exceptions it may throw.  The only way to find out in general
is to run the code and see what happens.  As a consequence, the
determinism of @try@ is @cc_multi@: it always returns a result, but we
have no way of knowing, analytically, whether the result will be an
exception or a ``normal'' return.  Either way, we only get the one
result and cannot backtrack into the computation.

\section{Er, Something}

\XXX{There was something I was going to say here\ldots}

\section{IO States and Stores}

The @io@ state and @store@ types play a special role in Mercury in as
much as they are the main exploiters of uniqueness.  \XXX{What a
horrible way to put it.}

The usual exception handling method is not sufficient for handling
computations that manipulate @io@ states and @store@s.  The problem is
that we do not want an exception to cause us to lose the @io@ state or
@store@ -- the former because then our program could no longer perform
any IO at all (and hence not report the results of any computation) and
the latter because we do not want to lose access to the information
in the @store@.

One might imagine that these problems could be remedied by including the
@io@ state or @store@ as part of the exception result -- for example
\begin{verbatim}
:- type excn_with_io ---> excn_with_io(string, io).

p(!IO) :-
    do_something(X, !IO),
    if   something_went_wrong(X)
    then throw(excn_with_io("Aieee!", !.IO))
    else carry_on_as_usual(X, !IO).
\end{verbatim}
The first difficulty here is that the argument to @throw@ has mode @in@,
which means the @io@ state contained in the @excn_with_io@ will not be
seen as @unique@ by the exception handler.

The second objection is that the predicate that throws the exception may
not be in posession of the @io@ state or @store@.  Higher order code is
most likely to have this problem:
\begin{verbatim}
p(P, !IO) :-
    P(X),       % P may throw an exception, but
                % does not have the IO state.
    ...
\end{verbatim}
while we could conceivably place our own exception handler around the
call to @P@, catch any exception thrown by P and wrap that up in
another exception that included the @io@ state, we then have the
problem that any exception handler further up the call stack looking
for exceptions thrown by @P@ will not recognise them because of our
wrapper.  One could probably find a solution to the conundrum, but
it seems unlikely that it would be elegant.

The third problem is that the current version of the Mercury compiler
does not yet handle nested unique objects -- that is, even if we could
place an @io@ state or @store@ inside an exception value \emph{and}
arrange for a mode of @try@ that preserved its @inst@ \emph{and}
came up with a means of transmitting the requisite @inst@
information to the exception handler, we would still have to wait for
a version of the compiler to come along that handled nested @unique@
objects.  \XXX{To our shame!}

Unfortunately, no general solution to all these problems presents
itself -- there remains research to be done.  It should be recognised
that these problems are not specific to Mercury; they exist in all
exception systems that have code that performs destructive update.
The problems are merely highlighted by Mercury's strict type and mode
systems, which the designers do not consider worth compromising for
the sake of a friendlier exception system.  \XXX{Could probably put
that more diplomatically.}

The pragmatic solution currently adopted is to supply two extra
versions of @try@:
\begin{verbatim}
:- pred try_io(pred(T,   io, io),             exception_result(T),
            io, io).
:- mode try_io(pred(out, di, uo) is det,      out(cannot_fail),
            di, uo) is cc_multi.
:- mode try_io(pred(out, di, uo) is cc_multi, out(cannot_fail),
            di, uo) is cc_multi.

:- pred try_store(pred(T,   store, store),       exception_result(T),
            store, store).
:- mode try_store(pred(out, di, uo) is det,      out(cannot_fail),
            di,    uo) is cc_multi.
:- mode try_store(pred(out, di, uo) is cc_multi, out(cannot_fail),
            di,    uo) is cc_multi.
\end{verbatim}
These work very much like @try@, except that in this case the goal
argument takes two extra arguments for the @io@ state or @store@ as
appropriate and that, of course, the @io@ state or @store@ is
preserved across the exception.

So, say we have a @io@ based predicate @parse_input@ that throws
an exception on detecting an error rather than returning an error
code, we could handle it like this:
\begin{verbatim}
:- pred parse_input(parse_result, io, io).
:- mode parse_input(out,          di, uo) is det.
...
    try_io(parse_input, Result, !IO),
    (
        Result = succeeded(Parse),
        io.format("Parsing succeeded\n", [], !IO),
        ...
    ;
        Result = exception(U),
        io.format("Parsing exception\n", [], !IO),
        ( if   U = univ(E)
          then handle_parser_exception(E)
          else rethrow(Result)
        )
    )
...
\end{verbatim}
As pointed out earlier, were it not for @try_io@ returning the @io@
state after catching the exception, we could not make the call to
@io.format@ in the exception handler.

It is important to be aware that it is the programmer's
responsibility to ensure that @store@s and various aspects of the
@io@ state are consistent with the program's logic after catching
an exception.  That is, they will be consistent as far as Mercury
is concerned, but various program-specific invariants concerning
them may or may not be preserved.  This is a difficult problem
and is what makes exception handling less attractive than it would
at first seem.

\section{All Solutions Predicates}

Occasionally one may want to enumerate all the solutions of a
nondeterministic predicate up to the point that all solutions
have been generated or the predicate in question throws an
exception.  This is what @try_all@ is for:
\begin{verbatim}
:- pred try_all(pred(T),              list(exception_result(T))).
:- mode try_all(pred(out) is det,     out(try_all_det))     is cc_multi.
:- mode try_all(pred(out) is semidet, out(try_all_semidet)) is cc_multi.
:- mode try_all(pred(out) is multi,   out(try_all_multi))   is cc_multi.
:- mode try_all(pred(out) is nondet,  out(try_all_nondet))  is cc_multi.
\end{verbatim}
the output argument is a list of the @exception_result@s and the last
element in the list will contain the exception, if any was thrown.
Otherwise, this predicate is very similar to the @builtin@
predicate @unsorted_solutions@.

Just as @unsorted_solutions@ has a companion, @unsorted_aggregate@,
that supports interleaving processing of the results of the
nondeterministic predicate, @try_all@ has a companion,
@incremental_try_all@:
\begin{verbatim}
:- pred incremental_try_all(
            pred(T1),
            pred(exception_result(T1), T2, T2),
            T2, T2).
:- mode incremental_try_all(
            pred(out) is nondet,
            pred(in, di, uo) is det,
            di, uo) is cc_multi.
:- mode incremental_try_all(
            pred(out) is nondet,
            pred(in, in, out) is det,
            in, out) is cc_multi.
\end{verbatim}

\XXX{Do I need more examples in this section?}

\section{The Down Side}

From the discussion above, it should be clear that exceptions are not
silver bullets.  The cases where they \emph{may} simplify your code are
those where there is some distance between the point where an exceptional
case may be detected and the point where it is best handled.  Even then,
one must be careful to ensure that the programs ``state'' (in the sense
of unique values such as @store@s and the @io@ state) are consistent with
the invariants of the program before leaving the exception handler.

In many cases, it turns out to be simpler to return error codes and
handle problems as locally as possible.

As always, it takes experience and a little experimentation to work out
which is the best solution in each particular situation.


% \chapter{* Foreign Language Interface}
\section{Declarations}
\section{Data Types}




% % vim: ft=tex ff=unix ts=4 sw=4 et wm=8 tw=0

\chapter{* Impure Code}
\section{Levels of Purity}
\section{Effect of Impurity Annotations}
\section{Promising Purity (pragma promise\_pure)}




% \section{* Pragmas}
\subsection{Inlining}
\subsection{Type Specialization}
\subsection{Obsolescence}
\subsection{Memoing}
\subsection{...Promises}




% \section{* Debugging}
\subsection{Compiling For Debugging}
\subsection{Basic Tour of the Debugger}
\subsection{Declarative Debugging}




% % vim: ft=tex ff=unix ts=4 sw=4 et wm=8 tw=0

\chapter{* Optimization}
\section{When to Do It and When to Avoid It}
\section{Profiling}
\section{Various Considerations}
\section{An Overview of Contemporary Optimizer Technology}




% \include{RTTI}

\end{document}
