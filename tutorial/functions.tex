\section{Functions}

Functions are @det@ (or @semidet@) relations with (at least) one output.

Essentially, a function is any relation with a single solution for
a given set of inputs.

Functions with single output values are so common that Mercury provides
special syntax to make working with them easier.  One of the key
advantages of functions is that they can be used as parts of
expressions, rather than having to have a separate goal computing each
subexpression in turn.  That is, one can use an expression as in
\begin{verbatim}
    X = (B + sqrt(B*B - 4*A*C)) / (2*A)
\end{verbatim}
rather than the somewhat verbose
\begin{verbatim}
    square(B, BSquared),
    multiply(4, A, FourA),
    multiply(FourA, C, FourAC),
    subtract(BSquared, FourAC, BSquaredMinusFourAC),
    sqrt(BSquaredMinusFourAC, Sqrt),
    add(B, Sqrt, Numerator),
    multiply(2, A, Denominator),
    divide(Numerator, Denominator, X)
\end{verbatim}

\subsection{Definition}

This example illustrates how functions are defined:
\begin{verbatim}
:- func length(list(T)) = int.
:- mode length(in) = out is det.

    % The length of an empty list is 0.
    % The length of a non-empty list is 1 for the head
    % plus the length of the tail.
    %
length([]      ) = 0.
length([_ | Xs]) = 1 + length(Xs).
\end{verbatim}
Like predicates, functions may be defined using several clauses
and make use of pattern matching.

Functions can be computed from goals, where the head and goal
are separated by @:-@ in the definition:
\begin{verbatim}
        % take(N, Xs) is the length min(N, length(Xs))
        % prefix of Xs.
        %
    :- func take(int, list(T)) = list(T).
    :- func take(in, in) = out is det.

    take(N, Xs) = Ys :-
        split(N, Xs, Ys, _).

        % drop(N, Xs) is the length max(0, length(Xs) - N)
        % suffix of Xs.
        %
    :- func drop(int, list(T)) = list(T).
    :- func drop(in, in) = out is det.

    drop(N, Xs) = Zs :-
        split(N, Xs, _, Zs).

        % split(N, Xs, Prefix, Suffix)
        %
    :- pred split(int, list(T), list(T), list(T)).
    :- mode split(in,  in,      out,     out    ) is det.

    split(N, Xs, Ys, Zs) :-
        ( if N > 0, Xs = [X | Xs0] then
            Ys = [X | Ys0],
            split(N - 1, Xs0, Ys0, Zs)
          else
            Ys = [],
            Zs = Xs
        ).
\end{verbatim}
By far the most common mode for a function is
\begin{verbatim}
:- mode f(in, in, ..., in) = out is det.
\end{verbatim}

If a function has (just) this sort of mode, then the mode
declaration can be ommitted and the compiler will simply assume
this mode is what is intended.  Hereafter we will omit unnecessary
mode declarations for functions.

\XXX{What exactly are the constraints on function
determinisms?  Remember to point out (somewhere) that
functions may also have multiple procedures.  See the ref.
manual section on Determinism.}

\subsection{Pattern Matching}

\XXX{Dealt with above.}

\subsection{Recursion}

\XXX{Dealt with above.}

\subsection{Conditional Expressions}

Conditional (if-then-else) expressions look like this:
\begin{verbatim}
    ( if ConditionGoal then YesExpr else NoExpr )
\end{verbatim}
where the usual rules for if-then-elses apply (in particular,
@ConditionGoal@ may bind output variables that are used in
YesExpr, but not elsewhere), except that the then and else
arms are \emph{expressions}, rather than goals.

Note that as with if-then-else goals, the else clause is \emph{not}
optional in a conditional expression (it would make no sense
not to have one.)

\subsection{* Partial (Semi-Deterministic) Functions}

\XXX{Leave for later.}

\subsection{Overview of Semidet Predicates}

\XXX{Deal with this in a later section.  It's sort-of advanced
stuff.}

\subsection{Polymorphism}

\XXX{Dealt with in section on types.}

\subsection{Infix Notation}

Mercury syntax supports a number of prefix, infix and postfix
operators, including all the usual arithmetic operators.  This
is just syntactic sugar, however, and there is no difference
between @X + Y@ and @+(X, Y)@ as far as the compiler is concerned.

If you want to use another name as an infix operator, you can
simply place it in @`@backquotes@`@:
\begin{verbatim}
    X `union` Y `union` Z
\end{verbatim}
is seen by the compiler as
\begin{verbatim}
    union(X, union(Y, Z))
\end{verbatim}
Backquoted symbols bind more tightly than anything else and
associate to the right.



