% vim: ft=tex ff=unix ts=4 sw=4 et wm=8 tw=0

%- Definitions and Customization ----------------------------------------------%

    % \newenvironment{name}[numargs]{begincmds}{endcmds}

    % The verbatim environment insists on adding \parskip before and
    % after each verbatim area which really screws with page layout
    % and is generally one of those incredibly annoying LaTeXisms.
    % God, I hate this poxy thing sometimes.  You wouldn't believe
    % how long it took to find this.
    %
\newenvironment{myverbatim}
{\setlength{\topsep}{0pt}\verbatim}%
{\endverbatim}

    % \newcommand{name}[numargs]{definition}
    %                           -- (default numargs 0)
    %                           -- refer to args as #1, #2 etc. in definition
    %                           -- use % at EOL if definition not finished
    %                           -- space after a cmd is ignored unless
    %                           --  preceded by {}

\newcommand{\eg} {e.g.\@ }
\newcommand{\ie} {i.e.\@ }
\newcommand{\XXX}[1] {{\small\textbf{XXX} \emph{#1}}}

\newcommand{\Aside}[1]%
{{\small{\begin{description}\item{\textbf{Aside:}} #1\end{description}}}}

\newcommand{\BoxedPar}[1]%
{{\center{\fbox{\parbox{\linewidth}{#1}}}}}

\newcommand{\True} {@true@}
\newcommand{\False} {@false@}
\newcommand{\Not} {\neg}
\newcommand{\Conj} {\wedge}
\newcommand{\Disj} {\vee}
\newcommand{\Imp} {\Rightarrow}
\newcommand{\Bimp} {\Leftarrow}
\newcommand{\Eqv} {\Leftrightarrow}
\newcommand{\All}[2] {\forall\ #1.\ #2}
\newcommand{\Some}[2] {\exists\ #1.\ #2}

\newcommand{\Union} {\cup}
\newcommand{\Intersection} {\cap}
\newcommand{\Excluding} {\backslash}

\newcommand{\Odd} {\text{odd}}
\newcommand{\FV} {\text{FV}}
\newcommand{\Wolf} {\text{wolf}}
\newcommand{\Fox} {\text{fox}}
\newcommand{\Bird} {\text{bird}}
\newcommand{\Caterpillar} {\text{caterpillar}}
\newcommand{\Snail} {\text{snail}}
\newcommand{\Animal} {\text{animal}}
\newcommand{\Herbivorous} {\text{herbivorous}}
\newcommand{\Carnivorous} {\text{carnivorous}}
\newcommand{\Eats} {\text{eats}}
\newcommand{\Plant} {\text{plant}}
\newcommand{\Grain} {\text{grain}}
\newcommand{\BiggerThan} {\text{bigger-than}}

\newcommand{\Csharp} {C$\sharp$}
