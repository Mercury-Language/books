\chapter{Preface}

This guide
started life as some notes and slides
aimed at helping YesLogic developers
learn the basic principles of logic programming.
Most of the developers had plenty of experience
in languages such as Haskell and Rust;
the main barrier to learning logic programming was, I think,
a lack of familiarity with much of the jargon,
as well as much of the folklore.
Some help was in order from the developers more experienced in Mercury.

The notes first evolved into an article,
and then into the format it currently takes.
There's a lot more information that could be added,
but publishing it in its current form seems like it would still be of use.
YesLogic has generously agreed to its release.

Some additional topics:
\begin{itemize}
\item
A lot more about types, modes, determinism, and uniqueness.
\item
Modules and abstract data types.
\item
Minimal model semantics.
\item
Many-sorted algebra.
In the current version we just map everything down to first-order,
since part of the argument for declarative programming
is that the models are relatively simple.
\item
User-defined equality and comparison.
\item
Partial instantiatedness.
\item
Bibliographical references.
\item
\ldots
\end{itemize}
There is also room for many more working examples.

Chapter~\ref{sec:non-classical} is not yet written,
though I consider Lee and Harald's work in this area
to be important to draw people's attention to.
Hopefully, this will be able to be addressed in a future version.

\bigskip
\noindent
Mark Brown \\
March 2023
