\chapter{Preface}

This guide
started life as some notes and slides
aimed at helping YesLogic developers
learn the basic principles of logic programming.
Most of the developers had plenty of experience
in languages such as Haskell and Rust;
the main barrier to learning logic programming was, I think,
a lack of familiarity with much of the jargon,
as well as much of the folklore.
Some help was in order from the developers more experienced in Mercury.
This document is intended to form part of that help.

The notes first evolved into an article,
and then into the format it currently takes.
YesLogic generously agreed to the release of the material,
which becaome Version 0.4;
it is currently maintained by the Mercury team.

This current version is already quite useful,
there is still a lot of room for expansion.
We could include more information about some existing topics,
such as types, modes, determinism, and uniqueness,
and we could cover additional topics such as the following.
\begin{itemize}
\item
Modules and abstract data types.
\item
Minimal model semantics.
\item
Many-sorted algebras.
In the current version we just map everything down to first-order,
since part of the argument for declarative programming
is that the models are relatively simple.
\item
User-defined equality and comparison.
\item
Partially instantiated data structures.
\item
\ldots
\end{itemize}
There is also room for more working examples,
and for more bibliographical references.

Chapter~\ref{sec:non-classical} is not yet written,
though I consider it important to draw people's attention
to the work of Lee Naish and Harald Sondergaard in this area.
Hopefully, this will be addressed in a future version.

\bigskip
\noindent
Mark Brown \\
March 2023
