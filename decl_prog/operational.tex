\chapter{Operational semantics}
\label{sec:op-sem}

\section{Overview}

In Chapter~\ref{sec:fopc}
we presented the declarative semantics of Mercury,
and showed how it is possible to use the semantics
to prove theorems about the program,
and in particular about the solutions to formulas.
The deductive system we used to construct these proofs,
although its rules of inference were only hinted at informally,
was essentially the standard one
used with the predicate calculus.

The operational semantics of Mercury
is a deductive system that,
like the standard one,
can be used to generate theorems.
Unlike the standard one,
it is based around a single rule of inference
known as \emph{SLD resolution}.
This rule is able to give a top-down
computational interpretation to a program,
by which we mean that it
defines the sequence of steps by which computation proceeds.

Generally,
computation starts with a goal known as the \emph{query},
and produces zero or more answers in the form of \emph{substitutions}.
We start this chapter by describing queries, substitutions, and unification,
which are the key building blocks of the operational semantics.

We then introduce the SLD resolution inference rule,
and show how \emph{SLD trees} are defined.
These give the operational semantics
of ``definite\label{gi:definite}'' logic programs,
which are programs in which each clause body
is a conjunction of atoms
(that is, not using disjunction, negation, or if-then-else).

Some important results in the meta-theory,
known as soundness and completeness,
will then be covered.
These meta-theorems demonstrate that
the declarative semantics and the operational semantics
give non-conflicting views of the program behaviour.

After this we extend our semantics to deal with Mercury goals in general.
We introduce the negation-as-failure rule used in SLDNF resolution,
which defines the behaviour of if-then-elses and negated goals.
We provide a set of structural rules to deal with
other Mercury goals.

Finally,
we will briefly explain the origin of the phrase ``SLD resolution'',
which may be something that has attracted the reader's curiosity.


\section{Queries}
\label{sec:queries}

Queries are (usually non-ground) goals that represent
the starting point of a computation.
Executing the query involves finding substitutions,
which we refer to as \emph{answers\label{gi:answer}},
for which each ground instance corresponds to
an assignment that makes the goal valid.
A query essentially asks,
``What are the assignments of the free variables
for which this goal is valid?''

The initial query for the
execution of a Mercury program is always
a single call to \sym{main/2}.
It will, however,
also be useful to consider queries that represent
sub-computations within the program.
For definite logic programs,
such queries can be written in the following form:
\begin{verbatim}
    :- Goal1, Goal2, ..., GoalN.
\end{verbatim}
where each \co{GoalI} is an atomic goal,
and commas are read as conjunction.
If the conjunction of goals is in solved form,
as defined in the next section,
then no further computation is required:
the corresponding substitution is the only answer.

The query is interpreted as the formula:%
\footnote{
The origin of this notation,
as with many other things,
comes from theorem provers.
The list of goals on the right-hand side of `\co{:-}'
is taken as a conjunction,
as we have done.
The list of goals of the left-hand side
is taken as a disjunction,
and since in our case the list is empty,
the disjunction is equivalent to \false.
As usual, free variables are implicitly universally quantified.
}
\[
    \forall \bar{x}.\,
        \false \leftarrow \phi_1 \land \ldots \land \phi_n
\]
where $\phi_1, \ldots, \phi_n$
are the formulas corresponding to the goals,
and $\bar{x}$ is the set of free variables
occurring in the goals.
Effectively,
the query is interpreted as the \emph{negation} of the goal,
and the aim is to find all substitutions
that make this negation false.
Hence this is an attempt at proof by contradiction,
similar to our proof from Section~\ref{sec:reasoning}.
That is, a proof, if found, is a refutation of the query,
which in turn implies that the substitution is an answer to the goal.

Proof by contradiction
may seem a circuitous way of doing things,
and indeed some authors
define the execution procedure directly to avoid this,
but we define things this way
because the resulting proof steps, when written out,
have the premise on top and the conclusion underneath.
This is the conventional way of writing proofs,
but the choice ultimately makes no difference to the outcome.


\section{Substitutions}
\label{sec:substitutions}

A \emph{substitution\label{gi:substitution}}
is a partial mapping from variables to data terms,
such that no variable that maps to a term
occurs in any of the terms in the mapping.
That is, variables on the left-hand side
do not occur on the right-hand side.
We write substitutions in the form:
\begin{verbatim}
    {V1 = t1, ..., VN = tN}
\end{verbatim}
where \co{V1} to \co{VN} are variables,
and \co{t1} to \co{tN} are arbitrary data terms
(possibly variables themselves)
that the variables respectively map to.

A substitution applied to an expression \co{t}
yields an expression which is the same as \co{t},
but with each occurrence of a variable
that maps to a term in the substitution
replaced with the mapped term.
A substitution can be applied to a goal in a similar way,
by replacing each free occurrence of a variable
with the term it maps to, if any.

Observe that a substitution without the braces
is just a goal consisting of a conjunction of unifications.
Indeed, substitutions can be thought of as
goals that are in ``solved form''\label{gi:solved-form},
in that applying them once to an expression or goal
is straightforward
and is sufficient to produce the entire effect
(that is, substitutions are idempotent).
The aim of computation is essentially to put goals into
their solved forms.

Substitutions represent the state of computation:
they record all we know about the variables so far.
In Mercury,
the instantiatedness of a set of variables
describes the possible substitutions at that point in the code,
which tells us something about
what form the substitution must take.
If the \co{inst} of a variable is \co{free\label{gi:free2}}
then the variable is not mapped to anything.
If it is \co{bound\label{gi:bound2}} then the variable is mapped to
a term whose principal functor is one of the ones listed,
and whose arguments are described by
the corresponding argument \co{inst}s.
If it is \co{ground\label{gi:ground}} then the variable is mapped to
a term that contains no variables.

While substitutions bear a resemblance to the assignments
that we defined in Section~\ref{sec:assignments},
note that assignments map variables to elements of the universe.
That is, assignments are semantic in nature,
whereas substitutions map variables to data terms,
and are thus syntactic.


\section{Unification}
\label{sec:unification}

Given two possibly non-ground data terms,
unification is the process of finding a substitution on variables
such that applying it to either term yields the same result.
A substitution that makes two terms identical in this way
is called a \emph{unifier} of those terms.

Terms do not always have a unifier,
in which case we say that the terms do not unify.
If they do unify, however,
there is always a ``most general unifier''
that does the least amount of binding possible.
There may be more than one most general unifier,
but they will be unique up to renaming of variables.
The unification algorithm aims to find one such unifier.

For example,
consider the terms \co{f(X,g(X))} and \co{f(h(Y),Z)},
and the substitution:
\begin{verbatim}
    {X = h(Y), Z = g(h(Y))}
\end{verbatim}
Applying this substitution to either of the terms
yields the same result,
namely \co{f(h(Y),g(h(Y)))},
so this substitution is a unifier.
It is not difficult to see that it is a most general unifier.

The unification algorithm can be seen as
a procedure for putting equations into solved form\label{gi:solved-form2},
that is, in the form of a substitution
that is a most general unifier.
Taking the above example, if the query is:
\begin{verbatim}
    :- f(X,g(X)) = f(h(Y),Z).
\end{verbatim}
then in solved form this would be:
\begin{verbatim}
    :- X = h(Y), Z = g(h(Y)).
\end{verbatim}
which corresponds to the substitution we had above.

In general, consider a query
that is a set of equations as follows,
where \co{s1} to \co{sN},
and \co{t1} to \co{tN},
are arbitrary data terms (possibly variables):
\begin{verbatim}
    :- s1 = t1, s2 = t2, ..., sN = tN.
\end{verbatim}
Initially, all of the goals are marked as unsolved.
We first select a goal \co{G} that is not marked as solved.
If there is no such goal then the algorithm terminates.
We then apply one of the following rules,
depending on the form \co{G} takes.
\begin{itemize}
\item
If \co{G} is
\co{X = X}
for some variable \co{X},
remove it.
\item
If \co{G} is
\co{f(s1, ..., sN) = f(t1, ..., tN)}
for data constructor \co{f/N},
remove it and replace it with the set of equations
\co{s1 = t1, ..., sN = tN}.
\item
If \co{G} is
\co{f(s1, ..., sN) = g(t1, ..., tM)}
for distinct data constructors \co{f/N} and \co{g/M},
the algorithm fails.
\item
If \co{G} is
\co{t = X}
for some non-variable data term \co{t}
and variable \co{X},
remove it and replace it with
\co{X = t}.
\item
If \co{G} is
\co{X = t}
where \co{t} is a data term not containing \co{X},
then replace all free occurrences of \co{X}
elsewhere in the query
by \co{t}.
If \co{X} occurred freely in the original query
then keep \co{G} and mark it as solved,
otherwise discard it.
\item
If \co{G} is
\co{X = t}
where \co{t} is a non-variable data term
and \co{X} occurs in \co{t},
the algorithm fails.
\end{itemize}
After applying the appropriate rule
we go back and select another unsolved goal,
or else terminate if there are none.

If the algorithm terminates without failing
then the query will be in solved form,
and the corresponding substitution
will be a most general unifier of the equations.
If the algorithm fails then
the equations do not have a unifier.
Note that the order in which goals are selected does not matter,
since the results will be equivalent irrespective of
the selection order.

A single unification can be solved
as a special case of the above algorithm,
by starting with the set containing just that equation.
For our above example,
three applications of the rules gives us
the equation in solved form:
\begin{center}
\co{f(X,g(X)) = f(h(Y),Z)} \\
$\Downarrow$ \\
\co{X = h(Y), g(X) = Z} \\
$\Downarrow$ \\
\co{X = h(Y), Z = g(X)} \\
$\Downarrow$ \\
\co{X = h(Y), Z = g(h(Y))} \\[1.5em]
\end{center}
Had this query included the goal \co{Y = Z},
the outcome would have been different,
as we would eventually reach the equation \co{Z = g(h(Z))}
and thus we would fail due to the last rule,
which is the occurs check.

This algorithm described here
is originally due to Martelli and Montanari.
We can extend the algorithm
to also allow for function calls in the goals,
not just data terms as we currently do,
however we will need to handle function calls differently
in order to implement
semantic rather than syntactic equality.
We will see how to do that as part of the resolution algorithm,
in the next section.


\section{SLD Resolution}
\label{sec:resolution}

Deductive systems often use
inference rules that follow a pattern of
one introduction and one elimination rule
for each logical symbol.
This provides an elegant understanding of how the logic works,
but for logic programming this view is not particularly useful
since these rules do not provide any computational interpretation---%
they do not say how a program should be executed.
Instead, at least for logic programs not using negation,
there is one main inference rule.
This rule is known as SLD resolution\label{gi:resolution}%
\footnote{
Historically,
resolution was used as an inference technique
in automated theorem provers.
SLD resolution, which is an instance of this technique,
was found to have a useful computational interpretation.
It is from this that logic programming was developed.
}.

Proofs start with a query in the form:
\begin{verbatim}
    :- Goal1, ..., GoalN.
\end{verbatim}
Each of the goals is atomic\label{gi:atomic},
meaning it is either a unification, a predicate call, or a logical constant.
We assume for now that the program clauses are definite,
so the body of each clause
consists of a (possibly empty) conjunction of atomic goals.
The more general case of Mercury clauses
will be addressed in Section~\ref{sec:structure}.

As discussed in Section~\ref{sec:queries},
a query represents an assertion that
the conjunction of goals is false for all variable assignments.
The aim of SLD resolution is to derive a contradiction,
thereby refuting the assertion.
This occurs when the goals are in solved form,
or there are no more goals left in the query.
If such a contradiction is reached then
the substitution corresponding to the solved goal is an answer.

The answer represents an assignment
(or, if free variables remain, a set of assignments)
for which the assertion is refuted,
and therefore for which the conjunction of goals is valid.
We refer to a derivation that results in a contradiction,
and the substitution that it produces,
as a \emph{success\label{gi:success}}.
We refer to the assignment or set of assignments
that the answer\label{gi:answer2} represents
as a \emph{solution\label{gi:solution3}}.

Conversely,
if at any stage the unification procedure fails
then the derivation has reached a tautology.
This means that we have failed to find a refutation,
and we refer to this case as \emph{failure\label{gi:failure}}.

The third possibility is that
neither a contradiction nor a tautology is found.
We refer to this case as \emph{nontermination},
but note that, aside from running forever,
this also includes cases of
abnormal termination such as throwing an exception.

The SLD resolution algorithm is
parameterized by a selection function
that returns a selected goal
based on the current and previous queries.
As before we will mark some goals as solved as we go,
and require the selection function
to choose a goal that is as yet unsolved.
If the selected goal contains any function calls
then the selection function also returns
a selected function call from the goal.

The algorithm proceeds as follows.
If there are no unsolved goals, the derivation succeeds.
Otherwise, select an unsolved goal \co{G}
using the selection function.
Apply one of the following rules,
depending on the form \co{G} takes.
\begin{itemize}
\item
If \co{G} is \co{true},
delete it.
\item
If \co{G} is \co{false},
the derivation fails.
\item
If \co{G} is
a unification between two data terms,
handle it according to the rules
given in the last section
for the unification algorithm.
If that fails then the derivation fails.
\item
If \co{G} is an atomic goal
that contains a function call,
and the selected function call in that goal is
\co{f(t1,...,tN)} for a function \co{f/N},
then choose a clause
whose head takes the form \co{f(s1,...,sN) = sR}.
Rename variables as necessary so
they do not conflict with
any variables already present in the query.
Remove the selected function call
and replace it with \co{sR},
and to the query add
the unifications \co{t1 = s1, ..., tN = sN},
followed by the clause body if present.
\item
If \co{G} is
a predicate call \co{p(t1,...,tN)}
for a predicate \co{p/N},
then choose a clause
whose head takes the form \co{p(s1,...,sN)}.
Rename variables as necessary so
they do not conflict with
any variables already present in the query.
Remove the selected predicate call
and replace it with
the unifications \co{t1 = s1, ..., tN = sN},
followed by the clause body if present.
\end{itemize}
After applying the appropriate rule
we go back and select another unsolved goal.
If there are no such goals,
the derivation succeeds.

The rules we have given are nondeterministic,
in the sense that
the order in which goals and clauses are selected
is not fully specified.
For the unification rules the order does not matter
as the procedure will always lead to the same result,
regardless of the choices made.
The SLD resolution rules, on the other hand,
require a bit more attention.

When it comes to selecting a goal,
Mercury's selection function chooses a goal or function call
whose initial \co{inst}s
are satisfied by the argument terms.
In the strict sequential semantics,
the leftmost goal that satisfies this condition is selected.
In the strict commutative semantics
the selected goal need not be the leftmost one,
but if there are any goals
that were expanded from the body of a clause
then one of the most recently expanded goals is selected.

For predicate and function calls,
the algorithm also requires us to choose a clause from the program.
In contrast to goal selection,
where a single choice is committed to
without going back and trying alternatives,
clause selection results in a ``choice point\label{gi:choice-point}''
being created.
After exploring the derivations that follow from one choice,
if more solutions are sought then
execution \emph{backtracks\label{gi:backtrack}}
to the most recent choice point
and makes a different choice.
If no such choice point exists,
that is, if all choices have previously been explored,
then execution of the query fails\label{gi:failure2}.

In Mercury,
the \co{nondet} determinism category
refers to the second form of nondeterminism above,
namely that relating to clause selection.
Users do not need to declare the first form,
as the compiler is responsible for making the choices in that regard.

The collection of possible derivations arising from a query
can be arranged into a tree,
where derivations branch off at each choice point
in accordance with execution of the query.
A tree of this form is known as an SLD tree\label{gi:sld-tree}.

To illustrate the algorithm,
consider the call \co{append([a,b],[c],Xs)}.
For convenience we repeat the clauses for \co{append/3} here:
\begin{verbatim}
    append([], Bs, Bs).
    append([V | As], Bs, [V | Cs]) :-
        append(As, Bs, Cs).
\end{verbatim}
The initial query is as follows:
\begin{verbatim}
    :- append([a,b], [c], Xs).
\end{verbatim}
If the first clause were chosen
we would get the unification \co{[a,b] = []},
among others,
but this unification would fail.
We therefore need to choose the second clause.

We rename this clause's variables by adding a numerical suffix,
and replace the call with argument unifications
and the renamed clause body.
This results in the following query:
\begin{verbatim}
    :- [a,b] = [V1 | As1], [c] = Bs1, Xs = [V1 | Cs1],
       append(As1, Bs1, Cs1).
\end{verbatim}
After selecting each of the unifications
and applying the unification rules,
we get:
\begin{verbatim}
    :- Xs = [a | Cs1], append([b], [c], Cs1).
\end{verbatim}
If the first clause were chosen
we would get the unification \co{[b] = []},
which would fail,
so again we choose the second clause.
We rename the clause variables with a different suffix
and replace the call as before,
which results in:
\begin{verbatim}
    :- Xs = [a | Cs1], [b] = [V2 | As2], [c] = Bs2,
       Cs1 = [V2 | Cs2], append(As2, Bs2, Cs2).
\end{verbatim}
Running unification rules again, we get:
\begin{verbatim}
    :- Xs = [a,b | Cs2], append([], [c], Cs2).
\end{verbatim}
This time we can choose the first clause,
resulting in:
\begin{verbatim}
    :- Xs = [a,b | Cs2], [] = [], [c] = Bs3, Cs2 = Bs3.
\end{verbatim}
Running unification rules one last time we end up with:
\begin{verbatim}
    :- Xs = [a,b,c].
\end{verbatim}
which is the answer to the query.

Substitutions do not necessarily end up ground,
as happened in this case,
but in typical Mercury usage
the modes will require that they do.
In any case,
for each ground instance of the answer,
the derivation proves that the query formula is valid
under the assignment corresponding to that ground instance.
This motivates a definition to end the section.

\begin{definition}[Provability]
Let $\phi$ be a formula.
We say that $\phi$ is \emph{provable\label{gi:provable}},
written $\Gamma \vdash \phi$,
if there is some goal $G$
with a successful derivation giving substitution $S$,
such that $\phi$ is the formula corresponding to
$G$ with $S$ applied to it.
Furthermore,
if $\Gamma \vdash \phi$
then $\Gamma \vdash \forall x.\, \phi$
for any variable $x$.
\end{definition}


\section{Soundness and completeness}
\label{sec:meta}

In the course of this chapter and the previous one
a number of concepts have been introduced,
some of which are essentially declarative in nature,
others operational.
In many cases the concepts come in pairs,
one corresponding to the declarative view
and the other to the operational view,
which nonetheless reflect the same underlying concept.

\begin{figure}
\begin{center}
\begin{tabular}{rcl}
\bf{Declarative concept} & & \bf{Operational concept} \\[1em]
values & $\quad\longleftrightarrow\quad$ & ground data terms \\
equality & $\quad\longleftrightarrow\quad$ & unification \\
assignment & $\quad\longleftrightarrow\quad$ & substitution \\
solution & $\quad\longleftrightarrow\quad$ & answer \\
truth & $\quad\longleftrightarrow\quad$ & success \\
falsity & $\quad\longleftrightarrow\quad$ & failure \\
existence of model & $\quad\longleftrightarrow\quad$ & consistency \\
validity & $\quad\longleftrightarrow\quad$ & provability
\end{tabular}
\end{center}
\caption{
Correspondences between declarative and operational concepts.
\label{fig:correspondence}
}
\end{figure}

Figure~\ref{fig:correspondence} shows some of the correspondences
between declarative concepts and their operational counterparts.
Of particular importance is that between validity and provability,
as that provides the basis for many of the other equivalences.
The relationship between them
is expressed by the following theorems.

\begin{theorem}[Soundness] \label{thm:soundness}
Let $\phi$ be any formula.
If\: $\Gamma \vdash \phi$ then $\Gamma \models \phi$.
That is, provability implies validity.
\end{theorem}

\begin{theorem}[Completeness] \label{thm:completeness}
Let $\phi$ be any formula.
If\: $\Gamma \models \phi$ then $\Gamma \vdash \phi$.
That is, validity implies provability.
\end{theorem}

\noindent
Between them,
these theorems state that there is an equivalence between
truth as expressed in the model,
and truth as expressed by the program execution.

A deductive system like this,
for which soundness and completeness holds,
is sometimes referred to as a ``full logic\label{gi:full-logic}''.
In this context the declarative view
is referred to as model-theoretic,
while the operational view
is referred to as proof-theoretic.
The equivalence between the model-theoretic and proof-theoretic,
as established by the theorems,
can be expressed by the somewhat cute formula
$\models\; \equiv\;\, \vdash$.

Of practical significance to programmers
is that they can freely switch between
thinking declaratively and thinking operationally.
As we have seen, the former can allow for
much simpler reasoning about programs than the latter.
But, as we have also seen,
this can only go as far as reasoning about partial correctness---%
there will always be situations where
the programmer needs to reason operationally in order to verify correctness.
Being able to freely switch between the two means programmers
can use the declarative semantics most of the time,
but can temporarily switch to the operational semantics when that is required.

It is, therefore,
this correspondence between declarative and operational notions
that justifies the use of the dual semantics of declarative programming,
above and beyond the operational semantics
that programming languages in general provide.
In other words,
we are justified in having two horizontal arrows
in the middle and lower part of Figure~\ref{fig:nutshell}
on page~\pageref{fig:nutshell},
instead of one.

\label{end:op-sem}


\section{Operational incompleteness}
\label{sec:incompleteness}

At first glance,
the Completeness theorem appears to put us in
a kind of programmer's utopia.
All that is required, it seems,
is for the programmer to specify the logical outcomes they want,
and the deductive system will know what to do.

Unfortunately, and perhaps unsurprisingly, this is not the case.
A careful examination of the Completeness theorem shows that
what the theorem states is that, if a formula is valid,
there \emph{exists} some proof that can be reached
via application of the resolution rule.
The resolution rule, however, is not deterministic,
and while execution nominally involves
making all possible clause selection choices eventually,
the compiler still has to commit to a particular clause ordering,
as well as needing to commit to a particular goal selection at each stage.
If the wrong choices are made,
then a nonterminating derivation may be explored
when there is in fact a successful or failed branch
that would have been reached with different choices.

The code in Figure~\ref{fig:incompleteness} illustrates this point.
The predicate \co{p/0} has two clauses,
the first of which immediately loops.
If clause selection chooses this clause first for every call
then the program loops indefinitely without succeeding.
The second clause, however, proves that \sym{p} is valid.
As required by the Completeness theorem,
the proof of this validity does exist---%
execution needs only to select the second clause at some stage---%
but with the above clause selection this proof will never be reached.

\begin{figure}
\begin{center}
\begin{minipage}[t]{9em}
\begin{verbatim}
p :- p.
p.
\end{verbatim}
\end{minipage}
\begin{minipage}[t]{9em}
\begin{verbatim}
q :- q, false.
\end{verbatim}
\end{minipage}
\end{center}
\caption{
Two predicate definitions that illustrate issues with completeness.
\label{fig:incompleteness}
}
\end{figure}

Similarly,
the predicate \co{q/0} has a single clause
whose body is comprised of two conjuncts.
The first conjunct immediately loops,
so if goal selection chooses this goal first for every call
then the program loops indefinitely without failing.
The second conjunct proves that \sym{q} is unsatisfiable,
and similarly to the previous example the proof would have been found
if the second conjunct was selected at some stage.

For the purposes of programming,
this situation is effectively a form of incompleteness,
despite the theorem that says otherwise.
This is why the Mercury Language Reference Manual
talks about implementations being
``at least as complete as'' the strict commutative semantics.
The term ``complete'' here refers to
the form of effective completeness discussed in this section,
rather than that discussed in Section~\ref{sec:meta}.

The concept of completeness is used in many ways in mathematics.
Indeed,
in Section~\ref{sec:partial-correctness}
we referred to clause completeness,
which is another distinct usage of the word.
It is thus helpful, in the context of logic programming,
to explicitly refer to the effective form of completeness
discussed in this section as \emph{operational completeness}.
The reader should be aware, however,
that other authors commonly use the terms unqualified,
which may cause confusion in some cases.


\section{Negation-as-failure}
\label{sec:naf}

So far we have been discussing definite logic programs,
that is, programs without negation.
This means programs without conditionals either,
since they use a form of negation.
In fact conditionals are more fundamental in Mercury,
as `\co{not G}' is a shorthand for
`\co{if G then false else true}'.

In the operational semantics,
the rule for implementing negation is known as negation-as-failure.
The principle is easy to understand:
if a goal succeeds then
the negation of that goal should fail,
and similarly, if a goal fails then
the negation of that goal should succeed.
We can implement negation-as-failure
by adding an extra rule to the SLD rules
from Section~\ref{sec:resolution};
such a system is known as SLDNF resolution.

Assuming that \co{G} is the selected goal,
the additional rule for negation-as-failure
is as follows:
\begin{itemize}
\item
If \co{G} is a conditional goal of the form
\co{if GC then GT else GE},
then execute \co{GC} as a new query.
If the query succeeds with a substitution \co{S},
replace \co{G} with
the result of applying \co{S} to \co{GT}.
If the query fails without having produced an answer,
that is, if all derivations have failed
even after exhausting all possible choices,
replace \co{G} with \co{GE}.
\end{itemize}
Note that if the condition succeeds,
it may leave choice points behind.
These will lead to alternative substitutions
being applied to \co{GT},
leading to different derivations.

It is possible to extend the Soundness theorem\label{gi:soundness2}
to negation-as-failure if and only if
the condition of the if-then-else
does not cause any non-local variables to become instantiated.
That is, resolving the condition
should not lead to any variables
that occur outside of the condition or then-branch
to appear on the left hand side of an equation in the substitution.
From the soundness of negation-as-failure,
combined with our original Completeness theorem,
We can also extend the Completeness theorem\label{gi:completeness2}
to programs with negation.

The requirement to ensure no non-local variables become instantiated
can be challenging to verify manually.
Fortunately,
Mercury's mode system tracks changes to variable instantiation,
so the compiler is able to perform the check at compile time
without manual assistance.


\section{Structural rules}
\label{sec:structure}

We are now in a position to give rules
that cover Mercury's other compound goals.
Assuming that \co{G} is the selected goal,
the additional rules are:

\begin{itemize}
\item
If \co{G} is a disjunction,
choose one of the disjuncts
and replace \co{G} with the chosen disjunct.
As with clause selection,
a choice point is created so that
execution may backtrack to the remaining disjuncts.
\item
If \co{G} is an explicit existential quantification,
the quantified variables are renamed apart in the goal
so that they do not conflict with any variables already in the query.
The existentially quantified goal is then removed
and replaced with the renamed goal.
\item
If \co{G} is defined as an abbreviation for another goal,
it is removed and replaced with that other goal.
Note that this rule is essentially applied at compile-time,
since the replacement occurs as part of desugaring.
\item
If \co{G} is a purity or determinism cast,
it is treated as if the cast was not present
and executed in the same way as other goals.
\item
If \co{G} is a trace goal,
the trace condition is evaluated
(at compile-time or run-time, or both).
If it is true then \co{G} is replaced
by the goal with the trace condition,
otherwise \co{G} is removed.
If an I/O state is passed to the trace goal
then the appropriate substitutions are made.
\item
If \co{G} is an event goal,
it is removed.
Doing so triggers a user-defined debug event
that may be seen in \co{mdb},
the Mercury debugger.
\end{itemize}


\section{What does SLD stand for?}
\label{sec:sld}

The reader may be curious as to what the acronym ``SLD''
in SLD resolution actually means.
Here is an explanation
based on what we have covered in this chapter.

``S'' stands for Selection function.
The resolution procedure is parameterized by
a selection function that dictates which goal to resolve next.

``L'' stands for Linear.
For each step in SLD resolution,
a new query is generated from the previous one only,
without needing to refer to other computations.
As such,
the proof tree consists of a single branch.
This is referred to as a linear proof.

``D'' stands for definite clauses.
SLD resolution applies to logic programs
consisting of definite clauses.

The full resolution rule is given the acronym SLDNF,
where the ``NF'' stands for negation-as-failure.
In the presence of negation the proofs are no longer linear,
since they include sub-computations for negated goals.
Also, obviously, the clauses are no longer definite.
So perhaps SLDNF is a bit of a misnomer
and only really makes sense in historical context,
but nonetheless the name has stuck.
