\chapter{Glossary index}
\label{sec:glossary}

\begin{description}

\item[answer]
(pp.~\pageref{gi:answer}, \pageref{gi:answer2})
The substitution that results from a successful computation.

\item[assignment]
(p.~\pageref{sec:assignments})
A mapping from variables to values in the universe.
(p.~\pageref{gi:assignment})
A unification of the form \co{Y = X},
where one of the sides has mode \co{in}
and the other has mode \co{out}.

\item[atomic goal]
(p.~\pageref{gi:atomic})
A unification, predicate call, or logical constant.

\item[axiom]
(p.~\pageref{gi:axiom})
A closed formula that is taken to be true without proof.

\item[backtrack]
(p.~\pageref{gi:backtrack})
When executing a query,
to jump to the most recent choice point,
if one exists,
in order to resume execution
with a different choice.

\item[bound]
(p.~\pageref{gi:bound})
A variable in a formula that is captured by a quantifier
is said to be bound.
(p.~\pageref{gi:bound2})
A variable that is mapped by a substitution is said to be bound.

\item[choice point]
(p.~\pageref{gi:choice-point})
A point in a computation at which a clause or disjunct selection was made,
and which can be backtracked to in order to execute alternative branches.

\item[clause completeness]
(p.~\pageref{gi:clause-completeness})
Of a predicate or function definition,
having clauses that cover every possible input that has a solution.

\item[clause soundness]
(p.~\pageref{gi:clause-soundness})
Of a clause defining a predicate or function,
producing solutions that are true in the intended interpretation.

\item[closed formula]
(p.~\pageref{gi:closed-formula})
A formula in which there are no free variables.

\item[closure]
(p.~\pageref{gi:closure})
The representation of a lambda term.
It consists of a code pointer,
along with values for the free variables
in the lambda term.

\item[completeness]
(pp.~\pageref{thm:completeness},~\pageref{gi:completeness2})
A deductive system is complete if validity implies provability.
Completeness, in reference to a deductive system,
is not to be confused with operational completeness or clause completeness.

\item[completion]
(p.~\pageref{sec:completion})
Conversion of a set of clauses that define a predicate or function
into a single closed formula.

\item[consistent]
(p.~\pageref{gi:consistent})
A theory is consistent if it does not contain any contradictions.

\item[constant]
(p.~\pageref{gi:constant})
A function that does not take any arguments.

\item[construction]
(p.~\pageref{gi:construction})
A unification of the form \co{Y = f(X1, ..., XN)},
where \co{Y} has mode \co{out}
and each \co{Xi} has either mode \co{in}
or mode \co{unused}.

\item[contingent]
(pp.~\pageref{gi:contingent},~\pageref{gi:contingent2})
True in some model, and false in some other model.

\item[data term]
(p.~\pageref{gi:data-term})
A term that does not contain any function calls.

\item[declarative debugging]
(pp.~\pageref{sec:decl-debug}--\pageref{end:decl-debug})
A debugging algorithm based on
comparing the declarative behaviour of the program
with the intended interpretation.

\item[declarative semantics]
(pp.~\pageref{sec:by-example},~\pageref{def:declarative-semantics})
The collection of models of a Mercury program.

\item[deconstruction]
(p.~\pageref{gi:deconstruction})
A unification of the form \co{Y = f(X1, ..., XN)},
where \co{Y} has mode \co{in}
and each \co{Xi} has either mode \co{out}
or mode \co{unused}.

\item[definite]
(p.~\pageref{gi:definite})
A clause is definite if it is either a fact,
or its body is a conjunction of atomic goals.

\item[failure]
(p.~\pageref{gi:failure})
A derivation that terminates without producing an answer.
(p.~\pageref{gi:failure})
Finishing execution of a query
after exhausting all choice points.

\item[free]
(p.~\pageref{gi:free})
A variable in a formula that is not captured by a quantifier
is said to be free.
(p.~\pageref{gi:free2})
A variable that is not mapped by a substitution is said to be free.

\item[full logic]
(p.~\pageref{gi:full-logic})
A logic equipped with a deductive system that is both sound and complete.

\item[ground]
(p.~\pageref{gi:ground})
A variable that is mapped by a substitution
to a term containing no variables
is said to be ground.
A term that contains no variables is also said to be ground.

\item[Herbrand interpretation]
(p.~\pageref{gi:herbrand-interpretation})
An interpretation where the universe of values
is just the set of ground data terms.

\item[Herbrand universe]
(p.~\pageref{gi:herbrand-universe})
The set of all ground data terms.

\item[intended interpretation]
(p.~\pageref{gi:intended-interpretation})
The interpretation of a specification.

\item[interpretation]
(pp.~\pageref{gi:interpretation},~\pageref{sec:interpretations})
A mapping from syntactic elements to the semantic domain.

\item[logical]
(p.~\pageref{gi:non-logical})
Of a predicate or function,
having a consistent declarative semantics across all calls; pure.
(p.~\pageref{gi:non-logical2})
Pertaining to the symbols that are a fixed part of the language,
in contrast to the predicate and function symbols defined by the program.

\item[missing answer]
(p.~\pageref{gi:missing-answer})
One of the two classes of bugs
that are observable in the declarative semantics.
An answer is missing if it is false according to the program as written,
but true in the intended interpretation.
Also see \emph{wrong answer}.

\item[model]
(p.~\pageref{gi:model})
An interpretation under which a set of axioms are all true.

\item[negation-as-failure]
(p.~\pageref{sec:naf})
An inference rule that says
if a goal succeeds, then the negation of that goal should finitely fail,
and vice versa.

\item[negative free logic]
(p.~\pageref{gi:nfl})
A logic that permits non-denoting terms,
and treats atomic formulas containing non-denoting terms as being false.

\item[non-denoting term]
(p.~\pageref{gi:non-denoting})
A term that contains a call to a \co{semidet} function,
with arguments for which the function fails.

\item[non-logical]
(p.~\pageref{gi:non-logical})
Of a predicate or function,
not having a consistent declarative semantics across all calls; impure.
(p.~\pageref{gi:non-logical2})
Pertaining to the predicate and function symbols defined by the program,
in contrast to the symbols that are a fixed part of the language.

\item[occurs check]
(p.~\pageref{gi:occurs-check})
A check that a variable is not bound to a term that contains that variable.
Also know as an occur check.

\item[operational completeness]
(p.~\pageref{sec:incompleteness})
The extent to which an operational semantics
is able to avoid unnecessary nontermination.

\item[operational semantics]
(pp.~\pageref{sec:op-sem}--\pageref{end:op-sem})
The computation defined by a program.

\item[partial correctness]
(p.~\pageref{gi:partial-correctness})
Correctness according to the declarative semantics.
It does not consider issues such as computational complexity.

\item[provable]
(p.~\pageref{gi:provable})
A closed formula for which there exists a successful derivation
is said to be provable.
Can be written as $\Gamma \vdash \phi$.

\item[pure]
(p.~\pageref{gi:pure})
Of a predicate or function,
having a consistent declarative interpretation across all calls,
regardless of the mode in which the call is made.

\item[query]
(p.~\pageref{sec:queries})
The initial goal for a computation.

\item[relativization]
(p.~\pageref{gi:relativization})
Adding explicit type checks that cause failure,
to an otherwise untyped logic program.

\item[resolution]
(p.~\pageref{gi:resolution})
A type of inference rule.
Logic programming uses SLD resolution
as its primary mechanism of computation.

\item[satisfiable]
(p.~\pageref{gi:satisfiable})
A model satisfies a closed formula if it maps the formula to \textit{true}.
If a model exists that satisfies a formula,
that formula is said to be satisfiable.

\item[semantic equality]
(p.~\pageref{gi:semantic-equality})
The relation between terms that holds
if and only if both terms denote the same value.

\item[SLD tree]
(p.~\pageref{gi:sld-tree})
A tree formed from SLD derivations by including choice point nodes,
where the derivations branching off from that choice point
are the child nodes.

\item[solution]
(pp.~\pageref{gi:solution},~\pageref{gi:solution2},~\pageref{gi:solution3})
An assignment or set of assignments under which, in every model,
a formula is true.

\item[solved form]
(pp.~\pageref{gi:solved-form},~\pageref{gi:solved-form2})
A goal that takes the form of a substitution.

\item[soundness]
(pp.~\pageref{thm:soundness},~\pageref{gi:soundness2})
A deductive system is sound if provability implies validity.

\item[substitution]
(p.~\pageref{gi:substitution})
A partial mapping from variables to terms,
such that no variable that maps to a term
occurs in any of the terms in the mapping.

\item[success]
(p.~\pageref{gi:success})
A derivation that produces an answer.

\item[test]
(p.~\pageref{gi:test})
A unification of the form \co{Y = X},
where both sides have mode \co{in}.

\item[theorem]
(p.~\pageref{gi:theorem})
A closed formula that is provable from a set of axioms.
The collection of all such closed formulas is known as a theory.

\item[unification]
(p.~\pageref{sec:unification})
The process of finding the most general substitution
that makes two terms identical.

\item[Unique Names Assumption]
(p.~\pageref{gi:una})
The assumption that two ground data terms are equal
only if they are syntactically identical.

\item[unsatisfiable]
(p.~\pageref{gi:unsatisfiable})
False in all models.

\item[valid]
(p.~\pageref{gi:valid})
True in all models.

\item[value]
(p.~\pageref{gi:value})
An element of the semantic universe.
Values are denoted by data terms.

\item[wrong answer]
(p.~\pageref{gi:wrong-answer})
One of the two classes of bugs
that are observable in the declarative semantics.
An answer is wrong if it is true according to the program as written,
but false in the intended interpretation.
Also see \emph{missing answer}.

\end{description}
