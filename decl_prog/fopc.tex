\chapter{First-order predicate calculus}
\label{sec:fopc}

\section{Overview}

In this chapter we introduce first-order predicate calculus,
also known as first-order logic, or classical logic.
This mathematical notion
forms the basis of Mercury's declarative semantics,
via the translation of Mercury code
into axioms of a predicate calculus theory.

Articles on first-order predicate calculus
typically define a syntax,
then introduce a deductive system that allows proofs to be constructed
from inference rules,
then define a semantics in terms of ``interpretations'' and ``models''
and prove some results in the meta-theory.
We will take a similar approach,
however we present the semantics,
with a particular focus on how (first-order) Mercury programs
are converted to logic formulas,
\emph{before} we cover the deductive system.
In the next chapter we will present the deductive system,
which gives a computational interpretation to the logic.

Mercury supports higher-order code,
so a first-order description will not directly cover all of it.
First-order logic is also untyped.
Despite these differences,
we can reduce Mercury's declarative semantics to first-order logic,
which is conceptually much simpler than the alternatives---%
first-order theories have relatively simple models,
such as the Herbrand interpretations from the previous chapter---%
so that is the approach we will take in this guide.
We will discuss types and higher-order code,
plus some other extensions of the basic logic,
in Chapter~\ref{sec:extensions}.

In the remainder of this chapter
we give the syntax of predicate calculus,
and show how basic Mercury constructs generate formulas,
and how Mercury programs generate axioms.
We then give a formal definition of the classical semantics
that results from those axioms.
Some examples of ad hoc proofs based on the semantics are provided,
to help illustrate the concepts introduced,
and to motivate the operational semantics
that we define in detail in the next chapter.
We conclude the chapter with some philosophical remarks about
the role of classical logic in understanding computer programs.


\section{Syntax}
\label{sec:syntax}

The syntax of first-order logic is given in terms of a set of symbols,
along with rules to say how they can be put together
to make well-formed terms and formulas.
The symbols we use are listed in Figure~\ref{fig:symbols}.

\begin{figure}
\begin{enumerate}
\item
An infinite set of variables:
$a$, $b$, $x$, $y$, $z$, \ldots{}
\item
Logical constants:
\true, \false
\item
Logical connectives:
$\land$, $\lor$, $\lnot$, $\leftarrow$, $\rightarrow$,
$\leftrightarrow$, \ldots
\item
Quantifiers:
$\forall$, $\exists$
\item
Logical relations: $=$
\item
Punctuation: we will use parentheses and commas in the usual way.
\item
A set of function symbols
\sym{f}, \sym{g}, \ldots,
each of which has a non-negative arity,
which we denote as, for example,
\sym{f/2}, \sym{g/1}, \ldots
\item
A set of predicate symbols
\sym{p}, \sym{q}, \ldots,
each of which has a non-negative arity,
which we denote as, for example,
\sym{p/2}, \sym{q/3}, \ldots
\end{enumerate}
\caption{Symbols used in predicate calculus.\label{fig:symbols}}
\end{figure}

Some sources refer to the symbols in categories 1 to 6
as \emph{logical},
while referring to the symbols in categories 7 and 8 as
\emph{non-logical\label{gi:non-logical2}}.
It is important to note that
this term is being used in a different sense
to that in Section~\ref{sec:purity}.
In this context it is referring to whether symbols are
defined as part of the logic itself,
or defined as part of a particular theory.
To put it in programming terms,
``logical'' here means that
the symbol is a fixed part of the language,
while ``non-logical'' really just means user-defined.

By convention,
we will use italics such as $a$, $b$, $x$, $y$ and $z$,
possibly with subscripts,
to stand for arbitrary variables.
A sequence of variables may be written with an overbar,
as in $\bar{x}$.
The logical constants, logical connectives and quantifiers
have their usual meanings,
and we will write terms and formulas using
the following rules of operator precedence:
\begin{itemize}
\item
The equals operator, `=',
binds more tightly than any logical connective.
\item
The negation operator, `$\lnot$',
binds more tightly than any of the other logical connectives.
\item
Conjunction, `$\land$', binds next most tightly after negation.
\item
Disjunction, `$\lor$, binds next most tightly after conjunction.
\item
The remaining operators,
for implication and bi-implication,
bind the least tightly.
\end{itemize}
We use parentheses to override these rules in the usual manner;
parentheses are always used for disambiguation
in the case of implications and bi-implications.
The scope of a quantifier extends as far to the right as possible;
parentheses will be used if the scope needs to be limited.

The user-defined symbols in the semantics of a Mercury program
come from the declarations in that program.
Function symbols in the semantics include
data constructors declared in discriminated union types,
as well as declared functions.
Predicate symbols come from declared predicates.
We will use names such as \sym{f} and \sym{g}
to stand for arbitrary function symbols,
and names such as \sym{p} and \sym{q}
to stand for arbitrary predicate symbols.
We will sometimes refer to function symbols with arity zero
as \emph{constants\label{gi:constant}},
and use names such as \sym{c} and \sym{d}
to stand for arbitrary constants.

The arity of a predicate or function symbol
is listed after a slash, for example \sym{f/2} or \sym{p/1}.
However,
we will sometimes leave the arity off entirely
when it is zero, and/or when the arity is clear from the context.
Note that the arity is part of each symbol's identity:
two symbols with the same name but different arities
are considered different symbols.

\begin{figure}
\begin{center}
\begin{tabular}{lll@{\hspace{3em}}l}
\multicolumn{4}{l}{Terms:} \\
$\qquad t$ & $::=$ & $x$ & variable \\
& $\:|$ & $\ct{a}$ & constant \\
& $\:|$ & $\ct{f}(t_1, \ldots, t_n)$ & compound term \\[1em]
\multicolumn{4}{l}{Formulas:} \\
$\qquad\phi$ & $::=$ & \true & always true \\
& $\:|$ & \false & always false \\
& $\:|$ & $t_1 = t_2$ & equation \\
& $\:|$ & $\ct{p}(t_1, \ldots, t_n)$ & predicate call \\
& $\:|$ & $\phi_1 \land \ldots \land \phi_n$ & conjunction \\
& $\:|$ & $\phi_1 \lor \ldots \lor \phi_n$ & disjunction \\
& $\:|$ & $\lnot \phi_1$ & negation \\
& $\:|$ & $\forall x.\, \phi_1$ & universal quantification \\
& $\:|$ & $\exists x.\, \phi_1$ & existential quantification
\end{tabular}
\end{center}
\caption{The grammar rules for terms and formulas.\label{fig:grammar}}
\end{figure}

The grammar rules for the predicate calculus
are shown in Figure~\ref{fig:grammar}.
Terms are constructed from variables and function symbols
in essentially the same way that expressions are in Mercury.
We will refer to arbitrary terms using names such as $s$ and $t$.
Terms that are constructed using only variables and data constructors
(that is, with no function calls)
play an important role in our discussion;
we will refer to these as \emph{data terms}.
A data term that contains no variables
we will refer to as a \emph{ground data term}.

We will assume the existence of a constant for each integer,
and the existence of two function symbols,
$[|]/2$ and $[]/0$,
representing the data constructors
for nonempty lists and the empty list respectively.
We will write list terms in an analogous way to Mercury list syntax,
for example,
we will write
$[1, 2, 3]$ as a shorthand for $[|](1, [|](2, [|](3, [])))$.

Formulas are either atomic or compound.
Atomic formulas (atoms\label{gi:atom2}) are either
logical constants, equations, or predicate calls.
Compound formulas are constructed from atoms
using connectives and quantifiers,
in essentially the same way that goals are in Mercury.
We will refer to arbitrary formulas
using names such as $\phi$ and $\psi$.

Quantifiers with multiple variables stand for
the same quantifier with each variable in turn,
that is, $\forall x y.\phi$
is an abbreviation for $\forall x.\forall y.\phi$.
A variable occuring in a formula is \emph{bound\label{gi:bound}}
if it is captured by a quantifier,
otherwise it is \emph{free\label{gi:free}}.
For example,
$x$ is bound and $y$ is free
in the formula $\forall x. \ct{f}(x, y) = y$.
(Despite the shared nomenclature,
these notions are different from the notions of bound and free
in Mercury's mode system.)

A formula that has no free variables
is said to be \emph{closed\label{gi:closed-formula}};
some sources refer to closed formulas as ``sentences''.
The universal closure of a formula is that formula with
all of its free variables universally quantified.


\section{Expressions and goals}
\label{sec:goals}

One of the building blocks
for understanding Mercury's declarative semantics
is to see how Mercury expressions are mapped to terms,
and how Mercury goals are mapped to formulas.
The relationship we want to describe here is
the one represented by the lower left vertical arrow
in Figure~\ref{fig:nutshell} on page~\pageref{fig:nutshell}.

The mappings for expressions and goals
are provided in full detail in the reference manual,
but for convenience
Figure~\ref{fig:goals} summarizes the most important bits.
In line with our notational convention,
$\phi$ is the formula that corresponds to goals such as \co{Goal}.
For goals that have more than one subgoal,
they are distinguished by suffixes in the expressions,
and by the same suffixes being used as subscripts in formulae,
so that \co{Goal1} maps to $\phi_1$, and so on.

\begin{figure}
\begin{center}
\begin{tabular}{l@{\hspace{3em}}l}
\multicolumn{2}{l}{Expressions $\Rightarrow$ Terms:} \\
\qquad\co{X} & $x$ \\
\qquad\co{c} & $\ct{c}$ \\
\qquad\co{f(t1, ..., tN)} & $\ct{f}(t_1, \ldots, t_n)$ \\[1em]

\multicolumn{2}{l}{Goals $\Rightarrow$ Formulas:} \\
\qquad\co{true} & \true \\
\qquad\co{false} & \false \\
\qquad\co{t1 = t2} & $t_1 = t_2$ \\
\qquad\co{p(t1, ..., tN)} & $\ct{p}(t_1, \ldots, t_n)$ \\
\qquad\co{not Goal} & $\lnot \phi$ \\
\qquad\co{Goal1, ..., GoalN} & $\phi_1 \land \ldots \land \phi_n$ \\
\qquad\co{( Goal1 ; ... ; GoalN )} & $\phi_1 \lor \ldots \lor \phi_n$ \\
\qquad\co{( if C then T else E )}
    & $(\phi_c \land \phi_t) \lor (\lnot \phi_c \land \phi_e)$ \\
\qquad\co{some [X] Goal} & $\exists x.\, \phi$ \\
\qquad\co{all [X] Goal} & $\forall x.\, \phi$ \\
\qquad\co{Goal1 => Goal2} & $\phi_1 \rightarrow \phi_2$ \\
\qquad\co{Goal1 <= Goal2} & $\phi_1 \leftarrow \phi_2$ \\
\qquad\co{Goal1 <=> Goal2} & $\phi_1 \leftrightarrow \phi_2$ \\[.5em]
\multicolumn{2}{l}{where $\phi$ is the formula corresponding to \co{Goal}}
\end{tabular}
\end{center}
\caption{Semantics of Mercury expressions and goals.\label{fig:goals}}
\end{figure}

% XXX for zs:
% I think the material in this section, and much of what is in
% the next section, should go into the operational semantics
% chapter. By that stage we have given a neat characterization
% of the instantiatedness of variables (it can be read off from
% the part of the resolvent that is in solved form), so it
% makes sense there to talk about how we translate these goals
% in such a way as to be classically valid as well as meaningful
% in terms of binding patterns.
%
% The remainder of this section (in this chapter) could have the
% member example just in mathematical form. So we'd point out
% that \forall x. member(x, y) \rightarrow property(x)
% means that every member of y has the property.
%
% The implicit quantification section in this chapter could just
% talk about the principle of least surprise. For example, that
% not p(x) means that there is no value of x for which p holds,
% rather than there being some value of x for which p doesn't hold.
% Likewise, ( if p(X) then q(X) else r ) maps to a formula that means
% either there is some X such that p(X) and q(X), or there is no X
% such that p(X), and r. Alternatively, if X is non-local, it means
% there is some X determined by the context such that either p(X) and
% q(X), or not p(X) and r. (So the "some" and "either" are reversed
% if the variable is non-local.)
% The implicit quantification rules are just implementing this.
%
% The translations can be given their own section in the next chapter,
% perhaps just after negation as failure. In these two sections
% (naf and the translations) we should be able to state that the way
% we handle bindings does implement the implicitly quantified formulas
% soundly.

Some kinds of goals
the Mercury compiler implements
by reducing to other kinds of goals.
This needs to be done with some care,
as operationally Mercury needs to be able to
make sense of the resulting goals in terms of
constructively binding variables.
Goals in the last four rows of this table,
unidirectional implications,
bidirectional implications,
and universal quantification,
are handled in this way.

In classical logic,
the semantics of unidirectional implication,
in both its forward \co{Goal1 => Goal2} form
and in its equivalent backward form \co{Goal2 <= Goal1},
is usually given as $(\lnot \phi_1) \lor \phi_2$.
This short formula is adequate for the purposes of classical logic,
because, when given a ground instance of this goal,
it can tell you whether it is true or false in a given interpretation.
However, it is inadequate for a programming language,
whose concern is not judging whether a ground instance of a formula is true,
but \emph{finding} the true ground instances,
or, in other words,
\emph{computing} the values of the variables that make the formula true.
For this purpose,
viewing \co{Goal1 => Goal2} as $(\lnot \phi_1) \lor \phi_2$ is limiting,
because the fact that $\phi_1$ is negated
does not allow variable bindings to flow from $\phi_1$ to $\phi_2$.
Mercury therefore views the semantics of \co{Goal1 => Goal2}
as being $(\lnot (\phi_1 \land (\lnot \phi_2))$.
This formula \emph{does} allow variable bindings
to flow from $\phi_1$ to $\phi_2$.

The formulas
$(\lnot \phi_1) \lor \phi_2$ and
$\lnot (\phi_1 \land (\lnot \phi_2))$
are equivalent in classical logic,
which is concerned with truth,
but they are \emph{not} equivalent in \emph{intuitionistic logic}
(sometimes called \emph{constructive logic}),
which is concerned with computability as well as truth.
The main difference between the two logics is that
while both accept proofs of the form
``there exists an entity with this given property,
and here is how you construct that entity'',
only classical logic accepts proofs of the form
``there must exist an entity with this given property,
because assuming its nonexistence leads to a contradiction,
but we don't have a method of constructing it''.
There are therefore propositions
that classical logic would say are true
for which intuitionistic logic cannot say whether they are true or false.

Just as $(\lnot \phi_1) \lor \phi_2$ is the obvious formula
for \co{Goal1 => Goal2},
$\forall x.\, \phi$ is the obvious formula for \co{all [X] Goal}.
However, this formula does not work in a programming language either,
because it does not provide a useful way
to enumerate the values of \co{X}
for which \co{Goal} should be checked to be true.
(In theory, we could try to check whether
\co{Goal} is true for all values in the type of \co{X},
but even types as simple as ``list of integers''
have an infinite number of values,
so that does not work.)
Mercury therefore views the semantics of \co{all [X] Goal}
as being $\lnot (\exists x.\, \lnot \phi)$.
This formula is equivalent to $\forall x.\, \phi$ in classical logic,
but it allows code within $\phi$ itself
to act the generator of the values of $x$.
For example, in the frequent case
where the goal inside the universal quantification is an implication,
such as
\begin{verbatim}
    all [Item] (
        set.member(Item, Set) => property(Item)
    )
\end{verbatim}
the goal calling \co{set.member}
will return all the items in \co{Set} one at a time as Item
(\co{Set} having been bound by earlier code),
until either
\begin{itemize}
\item
one of the items fails the property test
(in which case the universal quantification goal fails), or
\item
there are no more items left to return
(in which case the universal quantification goal succeeds).
\end{itemize}

In cases where the goal inside the universal quantification
is a bidirectional implication,
both subgoals of that implication
must be able to generate values for the quantified variable.
We will discuss the semantics of bidirectional implications
at the end of the next section.


\section{Implicit quantification}
\label{sec:implicit-quantification}

The usual rule in mathematics
is that free variables in a formula
are implicitly universally quantified across the whole formula.
(For example,
in a mathematical identity such as
$\sin^2\theta + \cos^2\theta = 1$,
the equation is meant to be taken as true
for \emph{all} values of $\theta$.)
This is not quite what we want for Mercury's semantics,
because in Mercury code there are common situations
in which universal quantification over the entire clause
clashes with what most programmers expect.
To avoid violating the ``law of least astonishment'',
Mercury therefore quantifies some variables
differently to the usual mathematical rule.

In first-order code,
the most important exception consists of variables
that are intended to be bound by the condition of an if-then-else.
The values of such variables are available
in the rest of the condition (the part after the goal that bound them)
and in the then-part of the if-then-else, but
\begin{itemize}
\item
they are not available in the else-part of that if-then-else
because that code is executed only if the condition fails,
and failed computations cannot generate any bindings; and
\item
they are not available in any part of the clause outside of that if-then-else,
because execution may get after taking the else branch in that if-then-else,
which again means that the variable binding won't have taken place.
\end{itemize}

This means that variables that
the programmer intends to be bound by the condition of an if-then-else
may occur in that condition and in the then-part that follows it,
but nowhere else in the clause.
Mercury implicitly quantifies an if-then-else \co{( if C then T else E )}
where $\bar{x}$ is the set of the variable for which this is true
like this:
\[
    (\exists \bar{x}.\, \phi_c \land \phi_t) \lor
    (\lnot (\exists \bar{x}.\, \phi_c) \land \phi_e)
\]
Note that
\begin{itemize}
\item
the variables in $\bar{x}$ are quantified existentially
and in these scopes derived from just the if-then-else goal,
rather than universally and over the scope of the entire clause, and
\item
the first quantifier's scope includes the then-branch goal
but the second quantifier's scope does not include the else-branch goal,
reflecting the fact that such variables cannot be used there.
\end{itemize}

This implicit introduction of existential quantifiers
interacts with the semantics of bidirectional implications
(or bi-implications for short).
The obvious semantics of \co{Goal1 <=> Goal2}
is the conjunction of the formulas
for \co{Goal1 => Goal2} and \co{Goal2 => Goal1},
which for Mercury is
\[
    (\lnot (\phi_1 \land (\lnot \phi_2))) \land
    (\lnot (\phi_2 \land (\lnot \phi_1)))
\]
However, this does have the effect of duplicating both $\phi_1$ and $\phi_2$.
If either $\phi_1$ or $\phi_2$ (or both) contain if-then-elses,
then any variable that \emph{used} to occur
only in the condition and the then-part of such an if-then-else,
will now occur outside of that if-then-else:
specifically, it will occur in the other copy of that if-then-else.
This is why the Mercury compiler expands out bi-implications
(in other words, it replaces \co{Goal1 <=> Goal2}
with the conjunction of \co{Goal1 => Goal2} and \co{Goal2 => Goal1})
only \emph{after} it has inserted all needed existential quantifiers
into if-then-elses.
Bi-implications are the only Mercury constructs
whose implementation in terms of other Mercury constructs
can duplicate subgoals,
so other Mercury constructs that are implemented this way
(such as unidirectional implications)
are expanded out much earlier.
% XXX Is the info added by that last sentence
% useful to readers of this guide?


\section{Clauses}
\label{sec:clauses}

To give a semantics to clauses,
we can consider a mapping that just
replaces \co{:-} with reverse implication.
Figure~\ref{fig:clauses} shows the effect this has
on the different forms of clauses---%
the formulas that result are implications
in which the body implies the head,
which is the interpretation discussed in Section~\ref{sec:partial-correctness}.
This essentially lines up with clause soundness,
in that if one of these implications is incorrect
it will lead to a wrong answer bug.

\begin{figure}
\begin{center}
\begin{tabular}{l@{\hspace{3em}}l}
\multicolumn{2}{l}{Clauses $\Rightarrow$ Formulas:} \\
\qquad\co{p(t1, ..., tN) :- Body}
    & $\ct{p}(t_1, \ldots, t_n) \leftarrow \phi$ \\
\qquad\co{p(t1, ..., tN)}
    & $\ct{p}(t_1, \ldots, t_n) \leftarrow \true$ \\
\qquad\co{f(t1, ..., tN) = t :- Body}
    & $\ct{f}(t_1, \ldots, t_n) = t \leftarrow \phi$ \\
\qquad\co{f(t1, ..., tN) = t}
    & $\ct{f}(t_1, \ldots, t_n) = t \leftarrow \true$ \\[.5em]
\multicolumn{2}{l}
    {where $\phi$ is the formula corresponding to \co{Body}, if present}
\end{tabular}
\end{center}
\caption{Mercury clauses as reverse implicatios.\label{fig:clauses}}
\end{figure}

Two problems still remain with this formulation of clauses.
First,
the resulting formulas are not necessarily closed.
We will see shortly that we require
the formulas that go into our semantics to be closed.
Second,
while we have made the connection to clause soundness,
we have not yet done so for clause completeness.
That will require looking at
all clauses that define a predicate or function,
instead of just one at a time.

Before we continue our discussion of clauses, however,
we will first need to take a closer look at equality
as it relates to the predicate calculus.
While the topic is usually taken as basic in logic,
there are slightly different approaches
and it is worth looking closely at the choices we have made.


\section{First-order logic with equality}
\label{sec:sem-equality}

In our syntax we included equality
as one of the logical symbols.
We define this as \emph{semantic} equality\label{gi:semantic-equality}:
two terms are equal if and only if they denote the same thing.
Thus, by definition, the relation is
reflexive, symmetric, and transitive,
as would be expected.
A substitutivity principle also applies,
in that if any argument to a function
is replaced by another argument that is equal according to our definition,
then the result is equal.
Similarly, if any argument to a predicate
is replaced by another equal argument,
then the result is logically equivalent.

Another possible definition of equality,
which we do not adopt,
is \emph{syntactic} equality.
This means that two terms are considered equal
if they are syntactically identical,
or if they can be made so via a variable substitution.
The two definitions are very similar,
and produce the same results in most cases.
There are subtle differences, however,
and these will eventually become relevant
when we talk about \co{semidet} functions
in Section~\ref{sec:partial},
so it important to take note of
precisely which definition we are using.

When the equality relation is defined as part of the logic,
as we have done in this guide,
the logic is sometimes called
\emph{first-order logic with equality}.
Some authors take an alternative approach where
the equality relation is treated
no differently to other predicate symbols.
This alternative approach is sometimes used in order to
define equality in terms of some other relation,
for example by saying that
two sets are equal if each is a subset of the other.
In typical cases, however,
equality is taken to be part of the logic,
and the term ``first-order logic''
is usually taken to mean first-order logic with equality.

We have chosen to include equality as part of the logic
since it is simpler to do so,
and we do intend the aforementioned properties to hold.
This does not capture our full intent, however,
and we will need to add to our definition in the next section.


\section{Axioms}
\label{sec:axioms}

\subsection{What are axioms?}
\label{gi:axiom}

Axioms are closed formulas that are taken to be true without proof,
meaning they may be used as a starting point for
proofs of other formulas.
We will look more at proofs in the next chapter.
A closed formula that can be proven from a set of axioms
is called a \emph{theorem\label{gi:theorem}},
and the collection of all such closed formulas
is known as a \emph{theory}.
Thus, the axioms form the basis of
a particular predicate calculus theory.

In this section we will show how to generate a set of axioms
from the declarations and clauses in a Mercury program.
The theory that results will ultimately determine
the declarative semantics of the program.
We will make this concept clearer
once we have discussed models,
which we will do in Section~\ref{sec:classical}.

Note that in some sources on this topic
the authors' aim is to axiomatize the logic itself---%
that is, include axioms that define the logical symbols---%
but we take the approach of defining the logic symbols directly
and just using axioms to define what is specific to a program.
This is because we want to state something \emph{using} the logic,
rather than explore theorems \emph{about} the logic.
Thus, when we refer to ``axioms''
we mean those that are generated from the program,
rather than the sort that define the logic being used.


\subsection{Equality axioms}
\label{sec:ax-equality}

Our definition of equality thus far
gives us some conditions that imply when two given terms are equal.
It does not, however, say when they are not equal.
As it stands,
we are not even able to rule out the case where
\emph{all} terms are equal.
To repair this,
we will need to add some axioms relating to equality.
Our intent is that two ground data terms\label{gi:data-term}
(that is, terms that do not include function calls)
should be equal only if they are syntactically identical.

Note that we are saying \emph{only if}.
In other words,
equality of ground data terms implies that
they must be syntactically identical.
This is not the same as saying
syntactic identity of ground data terms
implies that they must be equal,
since the implication is in the opposite direction.
If the implication were this way around
we would be defining equality as syntactic,
but, as mentioned in Section~\ref{sec:sem-equality},
that is not the definition of equality we have adopted.

The condition that ground terms are equal only if identical
is known as the \emph{Unique Names Assumption\label{gi:una}}.
As the name suggests, this is often left implicit,
particularly when discussing logic programming or databases.
In other cases, such as when dealing with ontologies,
it can be desirable to have two distinct names denote the same thing,
in which case the assumption is not made.
In our case,
there are of course terms containing function calls
that are intended to be equal even if they are not syntactically identical.
For example $2 + 2$ is equal to $1 + 3$
since both sides are equal to 4,
so we do not want this assumption to always apply.
We do intend that this assumption holds for ground data terms, however.

Another assumption we wish to make is that all terms are finite.
For an equation such as $x = \ct{f}(x)$,
where $f$ is a data constructor,
the solution if it exists must be the infinite term:
\[ x = \ct{f}(\ct{f}(\ct{f}(\ct{f}(\ldots)))) \]
This is not a term that can be written down using our syntax, however,
so we wish to exclude it as a possible solution.

To support these requirements,
a number of axioms are generated for each type.
These are known as the \emph{free equality axioms}.
Consider the type definitions in Figure~\ref{fig:decl-foobar}.
The constants are \sym{a}, \sym{b} and \sym{nil},
and \sym{cons/2} is the only other function symbol.
Note that \sym{foo} and \sym{bar} are not included,
as these are type constructors not data constructors
or declared functions.

\begin{figure}
\begin{verbatim}
    :- type foo ---> a ; b.
    :- type bar ---> nil ; cons(foo, bar).
\end{verbatim}
\caption{Type definitions for the \sym{foo} and \sym{bar} types.
\label{fig:decl-foobar}}
\end{figure}

The free equality axioms for \sym{foo} and \sym{bar}
are shown in Figure~\ref{fig:ax-foobar}.
The two in the first row say that
ground data terms are not equal
unless their principal functors are equal.
We have omitted any axioms that say
terms of different types are not equal,
since in general there is a large number of these,
and in a type correct program
such axioms will not make any real difference.
In a non-typechecked setting,
all $O(n^2)$ axioms would be required.

\begin{figure}
\[
    \lnot (\ct{a} = \ct{b})
    \qquad\quad
    \forall x y.\, \lnot (\ct{nil} = \ct{cons}(x, y))
\]
\[
    \forall x_1 x_2 y_1 y_2.\,
    (\ct{cons}(x_1, y_1) = \ct{cons}(x_2, y_2))
    \rightarrow
    (x_1 = x_2 \land y_1 = y_2)
\]
\begin{center}
$\forall x.\, \lnot (x = \ct{cons}(t_1, t_2))$,
if $x$ occurs in $t_1$ or $t_2$
\end{center}
\caption{Equality axioms for the \sym{foo} and \sym{bar} types.
\label{fig:ax-foobar}}
\end{figure}

The third axiom says that
ground data terms are not equal
unless their arguments are equal.
In other words,
this simply states that
\sym{cons/2} denotes an injective function.

The last line is an axiom schema
(that is, an infinite family of axioms),
where there is one axiom for each pair of data terms $t_1$ and $t_2$.
This schema is known as the \emph{occurs check\label{gi:occurs-check}}%
\footnote{
Or sometimes ``occur check'', without the plural.
},
and it states that a variable is never equal to
a data term that contains that variable.
Equivalently,
a data term never strictly contains itself as a subterm.
This implies that all data terms are finite;
for example the formula $x = \ct{cons}(\ct{a}, x)$,
which would otherwise describe an infinite term,
is always false because $x$ occurs in the term
that it is supposedly equal to.

With the free equality axioms we can no longer say that
ground data terms that are syntactically different may be equal.
For example, the axioms allow us to infer
\[
    \lnot (\ct{cons}(\ct{a}, \ct{nil}) = \ct{cons}(\ct{b}, \ct{nil}))
\]
since if we assume that
$\ct{cons}(\ct{a}, \ct{nil})$
and
$\ct{cons}(\ct{b}, \ct{nil})$
are equal,
then by the third axiom we have that $\ct{a} = \ct{b}$.
This contradicts the first axiom,
which means that our assumption must have been false.

Without the free equality axioms there would have been
no way to derive this proof.


\subsection{Clause soundness axioms}
\label{sec:ax-clause-soundness}

In Section~\ref{sec:partial-correctness}
we gave the clause soundness condition
for partial correctness,
which required that each clause in the program
must be true in the intended interpretation.
This can be expressed as an axiom
in a straightforward way.

Recall from Section~\ref{sec:clauses}
that a clause for a predicate \sym{p/n}
can be mapped to a formula $\psi \leftarrow \phi$,
where $\psi$ corresponds to the clause head
and takes the form $\ct{p}(t_1, \ldots, t_n)$,
for argument terms $t_1, \ldots, t_n$,
and $\phi$ is the formula corresponding to the clause body
(or \true\ if the clause is a fact).
Similarly for a function \sym{f/n},
except that $\psi$ takes the form $\ct{f}(t_1, \ldots, t_n) = t_r$,
where $t_r$ is the return expression.

The formula can be made into an axiom
in the same way mathematical formulas usually are,
by taking the universal closure as follows:
\[ \forall \bar{x}.\, \psi \leftarrow \phi \]
where $\bar{x}$ is the set of free variables
that occur in $\psi$ and $\phi$.

The resulting formula is known as
the \emph{clause soundness axiom}
for the clause.
As suggested above,
it expresses the clause soundness condition for the clause,
and it must be true in the intended interpretation.
Since it is an implication,
the only way it can be false
is if $\psi$ is true and $\phi$ is false.
If this is the case---%
that is, the axiom is false in the intended interpretation---%
then, in line with the discussion in Section~\ref{sec:partial-correctness},
the clause is a wrong answer bug.

Observe that,
while for predicates the axiom asserts
something about ground atoms for the predicate itself,
for functions the axiom asserts
something about the equality relation.
Unlike data constructors,
we do not generate axioms that say
the function is injective,
or that the function returns values
that are different from every other function.
The function completion adds new instances of terms being equal,
rather than excluding such instances.

The axiom also tells us a way to make inferences about the program.
We can choose any way we like
of assigning values to the variables $\bar{x}$,
and if we do so in a way that $\phi$, the clause body,
is something we have already established to be true,
then we can infer that $\psi$, the clause head,
will also be true.
Equivalently,
if we want to know whether or not an instance of $\psi$ is true,
it is sufficient to show that $\phi$ is true
under the same variable assignment.


\subsection{Combined clause soundness axioms}
\label{sec:ax-combined}

In the form given in the previous section,
the clause soundness axiom is convenient
for reasoning about a program one clause at a time.
It is also useful, however,
to be able to reason about the combination of all clauses
that make up a predicate or function definition.

One way to express a combined clause soundness axiom
is to directly conjoin all of the individual axioms,
as follows:
\[
    \forall \bar{x}_1.\, \psi_1 \leftarrow \phi_1
    \land \ldots \land
    \forall \bar{x}_m.\, \psi_m \leftarrow \phi_m
\]
where $m$ is the number of clauses in the definition.
This does not provide us with much additional insight as it stands,
since there are still just as many implications to consider.
If, however,
the clause heads all have identical arguments,
we could easily combine all the conjuncts
into a single implication.

We can make the clause heads take the required form
by performing a simple logical transformation.
For a predicate \sym{p/n},
let $v_1, \ldots, v_n$ be a sequence of variables
that are distinct from any other variables in the clauses.
We define the following formulas:
\begin{eqnarray*}
\psi' & \equiv & \ct{p}(v_1, \ldots, v_n) \\
\phi' & \equiv &
    \exists \bar{x}.\, v_1 = t_1 \land \ldots \land v_n = t_n \land \phi
\end{eqnarray*}
where $\bar{x}$ is the set of free variables
that occur in $\psi$ and $\phi$,
and $t_1, \ldots, t_n$ are the original argument terms.
In other words,
$\psi'$ is $\psi$ with the fresh variables
in place of the argument terms,
and $\phi'$ is $\phi$ conjoined with
equations between each of the fresh variables
and the corresponding argument terms,
and with all variables except the fresh ones existentially quantified.

Analogous definitions can be given for functions,
with the difference being that
we also need a variable, $v_r$, for the function result,
and an additional equation to bind it.
In this case we obtain:
\begin{eqnarray*}
\psi' & \equiv & \ct{f}(v_1, \ldots, v_n) = v_r \\
\phi' & \equiv &
    \exists \bar{x}.\, v_1 = t_1 \land \ldots \land v_n = t_n
    \land v_r = t_r \land \phi
\end{eqnarray*}
where $\bar{x}$ and $t_1, \ldots, t_n$ are as before,
and $t_r$ is the return expression for the clause.

Now consider the formula $\psi' \leftarrow \phi'$.
If we take the universal closure
in the same way as in the previous section,
the result is logically equivalent to
the clause soundness axiom.
Intuitively,
moving argument terms into the body
via equations with fresh variables
does not change the meaning of a clause.
The reason we existentially quantify
variables in the original clause
is because we are only interested in
the values of head variables,
and because when we combine the clauses,
as we will do in a moment,
we do not want variables from different clauses
to clash if they happen to have the same name.
Essentially, the existential quantification
reflects that the scope of variables in a clause
is just that single clause.

From the collection of clauses in a definition,
we obtain $\psi'$, which is common to all the clauses,
and which is implied by $\phi_i'$ for the $i$th clause.
The conjunction of the formulas is found by
taking the disjunction of the antecedents:
\[
    \psi' \leftarrow \phi_1' \lor \ldots \lor \phi_m'
\]
Universally quantifying the free variables, $\bar{v}$,
would give us something equivalent to
the conjunction of all of the clause soundness axioms
for the definition.


\subsection{Clause completeness axioms}
\label{sec:ax-clause-completeness}

Our definition of partial correctness
from Section~\ref{sec:partial-correctness}
requires not just clause soundness
but also clause completeness.
That is,
the set of clauses defining a predicate or function
must, between them, cover every possible ground atom
that is true in the intended interpretation.

The \emph{clause completeness axiom},
which expresses the clause completeness condition,
is the counterpart to the clause soundness axioms.
It can be obtained from
the combined clause soundness axiom from the previous section,
by changing the reverse implication
into a forward implication
prior to universally quantifying.

For a predicate or function defined by $m$ clauses,
the clause completeness axiom is therefore:
\[
    \forall \bar{v}.\, \psi' \rightarrow \phi_1' \lor \ldots \lor \phi_m'
\]
Like the clause soundness axioms,
it must be true in the intended interpretation.
In this case,
the only way it can be false is if the left-hand side is true
and the right-hand side is false.
In line with the discussion in Section~\ref{sec:partial-correctness},
if this happens then the definition contains a missing answer bug.

Intuitively,
putting the clauses into a disjunction
reflects the fact that execution can choose any one of the clauses,
and using a forward implication means that,
while each clause says what things are true,
these are the \emph{only} things that are true.
That is, every ground atom that is true
must be covered by one of the clauses,
as we expect.

We can use the clause completeness axiom
to make inferences about the program,
in much the same way as
we can with the clause soundness axiom.
The difference is that
it gives us \emph{negative} information---%
it allows us to infer that a particular ground atom must be false,
rather than true.
This in turn allows us to reason about
whether the negation of a ground atom is true,
or about which branch will be taken by a conditional goal.


\subsection{Predicate and function completion}
\label{sec:completion}

In most sources,
the clause soundness axioms and the clause compleness axiom
are combined into a single formula.
For a given predicate or function,
this formula,
which is logiclly equivalent to the conjunction of the axioms,
is known as the \emph{completion} of the predicate or function.

Since we already have the axioms in the form of implications
in one direction or the other,
the conjunction of them is just
a bi-implication between the two sides.
In other words, the formula is:
\[
    \forall \bar{v}.\, \psi' \leftrightarrow \phi_1' \lor \ldots \lor \phi_m'
\]
where $\bar{v}$ is the set of fresh variables we introduced.

An example should help to illustrate,
and for this we will turn once again to \sym{append/3}.
The clauses, when mapped to the $\psi \leftarrow \phi$ form,
are as follows
(variable names are chosen to fit with our notational convention):
\begin{IEEEeqnarray*}{l}
\ct{append}([], y, y) \leftarrow \true \\
\ct{append}([w | x], y, [w| z]) \leftarrow \ct{append}(x, y, z)
\end{IEEEeqnarray*}
We obtain the clause soundness axioms by universally quantifying:
\begin{IEEEeqnarray*}{l}
\forall y.\,
    \ct{append}([], y, y) \leftarrow \true \\
\forall w x y z.\,
    \ct{append}([w | x], y, [w | z]) \leftarrow \ct{append}(x, y, z)
\end{IEEEeqnarray*}
To get the combined formula,
we introduce fresh variables $v_1, \ldots, v_3$
and make the following definitions:
\begin{eqnarray*}
\psi' & \equiv & \ct{append}(v_1, v_2, v_3) \\
\phi_1' & \equiv & \exists y.\, v_1 = [] \land v_2 = y \land v_3 = y \\
\phi_2' & \equiv & \exists w x y z.\,
    v_1 = [w | x] \land v_2 = y \land v_3 = [w | z] \land
    \ct{append}(x, y, z)
\end{eqnarray*}
Note that we have omitted \true\ from the conjunctions,
since they do not have any effect.
Putting the definitions together we get the following
clause completeness axiom:
\begin{IEEEeqnarray*}{l}
\forall v_1 v_2 v_3.\, \ct{append}(v_1, v_2, v_3) \rightarrow \\
    \qquad (\exists y.\, v_1 = [] \land v_2 = y \land v_3 = y)\;\lor \\
    \qquad (\exists w x y z.\, v_1 = [w|x] \land v_2 = y \land v_3 = [w|y]
    \land \ct{append}(x, y, z))
\end{IEEEeqnarray*}
The completion of \sym{append/3}
is the same as the clause completeness axiom,
except it uses a bi-implication:
\begin{IEEEeqnarray*}{l}
\forall v_1 v_2 v_3.\, \ct{append}(v_1, v_2, v_3) \leftrightarrow \\
    \qquad (\exists y.\, v_1 = [] \land v_2 = y \land v_3 = y)\;\lor \\
    \qquad (\exists w x y z.\, v_1 = [w|x] \land v_2 = y \land v_3 = [w|z]
    \land \ct{append}(x, y, z))
\end{IEEEeqnarray*}
This is logically equivalent to the conjunction of
the clause soundness and clause completeness axioms
for \sym{append/3}.

To illustrate the procedure for function definitions,
we will of course look at the function \sym{length/1}.
We start with the clauses in their $\psi \leftarrow \phi$ form,
as before:
\begin{IEEEeqnarray*}{l}
\ct{length}([]) = 0 \leftarrow \true \\
\ct{length}([x | y]) = 1 + \ct{length}(y) \leftarrow \true
\end{IEEEeqnarray*}
The clause soundness axioms are therefore:
\begin{IEEEeqnarray*}{l}
\ct{length}([]) = 0 \leftarrow \true \\
\forall x y.\,
    \ct{length}([x | y]) = 1 + \ct{length}(y) \leftarrow \true
\end{IEEEeqnarray*}
No quantifier is required on the first of these,
since there are no free variables.
To get the combined formula,
we introduce fresh variables $v_1$ and $v_r$,
and define:
\begin{eqnarray*}
\psi' & \equiv & \ct{length}(v_1) = v_r \\
\phi_1' & \equiv & v_1 = [] \land v_r = 0 \\
\phi_2' & \equiv & \exists x y.\, v_1 = [x | y] \land v_r = 1 + \ct{length}(y)
\end{eqnarray*}
Putting these together we get the following clause completeness axiom:
\begin{IEEEeqnarray*}{l}
\forall v_1 v_r.\, \ct{length}(v_1) = v_r \rightarrow \\
    \qquad (v_1 = [] \land v_r = 0)\; \lor \\
    \qquad (\exists x y.\, v_1 = [x | y] \land v_r = 1 + ct{length}(y))
\end{IEEEeqnarray*}
As before,
the function completion is the same as the clause completeness axiom
with the implication replaced by a bi-implication.


\subsection{Mode-determinism assertions}
\label{sec:mode-det}

Modes and determinisms play a significant role
in helping people understand Mercury code.
They also play a part in the declarative semantics.
For modes that are \co{det} or \co{multi}
we generate an axiom that says that
for every value of each of the inputs,
there exists a solution.
For modes that are \co{det} or \co{semidet}
we generate an axiom that says that
for every value of each of the inputs,
there is at most one solution.

For example, consider the following modes for \sym{append/3}:
\begin{verbatim}
:- mode append(in, out, in) is semidet.
:- mode append(out, out, in) is multi.
:- mode append(in, in, out) is det.
\end{verbatim}
These three modes will generate
the three axioms shown in Figure~\ref{fig:mode-det}, respectively.
Note that the third axiom, for the \co{det} mode,
is the conjunction of the axioms that would apply
for the \co{semidet} and \co{multi} modes.

\begin{figure}
\begin{IEEEeqnarray*}{l}
\forall x y_1 y_2 z.\,
    \ct{append}(x, y_1, z) \land \ct{append}(x, y_2, z)
    \rightarrow y_1 = y_2
\\
\forall z. \exists x y.\, \ct{append}(x, y, z)
\\
\forall x y. (\exists z.\, \ct{append}(x, y, z))~\land \\
    \qquad (\forall z_1 z_2.\,
        \ct{append}(x, y, z_1) \land \ct{append}(x, y, z_2)
        \rightarrow z_1 = z_2)
\end{IEEEeqnarray*}
\caption{Mode-determinism assertions for three modes of \sym{append/3}.
\label{fig:mode-det}}
\end{figure}

One consequence of these axioms is that
if a predicate has a mode where all arguments are inputs
and where the determinism says it cannot fail,
then any call to that predicate is logically equivalent to \true.
Unless a strict operational semantics is used,
be aware that in most cases the compiler will optimize away such calls.
If that happens then exceptions may not be thrown,
trace goals may not be run, etc.

Similar axioms are generated for function modes that cannot fail.
The axiom generated for a function's default mode
states that the function is total---%
a vacuous statement in classical logic
since function symbols are always interpreted as total functions.
On the other hand,
this presents a problem for functions
that are \co{semidet} in the forward mode,
that is, the mode with all arguments as inputs
and the return value as an output.
These denote \emph{partial} functions,
which cannot be directly represented in classical logic.
We will discuss in Section~\ref{sec:partial}
how these function modes are handled.


\section{Classical semantics}
\label{sec:classical}

\subsection{Universes}

A \emph{universe} $U$ is a non-empty set of \emph{values\label{gi:value}}
representing the domain of discourse%
\footnote{
The term ``domain'' is also sometimes used instead of universe,
but this is not quite the same as the domains that sometimes appear
in the denotational semantics of other languages---%
there is no $\bot$ element, for example---%
so we will avoid using this term.
}.
The idea is that terms in the syntax
correspond to values in $U$.
If a term $t$ corresponds to a value $u$
in a given interpretation,
we say that $t$ denotes $u$,
and that $u$ is the denotation of $t$.
For example,
the code in Section~\ref{sec:decl-debug}
implementing arbitrary precision integers as lists of digits
could have the set of all integers as its universe.
In this interpretation,
each list of digits would denote an integer.

The examples in Section~\ref{sec:first-examples}
made use of Herbrand interpretations.
In these the universe is the
\emph{Herbrand universe\label{gi:herbrand-universe}},
which is defined as the set of all ground data terms,
and each ground data term simply denotes itself.
For first order code
the Herbrand universe can be considered as good as any other,
since, given an interpretation in any other universe $U$,
there exists a unique map
from the Herbrand universe to $U$
that ultimately leads to the same result.

In the following discussion,
for a universe $U$
we will make use of the sets $U^n$ for $n \geqslant 0$,
where $U^n$ is defined as the set of $n$-tuples of elements in $U$.
These sets represent the possible argument values
for predicates and functions of arity $n$.


\subsection{Assignments}
\label{sec:assignments}

An \emph{assignment} over a universe $U$
is a mapping from variables to values in $U$.
In the following
we will use $\sigma$ and $\rho$ to stand for arbitrary assignments.
For a variable $x$,
the value that it maps to under $\sigma$ is written as $\sigma(x)$.

If $\sigma_1$ and $\sigma_2$ are assignments such that
$\sigma_1(v) = \sigma_2(v)$
for all variables $v$ other than $x$,
then we say that $\sigma_1$ differs from $\sigma_2$ only at $x$.
Note that it can be the case that
$\sigma_1(x)$ does still equal $\sigma_2(x)$.
We write $\sigma \{ v_1 \mapsto u_1, v_2 \mapsto u_2, \ldots \}$,
or just $\sigma \{ v_i \mapsto u_i \}$,
for the assignment that differs from $\sigma$ only at each of the $v_i$,
where it maps to $u_i$.

It's possible to imagine applying an assignment to formula,
such that all of the free variables in the formula
are replaced by the values they take in the assignment.
If such an assignment makes the formula true,
then the assignment is thought of as a
\emph{solution\label{gi:solution}}.
In the next two sections we will make this concept precise.


\subsection{Interpretations}
\label{sec:interpretations}

We have used the term ``interpretation'' a number of times to mean,
intuitively, the way in which we understand
the symbols appearing in a formula.
For example,
we have said that the symbol \co{+} can be interpreted as integer addition.
The general idea of an interpretation is that
it maps from syntactic elements---the predicate and function symbols---%
to some semantic universe.

Formally, an interpretation $I$ over a universe $U$
is a mapping defined on predicate and function symbols, such that:
\begin{itemize}
\item
If \sym{f/n} is a function symbol,
then $I(\ct{f})$ is a total function $U^n \rightarrow U$.
In other words,
$I(\ct{f})$ takes $n$ arguments from $U$
and returns a value from $U$.
For a constant \sym{a},
$I(\ct{a})$ is just an element of $U$.
\item
If \sym{p/n} is a predicate symbol,
then $I(\ct{p})$ is a predicate (that is, a relation) over $U^n$.
That is,
$I(\ct{p})$ takes $n$ arguments from $U$
and returns either \true\ or \false.
\end{itemize}
Thus,
the function $I(\ct{f})$
is the denotation of the function symbol \sym{f/n},
and the predicate $I(\ct{p})$
is the denotation of the predicate symbol \sym{p/n}.

More generally,
we will need to show how an interpretation as defined above
can be extended to a mapping on terms and formulas.
Since these may contain variables,
the result will depend on how values are assigned to those variables.

Given an interpretation $I$ and an assignment $\sigma$,
we can extend $I$ to a mapping $I_\sigma$
from terms to elements of $U$
by applying the following rules:
\begin{itemize}
\item
If $x$ is a variable, then $I_\sigma(x)$ maps to $\sigma(x)$.
\item
If \sym{f/n} is a function symbol and $t_1, \ldots, t_n$ are terms,
then $I_\sigma(\ct{f}(t_1, \ldots, t_n))$ maps to
the function $I(\ct{f})$ applied to arguments
$I_\sigma(t_1), \ldots, I_\sigma(t_n)$.
For a constant \sym{a}, $I_\sigma(\ct{a})$ just equals $I(\ct{a})$.
\end{itemize}
Similarly,
this increasingly overloaded mapping
can be extended to atomic formulas.
The following rules are applied:
\begin{itemize}
\item
$I_\sigma(\true)$ always maps to \true.
\item
$I_\sigma(\false)$ always maps to \false.
\item
If $t_1$ and $t_2$ are terms,
then $I_\sigma(t_1 = t_2)$ maps to \true\ if
$I_\sigma(t_1)$ and $I_\sigma(t_2)$ map to the same value in $U$.
Otherwise it maps to \false.
\item
If \sym{p/n} is a predicate symbol and $t_1, \ldots, t_n$ are terms,
then $I_\sigma(\ct{p}(t_1, \ldots, t_n))$ maps to
the predicate $I(\ct{p})$ applied to arguments
$I_\sigma(t_1), \ldots, I_\sigma(t_n)$.
\end{itemize}
Continuing,
the mapping can be extended to compound formulas
by applying the following rules:
\begin{itemize}
\item
If $\phi$ is a formula constructed using a logical connective,
then $I_\sigma(\phi)$ maps to a truth value
via the usual (classical) truth table for that connective,
using the truth values of $I_\sigma$ for each sub-formula.
\item
If $\phi$ is of the form $\exists x.\, \psi$,
then $I_\sigma(\phi)$ maps to \true\ if $I_\rho(\psi)$ is true
for some assignment $\rho$ that differs from $\sigma$ only at $x$.
Otherwise it maps to \false.
\item
If $\phi$ is of the form $\forall x.\, \psi$,
then $I_\sigma(\phi)$ maps to \true\ if $I_\rho(\psi)$ is true
for all assignments $\rho$ that differ from $\sigma$ only at $x$.
Otherwise it maps to \false.
\end{itemize}
Finally, observe that if $\phi$ is a closed formula
then $I_\sigma(\phi)$ is always the same
regardless of the assignment.
We can therefore define $I(\phi)$,
without ambiguity,
as being equal to $I_\sigma(\phi)$
where $\sigma$ is an arbitrary assignment.

For example,
consider an interpretation $I$ in which
\sym{append/3} is the list append predicate
and \sym{length/1} is the list length function.
Let $\sigma$ be an assignment such that $\sigma(z) = [1, 2, 3]$,
and let $\phi$ be the following formula.
\[
\exists x y.\, \ct{append}(x, y, z) \land \ct{length}(x) = 2
\]
We can evaluate $I_\sigma(\phi)$ as follows.
Let $\rho = \sigma \{ x \mapsto [1, 2], y \mapsto [3] \}$.
That is, $\rho$ is the assignment
that differs from $\sigma$ only at $x$ and $y$,
where it takes the values $[1, 2]$ and $[3]$, respectively.
We have that $I_{\rho}(x)$ maps to $[1, 2]$,
therefore $I_{\rho}(\ct{length}(x))$
is the length of the list $[1, 2]$, which is 2.
Thus $I_{\rho}(\ct{length}(x) = 2)$ maps to \true.

Similarly,
$I_{\rho}(\ct{append}(x, y, z))$ maps to \true,
since the arguments map to
$[1, 2]$, $[3]$ and $[1, 2, 3]$, respectively,
and the last of these is the result of
appending the first two.
Given this result and the result from the previous paragraph,
we can see that
$I_{\rho}(\ct{append}(x, y, z) \land \ct{length}(x) = 2)$
maps to \true,
via the truth table for~`$\land$'.

Finally, $I_\sigma(\phi)$ maps to \true\ because
$\rho$ is an assignment that differs from $\sigma$
only at $x$ and $y$,
which satisfies our rule for the existential quantifier.

Had we used an assignment $\sigma'$ with $\sigma'(z) = [1]$,
then the append operation would have been false\ under
the assignment $\rho' = \sigma' \{ x \mapsto [1,2], y \mapsto [3] \}$,
since $[1, 2]$ and $[3]$ do not append to form $[1]$.
Furthermore,
no such assignment $\rho'$ would be able to make
the interpretation of the append operation true,
and still have $I_{\rho'}(\ct{length}(x))$ being true.
As such, $I_{\sigma'}(\phi)$ would have evaluated to \false.


\subsection{Models}
\label{sec:models}

An interpretation can be thought of as
one individual programmer's understanding of
how a program works.
If the programmer has an accurate picture in their head
of the observable behaviour---the inputs and outputs---%
then their interpretation can be said to be a
\emph{model\label{gi:model}}.

Formally,
let $I$ be an interpretation over the universe $U$.
We say that $I$ is a model of a set of axioms
if each of the axioms evaluates to true
in that interpretation
(recall that axioms are closed formulas,
so we do not need to specify an assignment).
If $I$ is a model of the axioms generated by a program,
then we say that $I$ is a model of the program.
Customarily, the variable $M$ is used rather than $I$
when discussing an interpretation that is a model.

We are now in a position to give the following.

\begin{definition}[Declarative semantics]
\label{def:declarative-semantics}
The declarative semantics of a Mercury program
is the collection of models of that program.
\end{definition}

\noindent
In other words,
the declarative semantics is essentially
all the possible ways of thinking about the program
which accurately reflect how the program behaves.

Considering once again
the arbitrary precision integers example
that we saw in Section~\ref{sec:decl-debug},
one programmer might interpret the digit lists as integers directly.
Another programmer might interpret them as
lists of integers that will produce
the actual integers oncce a particular function is applied.
As long as their individual interpretations as a whole
accurately reflect the program,
then both programmers stand in equally good positions
from which to make valid arguments about the program.
Our definition of the declarative semantics as a set of models
reflects this fact.

If there exists any model at all,
then there are an infinite number of possible models.
However, we generally do not have to consider all models
as it is possible to get the same results
by considering only a particular set of interpretations.
For first-order code,
it is sufficient to only consider models
that are Herbrand interpretations.

Even so, there can be multiple Herbrand interpretations
that are models of a program.
For example, consider the program \co{p :- p.}
The completion of this program is $\ct{p} \leftrightarrow \ct{p}$,
which is a tautology.
This means that it is true in every interpretation,
specifically,
\sym{p} could be assigned the value \true\ or the value \false,
but in either case the axiom would hold
so both of these interpretations are models.

More concerning is the program \co{p :- not p.}
In this case the completion is $\ct{p} \leftrightarrow \lnot \ct{p}$,
which is a contradiction.
This means that it is false in every interpretation,
which means that \emph{there are no models}.
Effectively, the behaviour of the entire program is undefined,
at least in the classical semantics.
This can be regarded as a moot point, however,
since execution of a program that calls such a predicate would not terminate.
Nonetheless, it motivates the following definition.

\begin{definition}[Consistency]
\label{gi:consistent}
A theory is consistent if it contains no contradictions.
That is,
there is no formula $\phi$ such that
both $\phi$ and $\lnot\phi$ are contained in the theory.
\end{definition}

\noindent
If there is at least one model of a theory
then there cannot be any contradictions,
so by this definition the theory must be consistent.

In the remainder of this guide we will assume
we are working with a program
whose axioms we denote by $\Gamma$.
We assume that there is at least one model of $\Gamma$,
so the program is consistent,
We will say ``model'' to mean a model of $\Gamma$.
With that in mind we give the following definitions.

\begin{definition}[Satisfaction]
Let $\phi$ be a formula.
If $M$ is a model and $\sigma$ is an assignment
such that $M_\sigma(\phi)$ is true,
we say that $M$ \emph{satisfies $\phi$ under $\sigma$},
and that $\phi$ is true in $M$ under $\sigma$.
If $\phi$ is closed,
we additionally say that $M$ satisfies $\phi$,
and that $\phi$ is true in $M$.
If there exists any model that satisfies $\phi$ under some assignment,
we say that $\phi$ is \emph{satisfiable\label{gi:satisfiable}}.
\end{definition}

\begin{definition}[Validity]
Let $\phi$ be a formula and $\sigma$ an assignment.
We say that $\phi$ is \emph{valid\label{gi:valid} under $\sigma$},
written `$\Gamma, \sigma \models \phi$',
if every model satisfies $\phi$ under $\sigma$.
If this holds for every possible assignment,
in particular if $\phi$ is closed,
we additionally say that $\phi$ is valid,
and that $\phi$ is a \emph{logical consequence} of\, $\Gamma$.
This is written as $\Gamma \models \phi$.
\end{definition}

\begin{definition}[Unsatisfiability]
Let $\phi$ be a formula and $\sigma$ an assignment.
We say that $\phi$ is
\emph{unsatisfiable\label{gi:unsatisfiable} under $\sigma$}
if there is no model $M$ that satisfies $\phi$ under $\sigma$.
If this holds for every possible assignment,
in particular if $\phi$ is closed,
we additionally say that $\phi$ is unsatisfiable.
\end{definition}

\noindent
Returning to the definition of solution that we gave earlier,
we can say that an assignment $\sigma$
is a solution\label{gi:solution2} to a formula $\phi$
if and only if $\Gamma, \sigma \models \phi$.
That is,
$\sigma$ is a solution for $\phi$
if every model satisfies $\phi$ under $\sigma$.

With the above definitions,
no formula can be both valid and unsatisfiable.
Furthermore, if a formula is valid
then its negation is unsatisfiable,
and vice-versa.
Not all formulas are one or the other, however:
if a formula is true in some models and false in others,
then it is neither valid nor unsatisfiable.
For example,
consider again the program \mbox{\co{p :- p}}.
Because there is a model in which \sym{p} is true
and another model in which \sym{p} is false,
the formula \sym{p} is satisfiable but not valid.
A formula like this that is neither valid nor unsatisfiable
is said to be \emph{contingent\label{gi:contingent}}.

A point about the $\models$ notation is worth underlining.
We have used this symbol to denote the validity relation between
sets of axioms, $\Gamma$, and closed formulas, $\phi$,
and the ``valid under'' relation between
sets of axioms, assignments, and open formulas.
This symbol is commonly overloaded, in other ways, too.
Some authors use it to denote
the satisfaction relation between \emph{models} and closed formulas,
or between models, assignments, and open formulas.
Other authors use it to denote
the modelling relationship between interpretations and sets of axioms.
That is, statements in the following forms may appear:
\[
M \models \phi
\qquad\qquad
M, \sigma \models \phi
\qquad\qquad
M \models \Gamma
\]
These mean, respectively,
that $M$ satisfies $\phi$,
that $M$ satisfies $\phi$ under $\sigma$,
and that $M$ is a model of the set of axioms, $\Gamma$.

We will only need to use the forms given in our definition of validity,
but the reader should be aware that
the other forms may be used in other sources.


\section{Example}
\label{sec:reasoning}

Now that we have defined our semantics
and specified what axioms generate a program's theory,
how do we actually come up with a theorem?
Theorems require proofs,
so in this section we will give a couple of ad hoc arguments
to try to prove that a formula is true.

Consider the formula $\ct{append}([1], [2], [1, 2])$.
We already know it is true
given that \sym{append/3} is interpreted as list concatenation,
but assume that we only know how it is defined,
and not its interpretation.
Can we prove it is true using just the axioms that are generated?

\begin{figure}
\begin{IEEEeqnarray*}{l}
\forall y.\,
    \ct{append}([], y, y) \leftarrow \true \\
\forall w x y z.\,
    \ct{append}([w | x], y, [w | z]) \leftarrow \ct{append}(x, y, z)
\end{IEEEeqnarray*}
\caption{Clause soundness axioms for \sym{append/3}.\label{fig:ax-append}}
\end{figure}

For convenience,
the clause soundness axioms for \sym{append/3}
from Section~\ref{sec:completion}
are repeated in Figure~\ref{fig:ax-append}.


\subsection*{First attempt}
\label{sec:reasoning1}

One approach to proving our formula
is to try to determine directly
which ground atoms are contained in
some arbitrary Herbrand model of the program,
which we will call $H$.
If the formula we are interested in
is contained in the model,
then it must be true.

Looking at the axiom for the first clause,
we can see that if we assign the value $[2]$ to $y$,
then the formula becomes:
\[
    \ct{append}([], [2], [2]) \leftarrow \true
\]
From this we immediately get that
$\ct{append}([], [2], [2]) \in H$.
We are allowed to choose, as we did,
any value we like for $y$,
since $y$ is universally quantified---%
the axiom is applicable to all possible choices.

Now consider the axiom for the second clause.
If we assign the values $[]$, $[2]$ and $[2]$
to the variables $x$, $y$ and $z$, respectively,
then the right hand side becomes $\ct{append}([], [2], [2])$,
which we just established is in $H$.
If we then assign to $w$ the value $1$,
the formula becomes:
\[
    \ct{append}([1], [2], [1, 2]) \leftarrow \ct{append}([], [2], [2])
\]
Thus we get that $\ct{append}([1], [2], [1, 2]) \in H$,
which completes our proof.

Our attempt at proving a formula has been successful,
but there are a lot of ways to go about building a proof,
and the proof we have arrived at has a drawback.
It starts by proving a small fact via the first clause,
then builds a larger fact via the second clause.
Such a proof is known as a ``bottom-up'' proof.
This is an interesting way of reasoning,
and in fact it reflects, to an extent,
how deductive databases go about answering queries.
but it does not reflect how Mercury programs are executed.

Mercury execution starts with a high level goal,
and reduces it down to smaller and smaller parts
until each part can be solved directly using a fact clause.
This style of proof is known as ``top-down'',
in contrast to the bottom-up proof above.
We will make a second attempt at proving our formula,
this time using a top-down approach.
While this will not completely describe how execution proceeds,
it should help provide some intuition,
and serve to motivate the operational semantics
that will be the subject of the next chapter.


\subsection*{Second attempt}
\label{sec:reasoning2}

For our second attempt we will try a different strategy.
We will start by assuming that the formula we wish to prove,
$\ct{append}([1], [2], [1, 2])$, is false.
From that we will try derive a contradiction,
which would show our assumption to be false,
and thus prove that the formula is true.
In other words,
this will be a proof-by-contradiction.
We refer again to the axioms in Figure~\ref{fig:ax-append}.

Our approach will be to choose one of the axioms
and try to match it to our formula
by setting the arguments appropriately.
Since the axioms are reverse implications
for which we have assumed the left-hand side is false,
we can infer that the right-hand side must also be false.
By performing this inference a number of times in sequence
we hope to reach a contradiction,
that is, where the right-hand side is in fact true.
We might fail to find such a contradiction, however.
If we instead reach a tautology,
it means we have failed to find the required contradiction.
If we want to keep searching,
we will need to go back to an earlier choice of axiom that we made,
and choose the other axiom instead.

The predicate call we are interested in
has argument values $[1]$, $[2]$ and $[1, 2]$.
If we choose the axiom for the first clause,
we find that there is no way to assign a value to $y$
such we can match the argument values,
because the first argument will always be $[]$.

We therefore choose the axiom for the second clause.
By assigning the values
$1$, $[]$, $[2]$ and $[2]$
to $w$, $x$, $y$ and $z$, respectively,
we get:
\[
    \ct{append}([1], [2], [1, 2]) \leftarrow \ct{append}([], [2], [2])
\]
We have assumed the formula on the left-side side to be false,
so the formula on the right-hand side must also be false.

Trying the first axiom on this new formula,
we can assign the value $[2]$ to $y$ to obtain:
\[
    \ct{append}([], [2], [2]) \leftarrow \true
\]
Again,
the left-hand side is false
so the right-hand side must also be false,
but this is a contradiction
because the right-hand side is \true.
We can therefore conclude that
our original assumption must have been false.
Thus $\ct{append}([1], [2], [1, 2]) \in H$,
which completes our proof.

This style of proof more closely reflects how programs are executed.
It is still missing one important thing, however,
which is that there were no variables in the formula we proved.
Programs in general have output variables
which become bound in the course of execution,
so our proof technique would need to take account of that.
We also want to be able to perform the inferences
in a way that is less ad hoc.
That, ultimately, is the aim of the operational semantics.

We will cover the operational semantics in detail
in the next chapter.
Before that, however,
we will make some brief philosophical remarks
regarding classical logic.


\section{Philosophical remarks}
\label{sec:philosophy}

The semantics we have presented is classical.
It is possible this will leave the reader with the impression
that Mercury is for classicists
and Mercury programmers must therefore
believe that classical logic is the One True Logic.
This impression would not be accurate,
as a lot of consideration has gone into understanding
the benefits and drawbacks of classical logic,
and likewise in understanding what other logics have to offer.

The motivation for the focus on classical logic is straightforward:
it provides an excellent trade-off between ease of reasoning
and ability to observe a broad class of bugs.
Its use does not, of course,
preclude the addtional use of non-classical logic.
In Chapter~\ref{sec:non-classical} we give
a non-classical meaning to the logical connectives
that enables more realistic reasoning about programs
and their correctness with respect to a specification.
As such, it demonstrates the usefulness of ``logical pluralism''
as a philosophy for reasoning about computer programs.

The underlying purpose of logic is to characterize how we think,
not to tell us how to think.
We humans have the innate ability to judge
between a good argument and a bad argument,
and the best logic in a given situation
is the one that most accurately reflects
the way in which we exercise this ability.
