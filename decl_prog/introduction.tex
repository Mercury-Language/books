\chapter{Introduction}
\label{sec:intro}

\section{Purpose}
\label{sec:purpose}

In the Formal Semantics chapter of the Mercury Language Reference Manual,
the declarative semantics of Mercury is given in a single paragraph.
Readers with sufficient background in logic programming
would find the definition familiar:
a predicate calculus theory
with a ``language'' specified by the type declarations in the program,
and a set of axioms derived from the ``completion'' of the program.
Readers without that background, however,
are likely to find this paragraph too opaque to be helpful.

The case is similar with the operational semantics,
which is defined with reference to ``SLDNF resolution''.
The vast majority of people who know what this means
also already know logic programming,
so this definition is not helpful
for those for whom Mercury is their first logic programming language.

The challenge for readers is particularly difficult
since existing resources on the predicate calculus
tend to come in two forms:
\begin{enumerate}
\item
Those that focus on logic as it pertains to logic programming%
\footnote{
For example,
Apt, K.R., 1997. \textit{From logic programming to Prolog}.
London: Prentice Hall.
}.
While these do a good job at connecting the logic
with an operational semantics
(that is, giving the logic a computational interpretation)
there is often relatively little focus on the completion semantics,
which is how Mercury is defined.
\item
Those that focus on classical logic in its own right%
\footnote{
For example,
\href{https://plato.stanford.edu/entries/logic-classical/}
{https://plato.stanford.edu/entries/logic-classical/}.
}.
While these generally offer a more complete picture of the logic,
they do not usually discuss resolution,
which is the computational mechanism used in logic programming.
In addition, the level of mathematical rigor, while important,
can obscure the issues most relevant to logic programming.
\end{enumerate}
This guide aims to bridge the gap between theory and practice.
It is intended for programmers who have some knowledge of Mercury
and want a deeper understanding,
but who are unable to derive much practical information
from the resources currently available.
It is not presented with the same level of rigor
as many other articles on this topic,
for example, proofs are not provided for our claims.
Rather, the intent is to put programmers in a better position
to make the most practical use of existing resources.

\section{Mercury programming in a nutshell}
\label{sec:nutshell}

The main theme of this guide will be to show
the parallels between syntax and semantics,
of which there are many.
By \emph{syntax} we mean
the sequences of characters that constitute part or all of a program.
The word \emph{semantics} means ``meaning'',
but it also has a technical definition
in the context of programming languages,
which is that a program semantics
constrains the program's behaviour
with respect to a particular set of observables.

\begin{figure}
\setlength{\unitlength}{0.01\textwidth}
\begin{center}
\begin{picture}(95,45)(0,10)
\put(10,50){\textsc{requirements}}
\put(70,50){\textsc{outcomes}}
\put(33,51){\vector(1,0){33}}
\put(38,52){\textit{user expectations}}
\put(20,48){\vector(0,-1){14}}
\put(76,34){\vector(0,1){14}}
\put(13,30){\textsc{formulas}}
\put(69,30){\textsc{solutions}}
\put(30,31){\vector(1,0){35}}
\put(34,32){\textit{declarative semantics}}
\put(21,28){\vector(0,-1){14}}
\put(75,14){\vector(0,1){14}}
\put(16,10){\textsc{goals}}
\put(69,10){\textsc{answers}}
\put(29,11){\vector(1,0){36}}
\put(34,12){\textit{operational semantics}}
\end{picture}
\end{center}
\caption{Mercury programming in a nutshell.\label{fig:nutshell}}
\end{figure}

Figure~\ref{fig:nutshell} gives
a conceptual view of the programming process in Mercury.
At a high level,
a programmer is given requirements,
and in some way or other they need to generate outcomes
in accordance with user expectations.
They formulate the requirements logically,
and with this formulation
they can use the declarative semantics
to determine what solutions---%
assignments of values to variables---%
arise as a consequence.
These solutions are then interpreted
in terms of the original problem domain,
to generate outcomes that (hopefully) satisfy the user.

At a lower level,
the programmer's mental formulation
is expressed as goals in Mercury.
The compiler and runtime system compute answers to the goals
in a manner determined by the operational semantics,
and these answers can in turn be
understood by the programmer as solutions to the formulas.

These two levels of reasoning,
the first more abstract and the second more concrete,
are represented by the two horizontal arrows
in the middle and in the lower part of the diagram.
At each level there is a syntax to express the ideas
and a semantics to reason about what they mean.
A close correspondence exists between the two,
in that they place the same constraints
on a program's observed behaviour.

The difference between the two levels of reasoning,
and the reason we would want to consider
having two distinct levels in the first place,
comes down to how they are defined.
The declarative semantics is defined in terms of semantic concepts
understood by the programmer,
and aims to characterize the programmer's mental picture
of how a program behaves.
On the other hand,
the computer does not have such a mental picture---%
it blindly manipulates symbols
without understanding how the programmer will interpret the results---%
so the operational semantics aims to characterize
the program's behaviour as symbolic manipulation.
Thus, the operational semantics is defined
in terms of the syntax.

It is the declarative and operational semantics,
and the correspondence between them,
that is the principal subject of this guide.


\section{Notation}
\label{sec:notation}

With the dual syntax vs. semantics view in mind,
we adopt a kind of parallel notation and terminology in this guide.
When discussing program elements
from a primarily syntactic or operational point of view,
we will use Mercury syntax written in \verb#this# font,
whereas when the discussion is from a semantic or declarative point of view
we will use conventional mathematical notation in normal font.
Similarly,
the terminology used differs between the two views,
with terms that apply to the declarative view
having their counterparts in the operational view.
Figure~\ref{fig:notation} shows some examples of notation and terminology.

\begin{figure}
\begin{center}
\begin{tabular}{l@{\hspace{4em}}l}
\bf{Semantic/declarative} & \bf{Syntactic/operational} \\[1em]
variables: & variables: \\
$\quad x$ & \verb#   X# \\
$\quad y_1, \ldots, y_n$ & \verb#   Y1, ..., YN# \\[1em]
values: & ground data terms: \\
$\quad 123$ & \verb#   123# \\
$\quad \ct{f}(\ct{a}, \ct{g}(\ct{b}))$ & \verb#   f(a,g(b))# \\
$\quad \ct{a}_1, \ldots, \ct{a}_n$ & \verb#   a1, ..., aN# \\[1em]
atomic formulas: & atomic goals: \\
$\quad y = \ct{f}(x)$ & \verb#   Y = f(X)# \\
$\quad \ct{p}(\ct{t}_1, \ldots, \ct{t}_n)$ & \verb#   p(t1, ..., tN)# \\[1em]
logical connectives: & operators: \\
$\quad \land$ & \verb#   ,# \\
$\quad \lor$ & \verb#   ;# \\
$\quad \leftarrow$ & \verb#   :-# \\
\end{tabular}
\end{center}
\caption{
Examples of the parallel notation we will use.
The elements themselves will be discussed in later chapters.
\label{fig:notation}
}
\end{figure}

We hope that this use of notation will help readers.
However, while the distinction it conveys
between the semantic/declarative view on the one hand
and the syntactic/operational view on the other hand is often useful,
in some cases, it is not.
Indeed, we will shortly introduce so-called Herbrand interpretations,
in which elements of syntax are used directly
as elements of the semantic domain.


\section{Outline of the guide}
\label{sec:outline}

The outline of the remainder of this guide is as follows.

In Chapter~\ref{sec:by-example} we give
an informal picture of the declarative semantics,
with a focus on some simple examples.
We aim to give a basic idea of
what is meant by declarative semantics,
and also discuss some of the advantages that can be obtained
by thinking about Mercury programs in this way.
This provides our motivation for
wanting a declarative semantics.

In Chapter~\ref{sec:fopc} we define
the syntax and semantics of first-order predicate calculus,
and show how the declarative semantics of a Mercury program
is expressed in a predicate calculus theory.
We give some examples of logical reasoning
that can tell us how our programs behave.

In Chapter~\ref{sec:op-sem}
we give abstract algorithms for unification and resolution,
and use these as building blocks to define the operational semantics.
We also define the negation-as-failure rule
that is used to implement negation and if-then-else.
We then give some important results in the meta-theory,
which we use to show the correspondence between
the operational and the declarative viewpoints.

In Chapter~\ref{sec:exec}
we give concrete details of how the implementation
behaves at run-time.
These details should enable programmers to better estimate
how different coding choices affect
such operational characteristics of their programs
as stack and heap usage.

In Chapter~\ref{sec:extensions} we extend our work to cover aspects of Mercury
which are not well-characterized in the classical semantics,
such as partial functions and exceptions.

In Chapter~\ref{sec:non-classical} we present
a non-classical interpretation of Mercury programs
that is useful for writing specifications
and checking that they are correctly implemented.
This interpretation demonstrates that
classical logic is not the only logic
that can be usefully applied to understanding Mercury programs.

Finally,
the glossary index in Appendix~\ref{sec:glossary}
provides short definitions,
as well as page references,
for many of the concepts discussed in the guide.
